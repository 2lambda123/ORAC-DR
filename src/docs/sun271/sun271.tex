\documentclass[twoside,11pt,nolof]{starlink}

\stardoccategory    {Starlink User Note}
\stardocinitials    {SUN}
\stardocsource      {sun\stardocnumber}
\stardoccopyright   {Copyright \copyright\ 2021 East Asian Observatory}
\stardocnumber      {271.0}
\stardocauthors     {G. S. Bell}
\stardocdate        {April 2021}
\stardoctitle       {Wesley --- Pre-processing \\ mode for ORAC-DR}
\stardocversion     {Version 1.0.0}
\stardocmanual      {User's Guide}
\stardocabstract  {
  \wesley{} is a pre-processing pipeline.  It is intended to
  apply corrections to raw data files which may be necessary
  prior to reduction with \oracdr{}.
}

\stardocname  {\stardocinitials /\stardocnumber}

\begin{document}
\scfrontmatter

\section{\xlabel{introduction}Introduction\label{se:intro}}

The \oracdr{} pipeline is a suite of recipes and primitives
for the automated processing of raw instrument data into
scientifically-usable products.
However for some observations there may be
issues with the raw data files which prevent
them from being able to be processed appropriately.

Wesley is a pre-processing pipeline
(actually a special mode of \oracdr{})
designed to apply corrections to problematic raw data.
The complete data reduction process
can therefore be envisioned as follows:

\begin{enumerate}
\item \wesley

Pre-processing (when necessary).

\item \oracdr

Main data reduction.

\item \picard

Further analysis and combination of data (if desired).
\end{enumerate}

\newpage
\section{\xlabel{wesley}Wesley Overview\label{se:wesley}}

\subsection{Running Wesley}

Wesley makes use of your current ORAC-DR environment to configure
an instrument and locate files.
Therefore before running Wesley, you should first set up ORAC-DR
as normal.
Then you can use any of ORAC-DR's options to specify observations
for pre-processing.
Wesley will write a list of the pre-processed files which can then
be used with ORAC-DR's \texttt{-{}-files} option.
The name of this file listing is normally automatically generated,
and reported by Wesley at the end of processing each observation,
but it can be specified via the recipe parameter \texttt{WESLEY\_FILE\_LIST}.
For example, observation 20 of the current night can be pre-processed
and then reduced as follows:

\begin{terminalv}
$ oracdr_scuba2_850 --cwd
$ wesley --list 20 --recpars="WESLEY_FILE_LIST=preproc.lis" INSERT_JCMT_WVM_DATA
$ oracdr --files preproc.lis
\end{terminalv}

\subsection{Wesley Options}

Wesley accepts the same command line options as ORAC-DR
(see \oracdrsun{} for more information).
However it is always necessary to specify the recipe name.

Some common command line options are as follows:

\begin{description}
\item[-{}-log sf] \mbox{}

Write text output to the terminal (\texttt{s}) and to a log file
(\texttt{f}).
Other options are an X-window (\texttt{x}) or an HTML log file (\texttt{h}).

\item[-{}-nodisplay] \mbox{}

Do not open graphical display windows.

\item[-{}-recpars file\_name] \mbox{}

Recipes requiring additional information can be controlled via
a recipe parameters file, in INI format with one block per recipe name.

Parameters can also be given directly in place of a file name,
for example:
\\ \texttt{-{}-recpars="JCMT\_WVM\_FILE=wvm.txt,WESLEY\_FILE\_LIST=out.lis"}.

\end{description}

\newpage
\appendix

\section{\xlabel{ap_list}Alphabetical List of Wesley Recipes\label{ap:list}}

\begin{description}
\item[FILTER\_DOME\_OPEN] \mbox {}
Filter file list by dome status.

\item[FIX\_HEADER\_IFFREQ] \mbox {}
Set IFFREQ header from OCS config XML.

\item[FIX\_INCONSISTENT\_OBJECT] \mbox {}
Set OBJECT header to first value from group.

\item[INSERT\_JCMT\_WVM\_DATA] \mbox{}
Put WVM data into raw JCMT files.
\end{description}

\newpage

\section{\xlabel{ap_full}Specifications of Wesley Recipes\label{ap:full}}

The following pages describe the current \wesley{} recipes in detail.

\input{mainrecipes}

\end{document}
