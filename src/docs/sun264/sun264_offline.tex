\section{\xlabel{offline}Offline processing of SCUBA-2 data\label{se:offline}}

\subsection{Running the science pipeline at your home institution}

If the data for your project have been downloaded from CADC and placed
in a single directory, the easiest procedure is to create a text file
containing the name of each of these raw files. That file should
contain either the full path to each file or relative to the current
directory (or the directory defined by \verb+ORAC_DATA_IN+). The data
can be processed with the commands:
\begin{terminalv}
% oracdr_scuba2_XXX -cwd
% oracdr -loop file -files <list_of_files>
\end{terminalv}
where \verb+XXX+ is the wavelength (450 or 850). An optional UT
observation date in the form \verb+YYYYMMDD+ may be given
(e.g.\ 20100301). If the date is omitted, the current date is assumed
- however, the file naming convention uses the date on which the data
were taken. The initialization command only needs to be run once per
UT date, and may be given the \verb+-honour+ flag to use existing
definitions of the relevant environment variables. Alternatively, the
\verb+-cwd+ flag may be given to force the pipeline to use the current
working directory for all input and output.

Note that there is no need to uncompress the data files prior to
running the pipeline: \oracdr\ can accept files compressed with
\verb+gzip+ (ending \verb+.sdf.gz+) and will uncompress them
itself. However, be aware that the uncompressed files are not deleted
at the end of processing (\oracdr\ does not delete raw data).

Each observation is processed separately and the images combined to
form a single output image. If the list of files contains data from
multiple sources, the pipeline will treat each source separately and
create different output files accordingly. Calibration is handled
automatically (see \ref{sse:cal} below).

The default science recipes will display the individual observation
images plus the final coadded image using \GAIA. (The display can be
turned off, if desired, by adding \texttt{-nodisplay} to the
\oracdr\ command line.)

\subsection{Pipeline products}

The science data products from the pipeline have a suffix of
\verb+_reduced+. The files beginning with \verb+s+ are the products
from individual observations; the files beginning \verb+gs+ are the
coadded observations for a single object. The products from
non-science observations may have different suffices, and may be
three-dimensional cubes. See the documentation on the individual
recipes in Appendix\,\ref{ap:full} for further details on those
products.

In addition to the data files, the reduced products have PNG format
images 64, 256 and 1024 pixels on a side for easy viewing in an image
viewer or web browser.

\subsection{Calibration\label{sse:cal}}

If no calibration observations are available, and unless otherwise
instructed, the pipeline will apply standard flux conversion factors
(FCFs) to calibrate the images in mJy\,beam$^{-1}$. Currently these
are 537000 mJy\,beam$^{-1}$ at 850\,$\mu$m and 491000 at
450\,$\mu$m. (See also \cite{scuba2calpaper}.)

\subsection{Customizing the map-making}

The pipeline uses the \SMURF\ dynamic iterative map-maker (\makemap)
to create maps from raw data. A detailed description of the map-maker
is given in \SMURFcook\ and \cite{smurfpaper}. The map-maker uses a
configuration file to control the data processing which may be
modified by users with advanced knowledge of the map maker. The
SCUBA-2 pipeline may be given the name of an alternative or customized
configuration file via the recipe parameter capability of \oracdr. A
number of pre-defined configuration files exist in the directory
\verb+$STARLINK_DIR/share/smurf+.

Once a suitable configuration file has been created, add its name to a
recipe parameter file as follows:
\begin{terminalv}
[REDUCE_SCAN]
MAKEMAP_CONFIG = myconfigfilename.lis
\end{terminalv}
and add \texttt{-recpars recipe\_params.lis} to the command line when
running the pipeline, where \verb+recipe_params.lis+ is the name of
recipe parameter file (which must be in the current directory if the
path is not given). The \task{makemap} configuration file must exist
in the current working directory or one of the directories defined by
the environment variables \verb+MAKEMAP_CONFIG_DIR+,
\verb+ORAC_DATA_OUT+, or \verb+ORAC_DATA_CAL+ or in
\verb+$STARLINK_DIR/share/smurf+. Each directory is searched in this
order and the first match is used.

Note that if running a recipe other than REDUCE\_SCAN (such as one of
the dedicated JLS recipes) that recipe name should be placed in square
brackets instead.

\subsection{Running the science pipeline at EAO/JCMT}

The raw data are stored at EAO in the same way as at the summit. (It
is also possible to do this yourself at your home institution but in
general will not be worth the effort: use the example above instead.)
The machine \verb+sc2dr5+ is available at the summit for general user
data processing.

If processing data from a single night, then \oracdr\ can be run with
the \texttt{-loop flag} option to indicate that the pipeline should
examine the contents of flag files (which end in \verb+.ok+). The flag
files contain the path to the files to be processed, and have a fixed
naming convention so the pipeline can recognize them. Use the
\texttt{-list} option to specify the observation numbers to be
processed (otherwise the pipeline will begin at observation 1). The
command \verb+scuba2_index+ will produce a summary of the available
data.

If processing data from multiple nights, create a text file with the
names of the relevant data files, as for running at your home
institution above, and follow the same procedure.

\subsection{Processing examples\label{sse:examples}}

To process a set of data downloaded from the JCMT archive at CADC,
where the files to be processed have been listed in a text file called
\verb+mydata.lis+:
\begin{terminalv}
% oracdr -loop file -files mydata.lis
\end{terminalv}

To process all files starting at observation 21 (skipping non-existent
files) until there are no more files:
\begin{terminalv}
% oracdr -loop flag -from 21 -skip
\end{terminalv}

To process the files from a list of observations (e.g. 21, 22, 28, 29 and
30):
\begin{terminalv}
% oracdr -loop flag -list 21,22,28:30
\end{terminalv}
Note the use of a colon to specify a contiguous range of observation
numbers.

Two additional options are useful when running on a remote machine or
when an X display is not available:
\begin{itemize}
\item Include the flag \verb+-log sf+ to print the pipeline output to
  a terminal rather than opening a separate window;
\item Include the flag \verb+-nodisplay+ to disable the display of
  pipeline images.
\end{itemize}

