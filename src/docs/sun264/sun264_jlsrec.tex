\section{\xlabel{jlsrecipes}Processing JCMT Legacy Survey data\label{se:jlsrec}}

Currently, three of the JCMT Legacy Surveys have recipes optimized for
the goals of the surveys. Support for the others will be added in as
timely manner as possible in response to survey
input. \picard\ recipes also exist which replicate the steps performed
on the processed data.

\subsection{Cosmology Legacy Survey (CLS)}

The CLS recipe employs a ``jack-knife'' approach using independent
halves of the data in order to estimate and remove residual noise on
large spatial scales.

\begin{itemize}
\item Maps are made with a modified blank-field config file and
  coadded into a single map.
\item An artificial gaussian source is inserted into the data and the
  maps are remade (and coadded to make a PSF map).
\item The signal-only maps are divided into two groups that are
  coadded separately and subtracted to form a jack-knife map.
\item The central portion of the jack-knife map is used to estimate
  the spatial power spectrum which is applied to the signal coadd to
  remove residual low-spatial frequency noise (``whitening'').
\item A matched filter is applied to highlight compact sources using a
  whitened version of the PSF map as the input PSF. A signal-to-noise
  ratio image is also calculated.
\end{itemize}

Currently, this recipe works best on single scanned fields (i.e.\ not
a mosaic of multiple fields).

\subsection{Survey Of Nearby Stars (SONS)}

This recipe is very similar to that for CLS with an additional step at
the beginning. This step makes maps for each 30-second subscan and
calculates the noise level in those maps to determine the noisiest
subscans which are ignored by the map-making step. (Note the option
exists to use the time-series noise instead.)

For this recipe, it is recommended that the artificial source used to
determine the FCF correction be offset from the centre to avoid
contamination of the signal.

\subsection{SCUBA-2 ``Ambitious-Sky'' Survey (SASSy)}

Processing of SASSy data is focused on detecting compact sources with
the aim of following up previously-unknown or interesting-looking
detections with more sensitive observations to probe the detailed
structure.

\begin{itemize}
\item Maps are made with a blank-field config file and coadded.
\item A matched filter is applied to highlight compact sources (using
  the default PSF).
\item A peak-detection task (\task{findclumps}) is run to identify
  5-$\sigma$ peaks, which are written to a catalogue file.
\end{itemize}

\subsection{JCMT Plane Survey (JPS)}

Data reduction is tailored to enable the recovery of bright emission on 
scales up to 480 arcseconds. Data are reduced using a modified bright 
extended config file and coadded.
