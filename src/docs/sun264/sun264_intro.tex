\section{\xlabel{introduction}Introduction\label{se:intro}}

The SCUBA-2 pipeline is a suite of recipes and primitives for the
\oracdr\ automated data processing software package. Additional
functionality for processing and analysis of reduced data is provided
through a number of \picard\ recipes. General documentation on
\oracdr\ and \picard\ can be found in \oracsun\ and
\picardsun\ respectively.

The fundamental operation of the pipeline is to begin with raw data
and produce calibrated science images with no additional user
input. All decisions are based on metadata stored in the data files
combined with basic quality assessment of reduced data products.

\subsection{Document conventions}

In an attempt to make this document clearer to read, different fonts
are used for specific structures.

Observing modes are denoted by all upper case body text (e.g.\
FLATFIELD).

Starlink package names are shown in small caps (e.g.\ \SMURF);
individual task names are shown in sans-serif
(e.g.\ \makemap). \oracdr\ recipes and primitives are also shown in
sans-serif and are always upper case (e.g.\ \task{REDUCE\_SCAN}).

Content listings are shown in fixed-width type (sometimes called
`typewriter'). Extensions and components within NDF (\ndfref) data
files are shown in upper case fixed-width type (e.g.\
\ndfcomp{HISTORY}).

Text relating to filenames (including suffices for data products), key
presses or entries typed at the command line are also denoted by
fixed-width type (e.g.\ \texttt{\% smurf}), as are parameters for
tasks which are displayed in upper case (e.g.\ \aparam{METHOD}).

References to Starlink documents, i.e., Starlink User Notes (SUN),
Starlink General documents (SG) and Starlink Cookbooks (SC), are given
in the text using the document type and the corresponding number
(e.g.\ SUN/95). Non-Starlink documents are cited in the text and
listed in the bibliography.

File name suffices represent the text between the final underscore
character and the three-letter \verb+.sdf+ extension. For example, a
file named \verb+s4a20101020_00002_0001_cal.sdf+ has the suffix
\verb+_cal+.
