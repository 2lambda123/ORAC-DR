\section{\xlabel{pipelines}SCUBA-2 Pipeline Variants\label{se:pipelines}}

There are three variants of the SCUBA-2 pipeline, users will likly only need
to run the science pipeline. The other two pipelines are designed to run in
real time at the JCMT. 

\begin{itemize}
\item The science pipeline has access to all the data observed for a
  given project and adopts a best-possible reduction approach. Images
  are made for each complete observation which are combined to create
  the final image.

\item The quick-look (QL) pipeline is primarily designed to perform
  quality-assurance analysis of the incoming data for real-time
  assessment of the instrument performance. It is also responsible for
  processing pointing and focus data.

\item The summit pipeline runs in parallel with the QL pipeline
  (though on a different machine) and is the primary image-processing
  pipeline. Processing is delayed until sufficient data exist to
  produce a higher quality image. In practice this happens after a
  certain time has elapsed since an image was last made.
\end{itemize}

\subsection{Requirements for running the SCUBA-2 pipeline}

The SCUBA-2 pipeline requires a recent Starlink installation. The
latest version may be obtained from
\htmladdnormallink{\texttt{http://starlink.eao.hawaii.edu/starlink}}{http://starlink.eao.hawaii.edu/starlink}. Since
development of the pipeline is an ongoing process, it is recommended
that the newest builds be used to access the full capabilities of the
pipeline. These builds can be obtained from\\
\htmladdnormallink{\texttt{http://starlink.eao.hawaii.edu/starlink/rsyncStarlink}}{http://starlink.eao.hawaii.edu/starlink/rsyncStarlink}
and may be kept up-to-date with rsync.

The Starlink Perl installation (Starperl) must be used to run the
pipeline due to the module requirements. The Starlink environment should be
initialized as usual before running \oracdr.

The software used to process raw data into images is called the
SubMillimetre User Reduction Facility (\SMURF). Detailed documentation
on \SMURF\ can be found in \SMURFsun, while \SMURFcook\ is a cookbook
that describes some of the background to SCUBA-2 data reduction.

The pipeline uses the following Starlink applications:
\begin{itemize}
\item \SMURF
\item \KAPPA
\item \FLUXES
\item \FIGARO
\item \CCDPACK
\item \CUPID
\end{itemize}

\subsection{Important environment variables}

The pipeline uses a number of environment variables to determine where
data should be read from and written to. Some are set automatically
when the pipeline is initialized, but they can be overridden manually
and, with the \verb+-honour+ flag may be left unchanged between
successive runs of the pipeline. The variables that must be defined in
order for the pipeline to run are denoted as `Mandatory' in the list
below.

\begin{itemize}

\item \verb+STARLINK_DIR+: location of the user's Starlink
  installation. [Mandatory]

\item \verb+ORAC_DATA_IN+: the location where data are read from. If
  running with \verb+-loop flag+, this is the location of the flag
  files, rather than the data files. [Mandatory]

\item \verb+ORAC_DATA_OUT+: location where pipeline data products are
  written. Also used as a location for user-specified configuration
  files for \task{makemap}. [Mandatory]

\item \verb+ORAC_DATA_ROOT+: root location for data. At the JCMT,
  this is \verb+/jcmtdata/raw/scuba2+. If not defined, the current
  directory is assumed. [Optional]

\item \verb+MAKEMAP_CONFIG_DIR+: a user-specified location for
  \task{makemap} configuration files. [Optional]

\item \verb+FINDCLUMPS_CONFIG_DIR+: a user-specified location for
  \task{findclumps} configuration files. [Optional]

\end{itemize}


As an example, to set up or override the pipeline environment variables, tcsh users will need to do:

\begin{terminalv}
% setenv ORAC_DATA_IN folder1/
% setenv ORAC_DATA_OUT folder2/
\end{terminalv}

and bash users will need to do:

\begin{terminalv}
$ export ORAC_DATA_IN=folder1/
$ export ORAC_DATA_OUT=folder2/
\end{terminalv}
 

\subsection{Getting help}

Basic help and a list of command-line options may be obtained after
initializing \oracdr\ by running :
\begin{terminalv}
% oracdr -h
\end{terminalv}
or
\begin{terminalv}
% oracdr -man
\end{terminalv}

More complete documentation on \oracdr\ can be found in \oracsun.

