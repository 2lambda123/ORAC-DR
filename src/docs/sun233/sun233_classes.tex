\subsection{ORAC::Calib\label{ORAC::Calib}\index{ORAC::Calib}}

Base class for selecting calibration frames in ORACDR

\subsubsection*{SYNOPSIS\label{ORAC::Calib_SYNOPSIS}\index{ORAC::Calib!SYNOPSIS}}\begin{verbatim}
  use ORAC::Calib;
\end{verbatim}
\begin{verbatim}
  $Cal = new ORAC::Calib;
\end{verbatim}
\begin{verbatim}
  $dark = $Cal->dark;
  $Cal->dark("darkname");
\end{verbatim}
\begin{verbatim}
  $Cal->standard(undef);
  $standard = $Cal->standard;
  $bias = $Cal->bias;
\end{verbatim}
\subsubsection*{DESCRIPTION\label{ORAC::Calib_DESCRIPTION}\index{ORAC::Calib!DESCRIPTION}}

This module provides the basic methods available to all ORAC::Calib
objects. This class should be used for selecting calibration frames.



Unless specified otherwise, a calibration frame is selected by first,
the nearest reduced frame; second, explicit specification via the
-calib command line option (handled by the pipeline); third, by search
of the appropriate index file.



Note this version: Index files not implemented

\subsubsection*{PUBLIC METHODS\label{ORAC::Calib_PUBLIC_METHODS}\index{ORAC::Calib!PUBLIC METHODS}}

The following methods are available in this class.

\begin{description}
\item[\textbf{new}] \mbox{}

Create a new instance of a ORAC::Calib object.
The object identifier is returned.

\begin{verbatim}
  $Cal = new ORAC::Calib;
\end{verbatim}
\item[\textbf{darkname}] \mbox{}

Return (or set) the name of the current dark - no checking

\begin{verbatim}
  $dark = $Cal->darkname;
\end{verbatim}
\item[\textbf{biasname}] \mbox{}

Return (or set) the name of the current bias - no checking

\begin{verbatim}
  $dark = $Cal->biasname;
\end{verbatim}
\item[\textbf{skyname}] \mbox{}

Return (or set) the name of the current sky frame - no checking

\begin{verbatim}
  $dark = $Cal->skyname;
\end{verbatim}
\item[\textbf{standardname}] \mbox{}

Return (or set) the name of the current standard frame - no checking

\begin{verbatim}
  $dark = $Cal->standardname;
\end{verbatim}
\item[\textbf{dark}] \mbox{}

Return (or set) the name of the current dark - 
checks suitability on return.

\item[\textbf{darknoupdate}] \mbox{}

Stops dark object from updating itself with more recent data



Used when using a command-line override to the pipeline

\item[\textbf{flatnoupdate}] \mbox{}

Stops flat object from updating itself with more recent data



Used when using a command-line override to the pipeline

\item[\textbf{biasnoupdate}] \mbox{}

Stops bias object from updating itself with more recent data



Used when using a command-line override to the pipeline

\item[\textbf{skynoupdate}] \mbox{}

Stops sky object from updating itself with more recent data



Used when using a command-line override to the pipeline

\item[\textbf{standardnoupdate}] \mbox{}

Stops standard object from updating itself with more recent data



Used when using a command-line override to the pipeline

\item[\textbf{bias}] \mbox{}

Return (or set) the name of the current bias.

\begin{verbatim}
  $bias = $Cal->bias;
\end{verbatim}
\item[\textbf{mask}] \mbox{}

Return (or set) the name of the bad pixel mask

\begin{verbatim}
  $mask = $Cal->mask;
\end{verbatim}
\item[\textbf{rotation}] \mbox{}

Return (or set) the name of the rotation transformation matrix

\begin{verbatim}
  $rotation = $Cal->rotation;
\end{verbatim}
\item[\textbf{flatname}] \mbox{}

Return (or set) the name of the current flat - no checking

\begin{verbatim}
  $flat = $Cal->flatname;
\end{verbatim}
\item[\textbf{flat}] \mbox{}

Return (or set) the name of the current flat.

\begin{verbatim}
  $flat = $Cal->flat;
\end{verbatim}
\item[\textbf{arc}] \mbox{}

Return (or set) the name of the current arc.

\begin{verbatim}
  $arc = $Cal->arc;
\end{verbatim}
\item[\textbf{sky}] \mbox{}

Return (or set) the name of the current "sky" frame

\item[\textbf{standard}] \mbox{}

Return (or set) the name of the current standard.

\begin{verbatim}
  $standard = $Cal->standard;
\end{verbatim}
\item[\textbf{darkindex}] \mbox{}

Return (or set) the index object associated with the dark index file

\item[\textbf{flatindex}] \mbox{}

Return (or set) the index object associated with the flat index file

\item[\textbf{biasindex}] \mbox{}

Return (or set) the index object associated with the bias index file

\item[\textbf{skyindex}] \mbox{}

Return (or set) the index object associated with the sky index file

\item[\textbf{standardindex}] \mbox{}

Return (or set) the index object associated with the standard index file

\item[\textbf{thing}] \mbox{}

Returns or sets the hash associated with the header of the object
(frame or group or whatever) needed to match calibration criteria
against.



Ending sentences with a preposition is a bug.

\end{description}
\subsubsection*{SEE ALSO\label{ORAC::Calib_SEE_ALSO}\index{ORAC::Calib!SEE ALSO}}

the \emph{ORAC::Group} manpage and
the \emph{ORAC::Frame} manpage

\subsubsection*{REVISION\label{ORAC::Calib_REVISION}\index{ORAC::Calib!REVISION}}

\$Id$

\subsubsection*{COPYRIGHT\label{ORAC::Calib_COPYRIGHT}\index{ORAC::Calib!COPYRIGHT}}

Copyright (C) 1998-2000 Particle Physics and Astronomy Research
Council. All Rights Reserved.

\subsection{ORAC::Calib::SCUBA\label{ORAC::Calib::SCUBA}\index{ORAC::Calib::SCUBA}}

SCUBA calibration object

\subsubsection*{SYNOPSIS\label{ORAC::Calib::SCUBA_SYNOPSIS}\index{ORAC::Calib::SCUBA!SYNOPSIS}}\begin{verbatim}
  use ORAC::Calib::SCUBA;
\end{verbatim}
\begin{verbatim}
  $Cal = new ORAC::Calib::SCUBA;
\end{verbatim}
\begin{verbatim}
  $gain = $Cal->gain($filter);
  $tau  = $Cal->tau($filter);
  @badbols = $Cal->badbols;
\end{verbatim}
\subsubsection*{DESCRIPTION\label{ORAC::Calib::SCUBA_DESCRIPTION}\index{ORAC::Calib::SCUBA!DESCRIPTION}}

This module returns (and can be used to set) calibration information
for SCUBA. SCUBA calibrations are used for extinction correction
(the sky opacity) and conversion of volts to Janskys.



It can also be used to set and retrieve lists of bad bolometers generated
by noise observations.



This class does inherit from \textbf{ORAC::Calib} although nearly all the
methods in the base class are irrelevant to SCUBA (this class only
uses the thing() method).

\subsubsection*{PUBLIC METHODS\label{ORAC::Calib::SCUBA_PUBLIC_METHODS}\index{ORAC::Calib::SCUBA!PUBLIC METHODS}}

The following methods are available in this class.
These are in addition to the methods inherited from \textbf{ORAC::Calib}.

\paragraph*{Constructor\label{ORAC::Calib::SCUBA_Constructor}\index{ORAC::Calib::SCUBA!Constructor}}\begin{description}
\item[\textbf{new}] \mbox{}

Create a new instance of a ORAC::Calib::SCUBA object.
The object identifier is returned.

\begin{verbatim}
  $Cal = new ORAC::Calib::SCUBA;
\end{verbatim}
\end{description}
\paragraph*{Accessor Methods\label{ORAC::Calib::SCUBA_Accessor_Methods}\index{ORAC::Calib::SCUBA!Accessor Methods}}\begin{description}
\item[\textbf{badbols}] \mbox{}

Set or retrieve the name of the system to be used for bad bolometer
determination. Allowed values are:

\begin{itemize}
\item index

Use an index file generated by noise observations
using the reflector blade. The bolometers stored in this
file are those that were above the noise threshold in 
the \_REDUCE\_NOISE\_ primitive. The index file is generated
by the \_REDUCE\_NOISE\_ primitive

\item file

Uses the contents of the file \emph{badbol.lis} (contains a space
separated list of bolometer names in the first line). This
file is in ORAC\_DATA\_OUT. If the file is not found, no
bolometers will be flagged.

\item 'list'

A colon-separated list of bolometer names can be supplied.
If badbols=h7:i12:g4,... then this list will be used
as the bad bolometers throughout the reduction.

\end{itemize}


Default is to use the 'file' method.
The value is always upper-cased.

\item[\textbf{badbolsindex}] \mbox{}

Return (or set) the index object associated with the bad bolometers
index file. This index file is used if badbols() is set to index.

\item[\textbf{badbolsnoupdate}] \mbox{}

Flag to prevent the badbols system from being modified during data
processing.

\item[\textbf{fluxes\_mon}] \mbox{}

Retrieves the ORAC::Msg::ADAM::Task object associated with
the Starlink fluxes monolith.



A new object is created if the value is undefined.



Relies on the Adam messaging system being available.
ADAM messaging is initialised if not present.



Currently this routine also assumes that no other fluxes
objects are started by this process (since there are a number
of things that must be configured before starting the monolith).

\item[\textbf{fluxes\_tmp\_dir}] \mbox{}

Name of temporary directory created for the fluxes monolith.
(set or retrieve)

\item[\textbf{gains}] \mbox{}

Determines whether gains are derived from the default values
(DEFAULT) or from the index files (INDEX). Default is to
use the default gains. The value is upper-cased.

\item[\textbf{gainsindex}] \mbox{}

Return (or set) the index object associated with the gains
index file. This index file is used if gains() is set to INDEX.

\item[\textbf{gainsnoupdate}] \mbox{}

Flag to prevent the gains selection from being modified during data
processing.

\item[\textbf{skydipindex}] \mbox{}

Return (or set) the index object associated with the skydip
index file. This index file is used if tausys() is set to skydip.

\item[\textbf{tausys}] \mbox{}

Set (or retrieve) the name of the system to be used for
tau determination. Allowed values are 'CSO', 'SKYDIP',
'850SKYDIP' or a number. Currently the number is assumed to be the 
CSO tau since this number is independent of wavelength.
'INDEX' is an allowed synonym for 'SKYDIP'. '850SKYDIP'
mode uses the results of 850 micron skydips from index
files to derive the opacity for the requested wavelength.



Additionally, modes 'DIPINTERP' and '850DIPINTERP' can be 
used to interpolate the current tau from skydips taken
either side of the current observation.



Currently there is no way to specify an actual 850 micron
tau value (the number is treated as a CSO value). In the future
this may change (or a tausys of 850=value will be used??)



If tausys has not been set it defaults to '850SKYDIP'

\item[\textbf{tausysnoupdate}] \mbox{}

Flag to prevent the tau system from being modified during data
processing.

\item[\textbf{taucache}] \mbox{}

Internal cache providing access to previously calculated tau values.
This is a reference to a hash of hashes with keys of uppercased
\texttt{tausys()}, ORACTIME and filter name.

\begin{verbatim}
 $cacheref = $Cal->taucache;
\end{verbatim}
\begin{verbatim}
 $tau = $Cal->taucache->{TAUSYS}->{'19980515.453'}->{$filter};
\end{verbatim}


Returns a hash reference.

\item[\textbf{csofit}] \mbox{}

Object containing all the tau fitting information.
The object is configured the first time the information
is requested. The fitting data are located in
\texttt{ORAC\_DATA\_CAL/csofit.dat}

\end{description}
\paragraph*{General methods\label{ORAC::Calib::SCUBA_General_methods}\index{ORAC::Calib::SCUBA!General methods}}\begin{description}
\item[\textbf{badbol\_list}] \mbox{}

Returns list of bolometer names that should be turned off for the
current observation. The source of this list depends on the setting
of the badbols() parameter (controlled by the user).
Can be one of 'index', 'file' or actual bolometer list. See the
badbols() method documentation for more information.

\item[\textbf{fluxcal}] \mbox{}

Return the flux of a calibrator source

\begin{verbatim}
  $flux = $Cal->fluxcal("sourcename", "filter", $ismap);
\end{verbatim}


The optional third argument is used to specify whether a map
flux (ie total integrated flux) is required (true), or 
simply a flux in beam (used for photometry). Default is to
return flux in beam. This should return the same answer if the
calibrator is a point source.



Currently, all secondary calibrators are assumed to be point like.



Returns undef if the flux could not be determined.

\item[\textbf{gain}] \mbox{}

Method to return the current gain (aka 'flux conversion factor') 
for the specified filter that is usable for the current frame.



\texttt{undef} is returned if no gain can be determined.

\begin{verbatim}
  $gain = $Cal->gain($filter, $units);
\end{verbatim}


The units must be either BEAM (for Jy/beam/V) or ARCSEC (for
Jy/arcsec**2/V). If no units are supplied the default is BEAM.



If gains() is set to DEFAULT then this method will simply return
the current canonical gain for this filter (first trying a specific
filter [eg \texttt{450w}] then trying a generic filter name [eg \texttt{450}]).
This value will not take into account observing mode (eg scan map
gain is lower than jiggle map gain).



If gains() is set to INDEX the index will be searched for a calibration
observation that matches the observation mode (ie Chop throw, sample
mode, observing mode agree).



The current index system refuses to continue if a calibration can
not be found. In future this may well be changed so that the
DEFAULT values are used if no calibration is available.



It may also be useful if the gains either side of current observation
are retrieved so that the gain can be interpolated (as for tau
calculation).

\item[\textbf{iscalsource}] \mbox{}

Given the source name and filter, work out whether we have calibration
information on this source (ie we know the flux for this filter). If
information is availble return true (1) else return (0).

\begin{verbatim}
  $yesno = $Cal->iscalsource("source_name","filter");
\end{verbatim}


If filter is not supplied, it is assumed we are simply asking
whether the source is a calibrator independent of whether we
actually have a calibration value for it....

\item[\textbf{tau}] \mbox{}

Returns the tau associated with the supplied filter.

\begin{verbatim}
  $tau = $Cal->tau($filter);
\end{verbatim}


This routine works as follows. First tausys() is queried to determine
the system to use to calculate the tau. If this is CSO, the current
frame is queried for the CSO tau value stored and the tau calculated
for FILTER. If tausys() returns a number it is assumed
to be the actual CSO tau to use. If it is set to Skydip (or index) then
the selected wavelength is updated in the frame header (Key=FILTER)
and the skydip index is queried for the skydip that matched the criterion
and is closest in time.



The tausys='850SKYDIP' mode uses the results of 850 micron skydips
from index files to derive the opacity for the requested wavelength.



Additionally, modes 'DIPINTERP' and '850DIPINTERP' can be used to
interpolate the current tau from skydips taken either side of the
current observation.



The skydip modes will default to using CSO if a suitable
skydip can not be found. Also, a warning is raised if a skydip
is found but was takan more than 3 hours before or after the
current observation.



undef is returned if an error occurred [eg the CSO is so high that the
tau can not be calculated using the linear relationship].



The value is cached for a given tausys and observation (ORACTIME is
used for uniqueness) to prevent delays in searching for a tau when the
observation has not changed. It is very unlikely that a tau calibration
will change during a data reduction of a single frame (and, in reality
it is required that if you use a particular tau for extinction correction
that you can retrieve the exact same tau that was used at a later date).
The tau value is not cached if it can not be determined.

\end{description}
\paragraph*{Destructor\label{ORAC::Calib::SCUBA_Destructor}\index{ORAC::Calib::SCUBA!Destructor}}\begin{description}
\item[\textbf{DESTROY}] \mbox{}

Removes any directories that may have been created by this
calibration class (eg by starting fluxes).



Assumes that only this object is interested in the fluxes monolith
associated with this object since we are about to remove the
temporary directory containing the JPL ephemeris file.

\end{description}
\subsubsection*{SEE ALSO\label{ORAC::Calib::SCUBA_SEE_ALSO}\index{ORAC::Calib::SCUBA!SEE ALSO}}

the \emph{ORAC::Calib} manpage

\subsubsection*{REVISION\label{ORAC::Calib::SCUBA_REVISION}\index{ORAC::Calib::SCUBA!REVISION}}

\$Id$

\subsubsection*{COPYRIGHT\label{ORAC::Calib::SCUBA_COPYRIGHT}\index{ORAC::Calib::SCUBA!COPYRIGHT}}

Copyright (C) 1998-2000 Particle Physics and Astronomy Research
Council. All Rights Reserved.

\subsection{ORAC::Constants\label{ORAC::Constants}\index{ORAC::Constants}}

Constants available to the ORAC system

\subsubsection*{SYNOPSIS\label{ORAC::Constants_SYNOPSIS}\index{ORAC::Constants!SYNOPSIS}}\begin{verbatim}
  use ORAC::Constants;
  use ORAC::Constants qw/ORAC__OK/;
  use ORAC::Constants qw/:status/;
\end{verbatim}
\subsubsection*{DESCRIPTION\label{ORAC::Constants_DESCRIPTION}\index{ORAC::Constants!DESCRIPTION}}

Provide access to ORAC constants.

\subsubsection*{CONSTANTS\label{ORAC::Constants_CONSTANTS}\index{ORAC::Constants!CONSTANTS}}

The following constants are available from this module:

\begin{description}
\item[\textbf{ORAC\_\_OK}] \mbox{}

This constant contains the definition of good ORAC status.

\item[\textbf{ORAC\_\_ERROR}] \mbox{}

This constant containst the definition of bad ORAC status.

\end{description}
\subsubsection*{TAGS\label{ORAC::Constants_TAGS}\index{ORAC::Constants!TAGS}}

Individual sets of constants can be imported by 
including the module with tags. For example:

\begin{verbatim}
  use ORAC::Constants qw/:status/;
\end{verbatim}


will import all constants associated with ORAC status checking.



The available tags are:

\begin{description}
\item[:status] \mbox{}

Constants associated with ORAC status checking: ORAC\_\_OK and ORAC\_\_ERROR.

\end{description}
\subsubsection*{USAGE\label{ORAC::Constants_USAGE}\index{ORAC::Constants!USAGE}}

The constants can be used as if they are subroutines.
For example, if I want to print the value of ORAC\_\_ERROR I can

\begin{verbatim}
  use ORAC::Constants;
  print ORAC_ERROR;
\end{verbatim}


or

\begin{verbatim}
  use ORAC::Constants ();
  print ORAC::Constants::ORAC__ERROR;
\end{verbatim}
\subsubsection*{SEE ALSO\label{ORAC::Constants_SEE_ALSO}\index{ORAC::Constants!SEE ALSO}}

the \emph{constants} manpage

\subsubsection*{REVISION\label{ORAC::Constants_REVISION}\index{ORAC::Constants!REVISION}}

\$Id$

\subsubsection*{AUTHOR\label{ORAC::Constants_AUTHOR}\index{ORAC::Constants!AUTHOR}}

Tim Jenness (t.jenness@jach.hawaii.edu) and
Frossie Economou (frossie@jach.hawaii.edu)

\subsubsection*{REQUIREMENTS\label{ORAC::Constants_REQUIREMENTS}\index{ORAC::Constants!REQUIREMENTS}}

The \texttt{constants} package must be available. This is a standard
perl package.

\subsubsection*{COPYRIGHT\label{ORAC::Constants_COPYRIGHT}\index{ORAC::Constants!COPYRIGHT}}

Copyright (C) 1998-2000 Particle Physics and Astronomy Research
Council. All Rights Reserved.

\subsection{ORAC::Display\label{ORAC::Display}\index{ORAC::Display}}

Top level interface to ORAC display tools

\subsubsection*{SYNOPSIS\label{ORAC::Display_SYNOPSIS}\index{ORAC::Display!SYNOPSIS}}\begin{verbatim}
  use ORAC::Display;
\end{verbatim}
\begin{verbatim}
  $Display = new ORAC::Display;
  $Display->usenbs(1);
  $Display->filename(filename);
  $Display->display_data('frame/group object');
  $Display->display_data('frame/group object',{WINDOW=>1});
\end{verbatim}
\subsubsection*{DESCRIPTION\label{ORAC::Display_DESCRIPTION}\index{ORAC::Display!DESCRIPTION}}

This module provides an OO-interface to the ORAC display manager.  The
display object reads device information from a file or notice board
(shared memory) [NBS not implemented], determines whether the supplied
frame object matches the criterion for display, if it does it
instructs the relevant device object to send to the selected window
(creating a new device object if necessary)

\subsubsection*{PUBLIC METHODS\label{ORAC::Display_PUBLIC_METHODS}\index{ORAC::Display!PUBLIC METHODS}}\paragraph*{Constructor\label{ORAC::Display_Constructor}\index{ORAC::Display!Constructor}}\begin{description}
\item[\textbf{new}] \mbox{}

Create a new instance of \textbf{ORAC::Display}. No arguments are
required.

\begin{verbatim}
  $Display = new ORAC::Display;
\end{verbatim}
\end{description}
\subsubsection*{Accessor Methods\label{ORAC::Display_Accessor_Methods}\index{ORAC::Display!Accessor Methods}}\begin{description}
\item[\textbf{display\_tools}] \mbox{}

Returns (or sets) a hash containing the current lookup of display tool
to display tool object. For example:

\begin{verbatim}
   $Display->display_tools(%tools);
   %tools = $Display->display_tools;
\end{verbatim}


where \%tools could look like:

\begin{verbatim}
     'GAIA' => Display::GAIA=HASH(object),
     'P4'   => Display::P4=HASH(object)
\end{verbatim}


etc. The current contents are overwritten when a new hash is supplied.



When called from an array context, returns the full hash contents.
When called from a scalar context, returns the reference to the hash.

\item[\textbf{filename}] \mbox{}

Set (or retrieve) the name of the file containing the display device
definition. Only used when usenbs() is false.

\begin{verbatim}
  $file = $Display->file;
  $Display->file("new_file");
\end{verbatim}
\item[\textbf{idstring}] \mbox{}

Set (or retrieve) the value of the string used for comparison
with the display device definition information (created by the
separate device allocation GUI).

\begin{verbatim}
  $Display->idstring($id);
  $id = $Display->idstring;
\end{verbatim}
\item[\textbf{usenbs}] \mbox{}

Determine whether NBS (shared memory) should be used to read the
display device definition. Default is false.

\begin{verbatim}
  $usenbs = $Display->usenbs;
  $Display->usenbs(0);
\end{verbatim}
\end{description}
\paragraph*{General Methods\label{ORAC::Display_General_Methods}\index{ORAC::Display!General Methods}}\begin{description}
\item[\textbf{definition}] \mbox{}

Method to read a display definition, compare it with the idstring 
stored in the object (this is usually a file suffix)
and return back an array of  hashes containing all the relevant entries
from the definition. If an argument is given, the object updates
its definition of current idstring (and then searches).

\begin{verbatim}
   @defn = $display->definition;
   @defn = $display->definition($id);
\end{verbatim}


An empty array is returned if the suffix can not be matched.

\item[\textbf{display\_data}] \mbox{}

This is the main method to be used for displaying data.  The supplied
object must contain a method for determining the filename and the
display ID (so that it can be compared with the information stored in
the device definition file). It should support the file(), nfiles()
and gui\_id() methods.



The optional hash can be used to supply extra entries in the
display definition file (or in fact do away with the definition file
completely). Note that the contents of the options hash will be used
even if no display definition can be found to match the current 
gui\_id.

\begin{verbatim}
  $Display->display_data($Frm) if defined $Display;
  $Display->display_data($Frm, { TOOL => 'GAIA'});
  $Display->display_data($Frm, { TOOL => 'GAIA'}, $usedisp);
\end{verbatim}


A third optional argument can be used in conjunction with the
options hash to indicate whether these options should be used
instead of the display definition file (false) or in addition
to (true - the default)

\item[\textbf{parse\_nbs\_defn}] \mbox{}

Using the current idstring, read the relevant information from
a noticeboard and return it in a hash. This routine takes no
arguments (idstring is read from the object) and should only
be used if the usenbs() flag is true.

\begin{verbatim}
  %defn = $self->parse_nbs_defn;
\end{verbatim}


Currently not implemented.

\item[\textbf{parse\_file\_defn}] \mbox{}

Using the current idstring, read the relevant information from
the text file (name stored in filename()) and return it in an array
of  hashes. There will be one hash per entry in the file that
matches the given suffix.
This routine takes no arguments (idstring is read from the object).



The input file is assumed to contain one line per ID of the following
format:

\begin{verbatim}
  ID  key=value key=value key=value..........\n
\end{verbatim}
\end{description}
\subsubsection*{SEE ALSO\label{ORAC::Display_SEE_ALSO}\index{ORAC::Display!SEE ALSO}}

Related ORAC display devices (eg the \emph{ORAC::Display::KAPVIEW} manpage)

\subsubsection*{REVISION\label{ORAC::Display_REVISION}\index{ORAC::Display!REVISION}}

\$Id$

\subsubsection*{COPYRIGHT\label{ORAC::Display_COPYRIGHT}\index{ORAC::Display!COPYRIGHT}}

Copyright (C) 1998-2000 Particle Physics and Astronomy Research
Council. All Rights Reserved.

\subsection{ORAC::Frame\label{ORAC::Frame}\index{ORAC::Frame}}

Base class for dealing with observation frames in ORAC-DR

\subsubsection*{SYNOPSIS\label{ORAC::Frame_SYNOPSIS}\index{ORAC::Frame!SYNOPSIS}}\begin{verbatim}
  use ORAC::Frame;
\end{verbatim}
\begin{verbatim}
  $Frm = new ORAC::Frame("filename");
  $Frm->file("prefix_flat");
  $num = $Frm->number;
\end{verbatim}
\subsubsection*{DESCRIPTION\label{ORAC::Frame_DESCRIPTION}\index{ORAC::Frame!DESCRIPTION}}

This module provides the basic methods available to all \textbf{ORAC::Frame}
objects. This class should be used when dealing with individual
observation files (frames).

\subsubsection*{PUBLIC METHODS\label{ORAC::Frame_PUBLIC_METHODS}\index{ORAC::Frame!PUBLIC METHODS}}

The following methods are available in this class:

\paragraph*{Constructors\label{ORAC::Frame_Constructors}\index{ORAC::Frame!Constructors}}

The following constructors are available:

\begin{description}
\item[\textbf{new}] \mbox{}

Create a new instance of a \textbf{ORAC::Frame} object.  This method also
takes optional arguments: if 1 argument is supplied it is assumed to
be the name of the raw file associated with the observation. If 2
arguments are supplied they are assumed to be the raw file prefix and
observation number. In any case, all arguments are passed to the
configure() method which is run in addition to new() when arguments
are supplied.  The object identifier is returned.

\begin{verbatim}
   $Frm = new ORAC::Frame;
   $Frm = new ORAC::Frame("file_name");
   $Frm = new ORAC::Frame("UT", "number");
\end{verbatim}


The base class constructor should be invoked by sub-class constructors.
If this method is called with the last argument as a reference to
a hash it is assumed that this hash contains extra configuration
information ('instance' information) supplied by sub-classes.

\end{description}
\paragraph*{Accessor Methods\label{ORAC::Frame_Accessor_Methods}\index{ORAC::Frame!Accessor Methods}}

The following methods are available for accessing the 
'instance' data.

\begin{description}
\item[\textbf{file}] \mbox{}

This method can be used to retrieve or set the file names that are 
currently associated with the frame. Multiple file names can be stored
if required (for example the names associated with different 
SCUBA sub-instruments).

\begin{verbatim}
  $first_file = $Frm->file;     # First file name
  $first_file = $Frm->file(1);  # First file name
  $second_file= $Frm->file(2);  # Second file name
  $Frm->file(1, value);         # Set the first file name
  $Frm->file(value);            # Set the first filename
  $Frm->file(10, value);        # Set the tenth file name
\end{verbatim}


Note that counting starts at 1 (and not 0 as is normal for Perl 
arrays) and that the filename can not be an integer (otherwise
it will be treated as an array index). Use files() to retrieve
all the values in an array context.



If a file has been marked as temporary (ie with the nokeep()
method) it is erased (running the erase() method) when the file
name is updated.



For example, the second file (file\_2) is marked as temporary
with \texttt{\$Frm-$>$nokeep(2,1)}. The next time the filename is updated
(\texttt{\$Frm-$>$file(2,'new\_file')}) the current file is erased before the
'new\_file' name is stored. The temporary flag is then reset to
zero.



If a file number is requested that does not exist, the first
member is returned.



Every time the file name is updated, the new file is pushed onto
the intermediates() array. This is so that intermediate files
can be tidied up when required.

\item[\textbf{files}] \mbox{}

Set or retrieve the array containing the current file names
associated with the frame object.

\begin{verbatim}
    $Frm->files(@files);
    @files = $Frm->files;
\end{verbatim}
\begin{verbatim}
    $array_ref = $Frm->files;
\end{verbatim}


In a scalar context the array reference is returned.
In an array context, the array contents are returned.



The file() method can be used to set or retrieve individual
filenames.



Note: It is possible to set and retrieve the array members using
the array reference rather than the file() method:

\begin{verbatim}
  $first = $Frm->files->[0];
\end{verbatim}


In this approach, the file numbering starts at 0. The file() method
is the recommended way of addressing individual members of this
array since the file() method could do extra processing of the
string (especially when setting the value, for example the automatic
deletion of temporary files).

\item[\textbf{format}] \mbox{}

Data format associated with the current file().
Usually one of 'NDF' or 'FITS'. This format should be
recognisable by \texttt{ORAC::Convert}.

\item[\textbf{group}] \mbox{}

This method returns the group name associated with the observation.

\begin{verbatim}
  $group_name = $Frm->group;
  $Frm->group("group");
\end{verbatim}


This can be configured initially using the findgroup() method.
Alternatively, findgroup() is run automatically by the configure()
method.

\item[\textbf{hdr}] \mbox{}

This method allows specific entries in the header to be accessed.  In
general, this header is related to the actual header information
stored in the Frame file. The input argument should correspond to the
keyword in the header hash.

\begin{verbatim}
  $tel = $Frm->hdr("TELESCOP");
  $instrument = $Frm->hdr("INSTRUME");
\end{verbatim}


Can also be used to set values in the header.
A hash can be used to set multiple values (but does not overwrite
other keys).

\begin{verbatim}
  $Frm->hdr("INSTRUME" => "IRCAM");
  $Frm->hdr("INSTRUME" => "SCUBA", 
            "TELESCOP" => 'JCMT');
\end{verbatim}


If no arguments are provided, the reference to the header hash
is returned.

\begin{verbatim}
  $Frm->hdr->{INSTRUME} = 'SCUBA';
\end{verbatim}


The header can be populated from the file by using the readhdr()
method.

\item[\textbf{intermediates}] \mbox{}

An array containing all the intermediate file names used
during processing. Filenames are pushed onto this array
whenever the file() method is used to update the current
file information.

\begin{verbatim}
  $Frm->intermediates(@files);
  @files = $Frm->intermediates;
  push(@{$Frm->intermediates}, $file);
  $first = $Frm->intermediates->[0];
\end{verbatim}


As for the files() method, returns an array reference when
called in a scalar context and an array of file names when
called from an array context.



The array does not store information relating to the position of the
file in the files() array [ie was it stored as \texttt{\$Frm-$>$file(1)} or
\texttt{\$Frm-$>$file(2)}]. The order simply reflects the order the files
were given to the file() method.

\item[\textbf{isgood}] \mbox{}

Flag to determine the current state of the frame. If isgood()
is true the Frame is valid. If it returns false the frame
object may have a problem (eg the recipe responsible for 
processing the frame failed to complete).



This flag is used by the \textbf{ORAC::Group} class to determine
membership.

\item[\textbf{nokeep}] \mbox{}

Flag used to determine whether the current filename should be
erased when the file() method is next used to update the current
filename.

\begin{verbatim}
  $Frm->erase($i) if $Frm->nokeep($i);
\end{verbatim}
\begin{verbatim}
  $Frm->nokeep($i, 1);  # make ith file temporary
  $Frm->nokeep($i, 0);  # Make ith file permanent
\end{verbatim}
\begin{verbatim}
  $nokeep = $Frm->nokeep($i);
\end{verbatim}


The mandatory first argument specifies the file number associated with
this flag (same scheme as used by the file() method). An optional
second argument can be used to set the flag. 'True' indicates that the
file should not be kept, 'false' indicates that the file is permanent.

\item[\textbf{nokeepArr}] \mbox{}

Array of flags. Used internally by nokeep() method.  Set or retrieve
the array containing the flags used by the nokeep() method to
determine whether the current filename should be erased when the
file() method is next used to update the current filename.

\begin{verbatim}
    $Frm->nokeepArr(@flags);
    @flags = $Frm->nokeepArr;
\end{verbatim}
\begin{verbatim}
    $array_ref = $Frm->nokeepArr;
\end{verbatim}


In a scalar context the array reference is returned.
In an array context, the array contents are returned.



The nokeep() method can be used to set or retrieve individual
flags (the numbering scheme is different).



Note: It is possible to set and retrieve the array members using
the array reference rather than the nokeep() method:

\begin{verbatim}
  $first = $Frm->nokeepArr->[0];
\end{verbatim}


In this approach, the numbering starts at 0. The nokeep() method
is the recommended way of addressing individual members of this
array since it could do extra processing of the
string.

\item[\textbf{nsubs}] \mbox{}

Return the number of sub-frames associated with this frame.



nfiles() should be used to return the current number of sub-frames
associated with the frame (nsubs usually only reports the number given
in the header and may or may not be the same as the number of sub-frames
currently stored)



Usually this value is set as part of the configure() method from
the header (using findnsubs()) or by using findnsubs() directly.

\item[\textbf{raw}] \mbox{}

This method returns (or sets) the name of the raw data file
associated with this object.

\begin{verbatim}
  $Frm->raw("raw_data");
  $filename = $Frm->raw;
\end{verbatim}
\item[\textbf{rawfixedpart}] \mbox{}

Return (or set) the constant part of the raw filename associated
with the raw data file. (ie the bit that stays fixed for every 
observation)

\begin{verbatim}
  $fixed = $self->rawfixedpart;
\end{verbatim}
\item[\textbf{rawformat}] \mbox{}

Data format associated with the raw() data file.
Usually one of 'NDF' or 'FITS'. This format should be
recognisable by \texttt{ORAC::Convert}.

\item[\textbf{rawsuffix}] \mbox{}

Return (or set) the file name suffix associated with
the raw data file.

\begin{verbatim}
  $suffix = $self->rawsuffix;
\end{verbatim}
\item[\textbf{recipe}] \mbox{}

This method returns the recipe name associated with the observation.
The recipe name can be set explicitly but in general should be
set by the findrecipe() method.

\begin{verbatim}
  $recipe_name = $Frm->recipe;
  $Frm->recipe("recipe");
\end{verbatim}


This can be configured initially using the findrecipe() method.
Alternatively, findrecipe() is run automatically by the configure()
method.

\item[\textbf{uhdr}] \mbox{}

This method allows specific entries in the user-defined header to be 
accessed. The input argument should correspond to the keyword in the header
hash.

\begin{verbatim}
  $tel = $Frm->uhdr("Telescope");
  $instrument = $Frm->uhdr("Instrument");
\end{verbatim}


Can also be used to set values in the header.
A hash can be used to set multiple values (but does not overwrite
other keys).

\begin{verbatim}
  $Frm->uhdr("Instrument" => "IRCAM");
  $Frm->uhdr("Instrument" => "SCUBA", 
             "Telescope" => 'JCMT');
\end{verbatim}


If no arguments are provided, the reference to the header hash
is returned.

\begin{verbatim}
  $Frm->uhdr->{Instrument} = 'SCUBA';
\end{verbatim}
\end{description}
\paragraph*{General Methods\label{ORAC::Frame_General_Methods}\index{ORAC::Frame!General Methods}}

The following methods are provided for manipulating
\textbf{ORAC::Frame} objects:

\begin{description}
\item[\textbf{calc\_orac\_headers}] \mbox{}

This method calculates header values that are required by the
pipeline by using values stored in the header.



Required ORAC extensions are:



ORACTIME: should be set to a decimal time that can be used for
comparing the relative start times of frames. For IRCAM this
number is decimal hours, for SCUBA this number is decimal
UT days.



ORACUT: This is the UT day of the frame in YYYYMMDD format.



This method should be run after a header is set. Currently the readhdr()
method calls this whenever it is updated.



This method updates the frame header.
Returns a hash containing the new keywords.

\item[\textbf{configure}] \mbox{}

This method is used to configure the object. It is invoked
automatically if the new() method is invoked with an argument. The
file(), raw(), readhdr(), findgroup(), findrecipe and findnsubs()
methods are invoked by this command. Arguments are required.  If there
is one argument it is assumed that this is the raw filename. If there
are two arguments the filename is constructed assuming that argument 1
is the prefix and argument 2 is the observation number.

\begin{verbatim}
  $Frm->configure("fname");
  $Frm->configure("UT","num");
\end{verbatim}
\item[\textbf{erase}] \mbox{}

Erase the current file from disk.

\begin{verbatim}
  $Frm->erase($i);
\end{verbatim}


The optional argument specified the file number to be erased.
The argument is identical to that given to the file() method.
Returns ORAC\_\_OK if successful, ORAC\_\_ERROR otherwise.



Note that the file() method is not modified to reflect the
fact the the file associated with it has been removed from disk.



This method is usually called automatically when the file()
method is used to update the current filename and the nokeep()
flag is set to true. In this way, temporary files can be removed
without explicit use of the erase() method. (Just need to
use the nokeep() method after the file() method has been used
to update the current filename).

\item[\textbf{file\_exists}] \mbox{}

Method to determine whether the Frame file() exists on disk or not.
Returns true if the file is there, false otherwise. Effectively
equivalent to using -e but allows for the possibility that the
information stored in file() does not directly relate to the
file as stored on disk (e.g. a .sdf extension). The base class is
very simplistic (ie does not assume extensions).

\begin{verbatim}
  $exists = $Frm->file_exists($i)
\end{verbatim}


The optional argument refers to the file number.

\item[\textbf{file\_from\_bits}] \mbox{}

Determine the raw data filename given the variable component
parts. A prefix (usually UT) and observation number should
be supplied.

\begin{verbatim}
  $fname = $Frm->file_from_bits($prefix, $obsnum);
\end{verbatim}
\item[\textbf{findgroup}] \mbox{}

Method to determine the group to which the observation belongs.
The default method is to look for a "GRPNUM" entry in the header.

\begin{verbatim}
  $group = $Frm->findgroup;
\end{verbatim}


The object is automatically updated via the group() method.

\item[\textbf{findnsubs}] \mbox{}

Find the number of sub-frames associated with the frame by looking in
the header. Usually run by configure().



In the base class this method looks for a header keyword of 'NSUBS'.

\begin{verbatim}
  $nsubs = $Frm->findnsubs;
\end{verbatim}


The state of the object is updated automatically.

\item[\textbf{findrecipe}] \mbox{}

Method to determine the recipe name that should be used to reduce
the observation.
The default method is to look for a "RECIPE" entry in the header.

\begin{verbatim}
  $recipe = $Frm->findrecipe;
\end{verbatim}


The object is automatically updated to reflect this recipe.

\item[\textbf{flag\_from\_bits}] \mbox{}

Determine the name of the flag file given the variable
component parts. A prefix (usually UT) and observation number
should be supplied

\begin{verbatim}
  $flag = $Frm->flag_from_bits($prefix, $obsnum);
\end{verbatim}


This method should be implemented by a sub-class.

\item[\textbf{gui\_id}] \mbox{}

Returns the identification string that is used to compare the
current frame with the frames selected for display in the
display definition file.



Arguments:

\begin{verbatim}
 number - the file number (as accepted by the file() method)
          Starts counting at 0. If no argument is supplied
          a 1 is assumed.
\end{verbatim}


To return the ID associated with the second frame:

\begin{verbatim}
 $id = $Frm->gui_id(2);
\end{verbatim}


If nfiles() equals 1, this method returns everything after the last
suffix (using an underscore) from the filename stored in file(1). If
nfiles $>$ 1, this method returns the everything after the last 
underscore, prepended with 's\$number'. ie if file(2) is test\_dk,
the ID would be 's2dk'; if file() is test\_dk (and nfiles = 1) the
ID would be 'dk'.

\item[\textbf{inout}] \mbox{}

Method to return the current input filename and the new output
filename given a suffix.  For the base class the input filename is
chopped at the last underscore and the suffix appended when the name
contains at least 2 underscores. The suffix is simply appended if
there is only one underscore. This prevents numbers being chopped when
the name is of the form ut\_num.



Note that this method does not set the new output name in this
object. This must still be done by the user.



Returns \$in and \$out in an array context:

\begin{verbatim}
   ($in, $out) = $Frm->inout($suffix);
\end{verbatim}


Returns \$out in a scalar context:

\begin{verbatim}
   $out = $Frm->inout($suffix);
\end{verbatim}


Therefore if in=file\_db and suffix=\_ff then out would
become file\_db\_ff but if in=file\_db\_ff and suffix=dk then
out would be file\_db\_dk.



An optional second argument can be used to specify the
file number to be used. Default is for this method to process
the contents of file(1).

\begin{verbatim}
  ($in, $out) = $Frm->inout($suffix, 2);
\end{verbatim}


will return the second file name and the name of the new output
file derived from this.

\item[\textbf{nfiles}] \mbox{}

Number of files associated with the current state of the object and
stored in file(). This method lets the caller know whether an
observation has generated multiple output files for a single input.

\item[\textbf{number}] \mbox{}

Method to return the number of the observation. The number is
determined by looking for a number at the end of the raw data
filename.  For example a number can be extracted from strings of the
form textNNNN.sdf or textNNNN, where NNNN is a number (leading zeroes
are stripped) but not textNNNNtext (number must be followed by a decimal
point or nothing at all).

\begin{verbatim}
  $number = $Frm->number;
\end{verbatim}


The return value is -1 if no number can be determined.



As an aside, an alternative approach for this method (especially
in a sub-class) would be to read the number from the header.

\item[\textbf{readhdr}] \mbox{}

A method that is used to read header information from the current
file and store that information in the object. For the base class,
this method does nothing since the base class does not know 
the format of the file associated with the object. There are
no return arguments.

\begin{verbatim}
  $Frm->readhdr;
\end{verbatim}


The calc\_orac\_headers() method is called automatically.

\item[\textbf{tagset}] \mbox{}

Associate the current filenames with a key (or tag). Once a tag
is initialised (it can be any string) the \texttt{tagretrieve} method
can be used to copy these filenames back into the object so that
the \texttt{files()} method will use those rather than the current
values. This allows the data reduction steps to be "rewound".

\begin{verbatim}
  $Frm->tagset('REBIN');
\end{verbatim}


The tag is case insensitive.

\item[\textbf{tagretrieve}] \mbox{}

Retrieve the files names from the tag and make them the default
filenames for the object.

\begin{verbatim}
  $Frm->tagretrieve('REBIN');
\end{verbatim}


Nothing happens if the tag does not previously exist.
The current filenames are stored in the 'PREVIOUS' tag (unless the
PREVIOUS tag is requested).

\item[\textbf{template}] \mbox{}

Method to change the current filename of the frame (file())
so that it matches a template. e.g.:

\begin{verbatim}
  $Frm->template("something_number_flat");
\end{verbatim}


Would change the first file to match "something\_number\_flat".
Essentially this simply means that the number in the template
is changed to the number of the current frame object.

\begin{verbatim}
  $Frm->template("something_number_dark", 2);
\end{verbatim}


would change the second filename to match "something\_number\_dark".
The base method assumes that the filename matches the form:
prefix\_number\_suffix. This must be modified by the derived
classes since in general the filenaming convention is telescope
and instrument specific.



The Nth filename is modified (ie file(N)).
There are no return arguments.

\end{description}
\subsubsection*{PRIVATE METHODS\label{ORAC::Frame_PRIVATE_METHODS}\index{ORAC::Frame!PRIVATE METHODS}}

The following methods are intended for use inside the module.
They are included here so that authors of derived classes are 
aware of them.

\begin{description}
\item[stripfname] \mbox{}

Method to strip file extensions from the filename string. This method
is called by the file() method. For the base class this method
does nothing. It is intended for derived classes (e.g. so that ".sdf"
can be removed).

\end{description}
\subsubsection*{SEE ALSO\label{ORAC::Frame_SEE_ALSO}\index{ORAC::Frame!SEE ALSO}}

the \emph{ORAC::Group} manpage

\subsubsection*{REVISION\label{ORAC::Frame_REVISION}\index{ORAC::Frame!REVISION}}

\$Id$

\subsubsection*{COPYRIGHT\label{ORAC::Frame_COPYRIGHT}\index{ORAC::Frame!COPYRIGHT}}

Copyright (C) 1998-2000 Particle Physics and Astronomy Research
Council. All Rights Reserved.

\subsection{ORAC::Frame::NDF\label{ORAC::Frame::NDF}\index{ORAC::Frame::NDF}}

Class for dealing with frames based on NDF files

\subsubsection*{SYNOPSIS\label{ORAC::Frame::NDF_SYNOPSIS}\index{ORAC::Frame::NDF!SYNOPSIS}}\begin{verbatim}
  use ORAC::Frame::NDF
\end{verbatim}
\begin{verbatim}
  $Frm = new ORAC::Frame::NDF;
\end{verbatim}
\subsubsection*{DESCRIPTION\label{ORAC::Frame::NDF_DESCRIPTION}\index{ORAC::Frame::NDF!DESCRIPTION}}

This class provides implementations of the methods that require
knowledge of the NDF file format rather than generic methods or
methods that require knowledge of a specific instrument.  In general,
the specific instrument sub-classes will inherit from the file type
(which inherits from ORAC::Frame) rather than directly from
ORAC::Frame. For JCMT and UKIRT the group files are based on NDFs and
inherit from this class.



The format specific sub-classes do not contain constructors; they 
should be defined in either the base class or the instrument specific
sub-class.

\subsubsection*{PUBLIC METHODS\label{ORAC::Frame::NDF_PUBLIC_METHODS}\index{ORAC::Frame::NDF!PUBLIC METHODS}}

The following methods are modified from the base class versions.

\paragraph*{General Methods\label{ORAC::Frame::NDF_General_Methods}\index{ORAC::Frame::NDF!General Methods}}\begin{description}
\item[\textbf{erase}] \mbox{}

Erase the current file from disk.

\begin{verbatim}
  $Frm->erase($i);
\end{verbatim}


The optional argument specifies the file number to be erased.
The argument is identical to that given to the file() method.
Returns ORAC\_\_OK if successful, ORAC\_\_ERROR otherwise.



Note that the file() method is not modified to reflect the
fact the the file associated with it has been removed from disk.



This method is usually called automatically when the file()
method is used to update the current filename and the nokeep()
flag is set to true. In this way, temporary files can be removed
without explicit use of the erase() method. (Just need to
use the nokeep() method after the file() method has been used
to update the current filename).

\item[\textbf{file\_exists}] \mbox{}

Checks for the existence of the frame file(). Assumes a \texttt{.sdf}
extension.

\begin{verbatim}
  $exists = $Frm->exists($i)
\end{verbatim}


The optional argument specifies the file number to be used.
All extension are removed from the file name before adding the
\texttt{.sdf} so that HDS containers can be supported (and files
that already have the extension)  -- but note that
this version of the method does not look inside HDS containers
looking for NDFs.

\item[\textbf{readhdr}] \mbox{}

Reads the header from the observation file (the filename is stored in
the object).  This method sets the header in the object (in general
that is done by configure() ).

\begin{verbatim}
    $Frm->readhdr;
\end{verbatim}


The filename can be supplied if the one stored in the object
is not required:

\begin{verbatim}
    $Frm->readhdr($file);
\end{verbatim}


but the header in \$Frm is over-written.
All exisiting header information is lost. The calc\_orac\_headers()
method is invoked once the header information is read.
If there is an error during the read a reference to an empty hash is 
returned.



Currently this method assumes that the reduced group is stored in
NDF format. Only the FITS header is retrieved from the NDF.



There are no return arguments.

\end{description}
\subsubsection*{PRIVATE METHODS\label{ORAC::Frame::NDF_PRIVATE_METHODS}\index{ORAC::Frame::NDF!PRIVATE METHODS}}

The following methods are intended for use inside the module.
They are included here so that authors of derived classes are 
aware of them.

\begin{description}
\item[\textbf{stripfname}] \mbox{}

Method to strip file extensions from the filename string. This method
is called by the file() method. For UKIRT we strip all extensions of the
form ".sdf", ".sdf.gz" and ".sdf.Z" since Starlink tasks do not require
the extension when accessing the file name.

\end{description}
\subsubsection*{REQUIREMENTS\label{ORAC::Frame::NDF_REQUIREMENTS}\index{ORAC::Frame::NDF!REQUIREMENTS}}

Currently this module requires the the \emph{NDF} manpage module.

\subsubsection*{SEE ALSO\label{ORAC::Frame::NDF_SEE_ALSO}\index{ORAC::Frame::NDF!SEE ALSO}}

the \emph{ORAC::Group} manpage

\subsubsection*{REVISION\label{ORAC::Frame::NDF_REVISION}\index{ORAC::Frame::NDF!REVISION}}

\$Id$

\subsubsection*{COPYRIGHT\label{ORAC::Frame::NDF_COPYRIGHT}\index{ORAC::Frame::NDF!COPYRIGHT}}

Copyright (C) 1998-2000 Particle Physics and Astronomy Research
Council. All Rights Reserved.

\subsection{ORAC::Frame::UKIRT\label{ORAC::Frame::UKIRT}\index{ORAC::Frame::UKIRT}}

UKIRT class for dealing with observation files in ORAC-DR

\subsubsection*{SYNOPSIS\label{ORAC::Frame::UKIRT_SYNOPSIS}\index{ORAC::Frame::UKIRT!SYNOPSIS}}\begin{verbatim}
  use ORAC::Frame::UKIRT;
\end{verbatim}
\begin{verbatim}
  $Frm = new ORAC::Frame::UKIRT("filename");
  $Frm->file("file")
  $Frm->readhdr;
  $Frm->configure;
  $value = $Frm->hdr("KEYWORD");
\end{verbatim}
\subsubsection*{DESCRIPTION\label{ORAC::Frame::UKIRT_DESCRIPTION}\index{ORAC::Frame::UKIRT!DESCRIPTION}}

This module provides methods for handling Frame objects that
are specific to UKIRT. It provides a class derived from \textbf{ORAC::Frame::NDF}.
All the methods available to \textbf{ORAC::Frame} objects are available
to \textbf{ORAC::Frame::UKIRT} objects.

\subsubsection*{PUBLIC METHODS\label{ORAC::Frame::UKIRT_PUBLIC_METHODS}\index{ORAC::Frame::UKIRT!PUBLIC METHODS}}

The following methods are available in this class in addition to
those available from \textbf{ORAC::Frame}.

\paragraph*{General Methods\label{ORAC::Frame::UKIRT_General_Methods}\index{ORAC::Frame::UKIRT!General Methods}}\begin{description}
\item[\textbf{findgroup}] \mbox{}

Returns group name from header.  For dark observations the current obs
number is returned if the group number is not defined or is set to zero
(the usual case with IRCAM)



The group name stored in the object is automatically updated using 
this value.

\item[\textbf{findrecipe}] \mbox{}

Find the recipe name. If no recipe can be found from the
'DRRECIPE' FITS keyword'QUICK\_LOOK' is returned by default.



The recipe name stored in the object is automatically updated using 
this value.

\end{description}
\subsubsection*{SEE ALSO\label{ORAC::Frame::UKIRT_SEE_ALSO}\index{ORAC::Frame::UKIRT!SEE ALSO}}

the \emph{ORAC::Group} manpage, the \emph{ORAC::Frame::NDF} manpage

\subsubsection*{REVISION\label{ORAC::Frame::UKIRT_REVISION}\index{ORAC::Frame::UKIRT!REVISION}}

\$Id$

\subsubsection*{COPYRIGHT\label{ORAC::Frame::UKIRT_COPYRIGHT}\index{ORAC::Frame::UKIRT!COPYRIGHT}}

Copyright (C) 1998-2000 Particle Physics and Astronomy Research
Council. All Rights Reserved.

\subsection{ORAC::Frame::JCMT\label{ORAC::Frame::JCMT}\index{ORAC::Frame::JCMT}}

JCMT class for dealing with observation files in ORACDR

\subsubsection*{SYNOPSIS\label{ORAC::Frame::JCMT_SYNOPSIS}\index{ORAC::Frame::JCMT!SYNOPSIS}}\begin{verbatim}
  use ORAC::Frame::JCMT;
\end{verbatim}
\begin{verbatim}
  $Frm = new ORAC::Frame::JCMT("filename");
  $Frm->file("file")
  $Frm->readhdr;
  $Frm->configure;
  $value = $Frm->hdr("KEYWORD");
\end{verbatim}
\subsubsection*{DESCRIPTION\label{ORAC::Frame::JCMT_DESCRIPTION}\index{ORAC::Frame::JCMT!DESCRIPTION}}

This module provides methods for handling Frame objects that
are specific to JCMT. It provides a class derived from \textbf{ORAC::Frame}.
All the methods available to \textbf{ORAC::Frame} objects are available
to \textbf{ORAC::Frame::JCMT} objects. Some additional methods are supplied.

\subsubsection*{PUBLIC METHODS\label{ORAC::Frame::JCMT_PUBLIC_METHODS}\index{ORAC::Frame::JCMT!PUBLIC METHODS}}

The following are modifications to standard ORAC::Frame methods.

\paragraph*{Constructors\label{ORAC::Frame::JCMT_Constructors}\index{ORAC::Frame::JCMT!Constructors}}\begin{description}
\item[\textbf{new}] \mbox{}

Create a new instance of a \textbf{ORAC::Frame::JCMT} object.
This method also takes optional arguments:
if 1 argument is  supplied it is assumed to be the name
of the raw file associated with the observation. If 2 arguments
are supplied they are assumed to be the raw file prefix and
observation number. In any case, all arguments are passed to
the configure() method which is run in addition to new()
when arguments are supplied.
The object identifier is returned.

\begin{verbatim}
   $Frm = new ORAC::Frame::JCMT;
   $Frm = new ORAC::Frame::JCMT("file_name");
   $Frm = new ORAC::Frame::JCMT("UT","number");
\end{verbatim}


This method runs the base class constructor and then modifies
the rawsuffix and rawfixedpart to be '.sdf' and '\_dem\_'
respectively.

\end{description}
\paragraph*{Subclassed methods\label{ORAC::Frame::JCMT_Subclassed_methods}\index{ORAC::Frame::JCMT!Subclassed methods}}

The following methods are provided for manipulating
\textbf{ORAC::Frame::JCMT} objects. These methods override those
provided by \textbf{ORAC::Frame}.

\begin{description}
\item[\textbf{calc\_orac\_headers}] \mbox{}

This method calculates header values that are required by the
pipeline by using values stored in the header.



ORACTIME is calculated - this is the time of the observation as
UT day + fraction of day.



ORACUT is simply read from UTDATE converted to YYYYMMDD.



This method updates the frame header.
Returns a hash containing the new keywords.

\item[\textbf{configure}] \mbox{}

This method is used to configure the object. It is invoked
automatically if the new() method is invoked with an argument. The
file(), raw(), readhdr(), findgroup(), findrecipe(), findsubs() 
findfilters() and findwavelengths() methods are
invoked by this command. Arguments are required.
If there is one argument it is assumed that this is the
raw filename. If there are two arguments the filename is
constructed assuming that arg 1 is the prefix and arg2 is the
observation number.

\begin{verbatim}
  $Frm->configure("fname");
  $Frm->configure("UT","num");
\end{verbatim}


The sub-instrument configuration is also stored.

\item[\textbf{file\_from\_bits}] \mbox{}

Determine the raw data filename given the variable component
parts. A prefix (usually UT) and observation number should
be supplied.

\begin{verbatim}
  $fname = $Frm->file_from_bits($prefix, $obsnum);
\end{verbatim}
\item[\textbf{findgroup}] \mbox{}

Return the group associated with the Frame. This group is constructed
from header information. The group name is automatically updated in
the object via the group() method.



The group membership can be set using the DRGROUP keyword in the
header. If this keyword exists and is not equal to 'UNKNOWN' the
contents will be returned.



Alternatively, if DRGROUP is not specified the group name is
constructed from the MODE, OBJECT and FILTER keywords. This may cause
problems in the following cases:

\begin{verbatim}
 - The chop throw changes and the data should not be coadded
 [in general this is true except for LO chopping scan maps
 where all 6 chops should be included in the group]
\end{verbatim}
\begin{verbatim}
 - The source name is the same, the mode is the same and the
 filter is the same but the source coordinates are different by
 a degree or more. In some cases [a large scan map] these should
 be in the same group. In other cases they probably should not
 be. Should I worry about it? One example was where the observer
 used RB coordinates by mistake for a first map and then changed
 to RJ -- the coordinates and source name were identical but the
 position on the sky was miles off. Maybe this should be dealt with
 by using the Frame ON/OFF facility [so it would be part of the group
 but the observer would turn the observation off]
\end{verbatim}
\begin{verbatim}
 - Different source names are being used for offsets around
 a common centre [eg the Galactic Centre scan maps]. In this case
 we do want to coadd but this means we should be using position
 rather than source name. Also, how do we define when two fields
 are too far apart to be coadded
\end{verbatim}
\begin{verbatim}
 - Photometry data should never be in the same group as a source
 that has a different pointing centre. Note this really should take
 MAP_X and MAP_Y into account since data should be of the same group
 if either the ra/dec is given or if the mapx/y is given relative
 to a fixed ra/dec.
\end{verbatim}


Bottom line is the following (I think).



In all cases the actual position in RJ coordinates should be calculated
(taking into account RB-$>$RJ and GA-$>$RJ and map\_x map\_y, local\_coords) 
using Astro::SLA. Filter should also be matched as now.
Planets will be special cases - matching on name rather than position.



PHOTOM observations

\begin{verbatim}
  Should match positions exactly (within 1 arcsec). Should also match
  chop throws [since the gain is different]. The observer is responsible
  for a final coadd. Source name then becomes irrelevant.
\end{verbatim}


JIGGLE MAP

\begin{verbatim}
  Should match positions to within 10 arcmin (say). Should match chop
  throw.
\end{verbatim}


SCAN MAP

\begin{verbatim}
  Should match positions to 1 or 2 degrees?
  Should ignore chop throws (the primitive deals with that).
\end{verbatim}


The group name will then use the position with a number of significant
figures changing depending on the position tolerance.

\item[\textbf{findnsubs}] \mbox{}

Forces the object to determine the number of sub-instruments
associated with the data by looking in the header (hdr()). 
The result is stored in the object using nsubs().



Unlike findgroup() this method will always search the header for
the current state.

\item[\textbf{findrecipe}] \mbox{}

Return the recipe associated with the frame.
The state of the object is automatically updated via the
recipe() method.



The recipe is determined by looking in the FITS header
of the frame. If the 'DRRECIPE' is present and not
set to 'UNKNOWN' then that is assumed to specify the recipe
directly. Otherwise, header information is used to try
to guess at the reduction recipe. The default recipes
are keyed by observing mode:

\begin{verbatim}
 SKYDIP => 'SCUBA_SKYDIP'
 NOISE  => 'SCUBA_NOISE'
 POINTING => 'SCUBA_POINTING'
 PHOTOM => 'SCUBA_STD_PHOTOM'
 JIGMAP => 'SCUBA_JIGMAP'
 JIGMAP (phot) => 'SCUBA_JIGPHOTMAP'
 EM2_SCAN => 'SCUBA_EM2SCAN'
 EKH_SCAN => 'SCUBA_EKHSCAN'
 POLMAP => 'SCUBA_JIGPOLMAP'
 ALIGN  => 'SCUBA_ALIGN'
 FOCUS  => 'SCUBA_FOCUS'
\end{verbatim}


In future we may want to have a separate text file containing
the mapping between observing mode and recipe so that
we dont have to hard wire the relationship.

\item[\textbf{inout}] \mbox{}

Method to return the current input filename and the 
new output filename given a suffix and a sub-instrument
number.



Returns \$in and \$out in an array context:

\begin{verbatim}
  ($in, $out) = $Frm->inout($suffix, $num);
\end{verbatim}


Returns \$out in a scalar context:

\begin{verbatim}
  $out = $Frm->inout($suffix, $num);
\end{verbatim}


The second argument indicates the sub-instrument number
and is optional (defaults to first sub-instrument).
If only one file is present then that is used as \$infile.
(handled by the file() method.)



Currently, the output filename is constructed by removing
everything after the last underscore and appending the suffix.
If no underscore is found the suffix is appended without chopping.
If the string after the last underscore is simply a number,
it is not removed.



Some examples (suffix = '\_trn'):



If input equals "o65" the output filename will be "o65\_trn".
If input equals "o65\_flat" the output filename will be "o65\_trn".
If input equals "19980123\_dem\_0065" the output filename will be
"19980123\_dem\_0065\_trn".

\item[\textbf{template}] \mbox{}

This method is identical to the base class template method
except that only files matching the specified sub-instrument
are affected.

\begin{verbatim}
  $Frm->template($template, $sub);
\end{verbatim}


If no sub-instrument is specified then the first file name
is modified



Note that this is different to the base class which accepts
a file number as the second argument. This may need some
rationalisation.

\end{description}
\subsubsection*{NEW METHODS FOR JCMT\label{ORAC::Frame::JCMT_NEW_METHODS_FOR_JCMT}\index{ORAC::Frame::JCMT!NEW METHODS FOR JCMT}}

This section describes methods that are available in addition
to the standard methods found in \textbf{ORAC::Frame}.

\paragraph*{Accessor Methods\label{ORAC::Frame::JCMT_Accessor_Methods}\index{ORAC::Frame::JCMT!Accessor Methods}}

The following extra accessor methods are provided:

\begin{description}
\item[\textbf{filters}] \mbox{}

Return or set the filter names associated with each sub-instrument
in the frame.

\item[\textbf{subs}] \mbox{}

Return or set the names of the sub-instruments associated
with the frame.

\item[\textbf{wavelengths}] \mbox{}

Return or set the wavelengths associated with each  sub-instrument
in the frame.

\end{description}
\paragraph*{New methods\label{ORAC::Frame::JCMT_New_methods}\index{ORAC::Frame::JCMT!New methods}}

The following additional methods are provided:

\begin{description}
\item[\textbf{file2sub}] \mbox{}

Given a file index, (see file()) returns the associated
sub-instrument.

\begin{verbatim}
  $sub = $Frm->file2sub(2)
\end{verbatim}


Returns the first sub name if index is too large.
This assumes that the file names associated wth the
object are linked to sub-instruments (as returned
by the subs method). It is up to the primitive writer
to make sure that subs() tracks changes to files().

\item[\textbf{findfilters}] \mbox{}

Forces the object to determine the names of all sub-instruments
associated with the data by looking in the hdr().



The result is stored in the object using filters(). The sub-inst filter
name is made to match the filter name such that a filter of '450w:850w'
has filter names of '450W' and '850W' despite the entries in the header
being simply '450' and '850'. Photometry filter names are not modified.



Unlike findgroup() this method will always search the header for
the current state.

\item[\textbf{findsubs}] \mbox{}

Forces the object to determine the names of all sub-instruments
associated with the data by looking in the header (hdr()). 
The result is stored in the object using subs().



Unlike findgroup() this method will always search the header for
the current state.

\item[\textbf{findwavelengths}] \mbox{}

Forces the object to determine the names of all sub-instruments
associated with the data by looking in the header (hdr()). 
The result is stored in the object using wavelengths().



Unlike findgroup() this method will always search the header for
the current state.

\item[\textbf{sub2file}] \mbox{}

Given a sub instrument name returns the associated file
index. This is the reverse of sub2file. The resulting value
can be used directly in file() to retrieve the file name.

\begin{verbatim}
  $file = $Frm->file($Frm->sub2file('LONG'));
\end{verbatim}


A case insensitive comparison is performed.



Returns 1 if nothing matched (ie just returns the first file
in file(). This is probably a bug.



Assumes that changes in subs() are reflected in files().

\end{description}
\subsubsection*{SEE ALSO\label{ORAC::Frame::JCMT_SEE_ALSO}\index{ORAC::Frame::JCMT!SEE ALSO}}

the \emph{ORAC::Frame} manpage, the \emph{ORAC::Frame::NDF} manpage

\subsubsection*{REVISION\label{ORAC::Frame::JCMT_REVISION}\index{ORAC::Frame::JCMT!REVISION}}

\$Id$

\subsubsection*{COPYRIGHT\label{ORAC::Frame::JCMT_COPYRIGHT}\index{ORAC::Frame::JCMT!COPYRIGHT}}

Copyright (C) 1998-2000 Particle Physics and Astronomy Research
Council. All Rights Reserved.

\subsection{ORAC::General\label{ORAC::General}\index{ORAC::General}}

Simple perl subroutines that may be useful for primitives

\subsubsection*{SYNOPSIS\label{ORAC::General_SYNOPSIS}\index{ORAC::General!SYNOPSIS}}\begin{verbatim}
  use ORAC::General;
\end{verbatim}
\begin{verbatim}
  $max = max(@values);
  $min = min(@values);
  $result = log10($value);
  $result = nint($value);
  $yyyymmdd = utdate;
  %hash = parse_keyvalue($string);
  @obs = parse_obslist($string);
\end{verbatim}
\subsubsection*{DESCRIPTION\label{ORAC::General_DESCRIPTION}\index{ORAC::General!DESCRIPTION}}

This module provides simple perl functions that are not available
from standard perl. These are available to all ORAC primitive writers,
but they are general in nature and have no connection to orac. Some of
these are used in the ORAC infastructure, so ORACDR does require this
library in order to run.

\subsubsection*{SUBROUTINES\label{ORAC::General_SUBROUTINES}\index{ORAC::General!SUBROUTINES}}\begin{description}
\item[min(ARRAY)] \mbox{}

Find the minimum value of an array. Can also be used to find
the minimum of a list of scalars since arguments are passed into
the subroutine in an array context.

\begin{verbatim}
  $min = min(@values);
  $min = min($a, $b, $c);
\end{verbatim}
\item[max(ARRAY)] \mbox{}

Find the maximum value of an array. Can also be used to find
the maximum of a list of scalars since arguments are passed into
the subroutine in an array context.

\begin{verbatim}
  $max = max(@values);
  $max = max($a, $b, $c);
\end{verbatim}
\item[log10(scalar)] \mbox{}

Returns the logarithm to base ten of a scalar.

\begin{verbatim}
  $value = log10($number);
\end{verbatim}


Currently uses the implementation of log10 found in the
POSIX module

\item[nint()] \mbox{}

Return the nearest integer to a supplied floating point
value. 0.5 is rounded up.

\item[utdate] \mbox{}

Return the UT date (strictly, GMT) date in the format yyyymmdd

\item[parse\_keyvalues] \mbox{}

Takes a string of comma-seperated key-value pairs and return a hash.

\item[parse\_obslist(list)] \mbox{}

Converts a comma separated list of observation numbers (as supplied
on the command line for the -list option) and converts it to
an array of observation numbers. Colons are treated as range arguments.



For example,

\begin{verbatim}
   5,9:11
\end{verbatim}


is converted to

\begin{verbatim}
   (5,9,10,11)
\end{verbatim}
\end{description}
\subsubsection*{SEE ALSO\label{ORAC::General_SEE_ALSO}\index{ORAC::General!SEE ALSO}}

the \emph{POSIX} manpage

\subsubsection*{REVISION\label{ORAC::General_REVISION}\index{ORAC::General!REVISION}}

\$Id$

\subsubsection*{COPYRIGHT\label{ORAC::General_COPYRIGHT}\index{ORAC::General!COPYRIGHT}}

Copyright (C) 1998-2000 Particle Physics and Astronomy Research
Council. All Rights Reserved.

\subsection{ORAC::Group\label{ORAC::Group}\index{ORAC::Group}}

Base class for dealing with observation groups in ORAC-DR

\subsubsection*{SYNOPSIS\label{ORAC::Group_SYNOPSIS}\index{ORAC::Group!SYNOPSIS}}\begin{verbatim}
  use ORAC::Group;
\end{verbatim}
\begin{verbatim}
  $Grp = new ORAC::Group("group1");
\end{verbatim}
\begin{verbatim}
  $Grp->file("Group_file_name");
  $group_name = $Grp->name;
  $Grp->push($frame);
  $total_in_group = $Grp->num;
  $frame3 = $Grp->frame(2);
  @good_members = $Grp->members;
\end{verbatim}
\subsubsection*{DESCRIPTION\label{ORAC::Group_DESCRIPTION}\index{ORAC::Group!DESCRIPTION}}

This module provides the basic methods available to all
\textbf{ORAC::Group} objects. This class should be used when 
storing information relating to a group of observations
processed in the \textbf{ORAC-DR} data reduction pipeline.



Groups are composed of frame objects (\textbf{ORAC::Frame})
or objects that can perform those methods.

\subsubsection*{PUBLIC METHODS\label{ORAC::Group_PUBLIC_METHODS}\index{ORAC::Group!PUBLIC METHODS}}

The following methods are available in this class.

\paragraph*{Constructors\label{ORAC::Group_Constructors}\index{ORAC::Group!Constructors}}

The following constructors are available:

\begin{description}
\item[\textbf{new}] \mbox{}

Create a new instance of a \textbf{ORAC::Group} object.
This method takes an optional argument containing the
name of the new group. The object identifier is returned.

\begin{verbatim}
   $Grp = new ORAC::Group;
   $Grp = new ORAC::Group("group_name");
\end{verbatim}


The base class constructor should be invoked by sub-class constructors.
If this method is called with the last argument as a reference to
a hash it is assumed that this hash contains extra configuration
information ('instance' information) supplied by sub-classes.

\item[\textbf{subgrp}] \mbox{}

Method to return a new group (ie a sub-group of the existing
group) that contains all members of the main group matching
certain header values.



Arguments is a hash that is used for comparison with each
frame.

\begin{verbatim}
  $subgrp = $Grp->subgrp(NAME => 'CRL618', CHOP=> 60.0);
\end{verbatim}


The new subgrp is blessed into the same class as \$Grp.
All header information (hdr() and uhdr()) is copied 
from the main group to the sub-group.



This method is generally used where access to members of the
group by some search criterion is required.



It is possible that the returned group will contain no 
members....

\item[\textbf{subgrps}] \mbox{}

Returns frames grouped by the supplied header keys.
A frame can not belong to more than one sub group created by this
method:

\begin{verbatim}
   @grps = $Grp->subgrps(@keys);
\end{verbatim}


The groups in @grps are blessed into the same class as \$Grp.
For example, if @keys = ('MODE','CHOP') then you can gurantee
that the members of each sub group will have the same values
for MODE and CHOP.



All header information from the main group is copied to the
sub groups.

\end{description}
\paragraph*{Accessor methods\label{ORAC::Group_Accessor_methods}\index{ORAC::Group!Accessor methods}}

The following methods are available for accessing the 
'instance' data.

\begin{description}
\item[\textbf{allmembers}] \mbox{}

Set or retrieve the array containing the current group membership.

\begin{verbatim}
    $Grp->allmembers(@frames);
    @frames = $Grp->allmembers;
\end{verbatim}


The setting function of this routine should only be used
if you know what you are doing (since it completely changes the group
membership). If setting the contents, the check\_membership() method
is run automatically so that the list of valid members can remain
synchronized.



All group members are returned regardless of the state of each member.
Use the members() method to return only valid members.



If called in a scalar context, a reference to an array is returned
rather than the array.

\begin{verbatim}
  $ref = $Grp->allmembers;
  $first = $Grp->allmembers->[0];
\end{verbatim}


Do not use this array reference to change the contents of the array
directly unless the check\_membership() method is run immediately
afterwards. The check\_membership() method is responsible for 
checking the state of each member and copying them to the members()
array.

\item[\textbf{badobs\_index}] \mbox{}

Return (or set) the index object associate with the bad observation
index file. A index of class \textbf{ORAC::Index::Extern} is used since 
this index is modified by an external user/program.



The index is created automatically the first time this method
is invoked.

\item[\textbf{coadds}] \mbox{}

Return (or set) the array containing the list of frame numbers that have
been coadded into the current group. This is not necessarily the same
as the return of the membernumbers() method since that can return numbers
for all the members of the group even if the full coaddition has not
taken place or the pipeline has been resumed partway through a coaddition
(in which case the coadds array will contain more numbers than are in the
group).

\begin{verbatim}
  @coadds = $Grp->coadds;
  $coaddref = $Grp->coadds;
  $Grp->coadds(@numbers);
\end{verbatim}


Returns an array reference in a scalar context, an array in an
array context.



The contents of this array are not automatically written to the 
group file when changed, see the coaddspush() or coaddswrite() methods
for further information on object persistence. The array is simply
meant as a storage area for the pipeline.

\item[\textbf{file}] \mbox{}

Set or retrieve the filename associated with the
reduced group.

\begin{verbatim}
    $Grp->file("group_filename");
    $group_file = $Grp->file;
\end{verbatim}


Currently only one filename can be associated with the group
(although the method will accept, but ignore, a number supplied
as first argument so as to provide compatibility with the
display system).



If raw() is undefined, it is set to this value when the filename is updated.

\item[\textbf{filesuffix}] \mbox{}

Set or retrieve the filename suffix associated with the
reduced group.

\begin{verbatim}
    $Grp->filesuffix(".sdf");
    $group_file = $Grp->filesuffix;
\end{verbatim}
\item[\textbf{fixedpart}] \mbox{}

Set or retrieve the part of the group filename that does not
change between invocation. The output filename can be derived using
this.

\begin{verbatim}
    $Grp->fixedpart("rg");
    $prefix = $Grp->fixedpart;
\end{verbatim}
\item[\textbf{hdr}] \mbox{}

This method allows specific entries in the header to be accessed.  In
general, this header is related to the actual header information
stored in the Group file. The input argument should correspond to the
keyword in the header hash.

\begin{verbatim}
  $tel = $Grp->hdr("TELESCOP");
  $instrument = $Grp->hdr("INSTRUME");
\end{verbatim}


Can also be used to set values in the header.
A hash can be used to set multiple values (but does not overwrite
other keys).

\begin{verbatim}
  $Grp->hdr("INSTRUME" => "IRCAM");
  $Grp->hdr("INSTRUME" => "SCUBA", 
            "TELESCOP" => 'JCMT');
\end{verbatim}


If no arguments are provided, the reference to the header hash
is returned.

\begin{verbatim}
  $Grp->hdr->{INSTRUME} = 'SCUBA';
\end{verbatim}


The header can be populated from the file by using the readhdr()
method.

\item[\textbf{members}] \mbox{}

Retrieve the array containing the valid objects within the group

\begin{verbatim}
    @frames = $Grp->members;
\end{verbatim}


This is the safest way to access the group members
since it only returns valid frames to the caller.



Use the allmembers() method to return all members of the group 
regardless of the state of the individual frames.



Group membership should not be set using ths method since that may lead
to a situation where the actual membership of the group does not match the
selected membership. [Valid group membership should only be set from
within this class].



If called in a scalar context, a reference to an array is returned
rather than the array.

\begin{verbatim}
  $first = $Grp->members->[0];
\end{verbatim}
\item[\textbf{name}] \mbox{}

Set or retrieve the name of the group (ie the 
group identifier)

\begin{verbatim}
    $Grp->name("group_name");
    $group_name = $Grp->name;
\end{verbatim}
\item[\textbf{raw}] \mbox{}

This method returns (or sets) the name of the raw data file
associated with this object. In the context of a group, it is
the name of the group file before any group level processing is
done.

\begin{verbatim}
  $Grp->raw("raw_data");
  $filename = $Grp->raw;
\end{verbatim}
\item[\textbf{uhdr}] \mbox{}

This method allows specific entries in the user-defined header to be 
accessed. The input argument should correspond to the keyword in the header
hash.

\begin{verbatim}
  $tel = $Grp->uhdr("Telescope");
  $instrument = $Grp->uhdr("Instrument");
\end{verbatim}


Can also be used to set values in the header.
A hash can be used to set multiple values (but does not overwrite
other keys).

\begin{verbatim}
  $Grp->uhdr("Instrument" => "IRCAM");
  $Grp->uhdr("Instrument" => "SCUBA", 
             "Telescope" => 'JCMT');
\end{verbatim}


If no arguments are provided, the reference to the header hash
is returned.

\begin{verbatim}
  $Grp->uhdr->{Instrument} = 'SCUBA';
\end{verbatim}
\end{description}
\paragraph*{General methods\label{ORAC::Group_General_methods}\index{ORAC::Group!General methods}}

The following methods are provided for manipulating \textbf{ORAC::Group}
objects:

\begin{description}
\item[\textbf{calc\_orac\_headers}] \mbox{}

This method calculates header values that are required by the
pipeline by using values stored in the header.



Required ORAC extensions are:



ORACTIME: should be set to a decimal time that can be used for
comparing the relative start times of frames. For IRCAM this
number is decimal hours, for SCUBA this number is decimal
UT days.



ORACUT: This is the UT day of the frame in YYYYMMDD format.



This method should be run after a header is set. Currently the header()
method calls this whenever it is updated.



This is an abstract method and should be defined by a sub-class.

\item[\textbf{check\_membership}] \mbox{}

Check whether any of the members of the group have been marked for
removal from the group. The valid group members are copied
to a new array and can be retrieved by the members() method.
Note that all group methods use the list of valid group
members.



This routine is automatically run whenever the group membership
is updated (via the push() or  allmembers() methods. This may
cause too high an overhead with push() in, for example, the
subgrps method).



This method works by looking in a text file created by the
observer in \$ORAC\_DATA\_OUT called index.badobs. This file
contains a list of numbers (two per line) relating to observations
that should be turned off. The first number is the UT date
(YYYYMMDD) and the second number is the observation
number. This is necessary so that ORAC\_DATA\_OUT can be reused
for a different UT date without worrying about the index file
file turning off incorrect observations.



The UT and observation number are compared with each member of
the group (the full list of members - see allmembers()).
For each group member, the following test is performed to test
for validity. First it is queried to check whether it is in a
good state (ie has been processed successfully). 
A frame will be marked as bad if the recipe fails to execute
successfully. If the frame is good (from the pipeline viewpoint)
the UT date and observation number is then compared with the
entries in the index file. If a match can \textbf{NOT} be found the
frame is considered to be valid and is copied to the list of valid
group members (see the members() method).



The format of the index file should be of the form:

\begin{verbatim}
 24 19980716 
 27 19980716 
 43 19980815 
 ...
\end{verbatim}
\item[\textbf{coaddspush}] \mbox{}

Used to push observation numbers onto the coadds() array. Automatically
runs coaddswrite() to update to sync the file contents with the coadds()
array.

\begin{verbatim}
  $Grp->coaddspush(@numbers);
\end{verbatim}
\item[\textbf{coaddspresent}] \mbox{}

Compares the contents of the coadds() array with the supplied (single)
argument. Returns true if the argument is present in the coadds()
array, false otherwise. Also, returns false if no arguments are supplied
or if the argument is undef.

\begin{verbatim}
  $present = $Grp->coaddspresent($number);
\end{verbatim}
\item[\textbf{coaddsread}] \mbox{}

Reads the coadds() information from the current group file and stores
it in the group using the coadds() method.
Should return ORAC\_\_OK if the coadds information was read successfully,
else returns ORAC\_\_ERROR.



This is an abstract method and should be defined by a subclass.

\item[\textbf{coaddswrite}] \mbox{}

Method to write the contents of the coadds() array to the current
group file. Should return ORAC\_\_OK if the coadds information was written
successfully, else returns ORAC\_\_ERROR.



If coadds() contains no entries, all coadds information is removed from
the group file if present.



This is an abstract method and should be defined by a subclass.

\item[\textbf{erase}] \mbox{}

Erases the group file from disk.

\begin{verbatim}
   $Grp->erase;
\end{verbatim}


Returns ORAC\_\_OK if successful, ORAC\_\_ERROR otherwise.

\item[\textbf{file\_exists}] \mbox{}

Method to determine whether the group file() exists on disk or not.
Returns true if the file is there, false otherwise. Effectively
equivalent to using \texttt{-e} but allows for the possibility that the
information stored in file() does not directly relate to the
file as stored on disk (e.g. a .sdf extension).

\item[\textbf{file\_from\_bits}] \mbox{}

Method to return the group filename derived from a fixed
variable part (eg UT) and a group designator (usually obs
number). The full filename is returned (including suffix).

\begin{verbatim}
  $file = $Grp->file_from_bits("UT","num");
\end{verbatim}


For the base class the return string is of the format

\begin{verbatim}
  fixedpart . prefix . '_' . number . suffix
\end{verbatim}


For example, with IRCAM using a UT date of 980104 and observation
number 25 the returned string would be 'rg980104\_25.sdf'.

\item[\textbf{frame}] \mbox{}

Retrieve the nth frame of the group.
Counting starts at 0 as for a standard perl array.

\begin{verbatim}
  $Frm = $Grp->frame(2);
\end{verbatim}


This is equivalent to

\begin{verbatim}
  $Frm = $Grp->members->[2];
\end{verbatim}


A second argument can be used to set the nth frame.

\begin{verbatim}
  $Grp->frame(3, $Frm);
\end{verbatim}


Note that this replaces the nth frame in the list of valid members and
also replaces the equivalent frame in the list of all members of the
group. This is done since the nth valid member is not necessarily the
nth group member. If the supplied position is greater than the current
number of members the supplied frame is simply pushed onto the
array. Remember that just because a frame has been inserted into the
group does not necessarily mean that it will be a valid member
(check\_membership() will be run when setting any member of the group).
If the current frame at the specified position can not be found in
allmembers() the supplied frame is pushed onto allmembers() and
membership is re-checked.

\item[\textbf{inout}] \mbox{}

Method to return the current filenames for each frame in the
group (similar to the membernames() method) and a set of output
names for each file. This is achieved by calling the inout()
method for each frame in turn. This will fail if the members of the
group do not possess the inout() method.



This method can take two arguments: the new suffix and, optionally,
the file number to use (see the inout() documentation for
\textbf{ORAC::Frame}). References to two arrays are returned when called
in an array context; returns the output array ref when called
from a scalar context

\begin{verbatim}
  ($inref, $outref) = $Grp->inout("suffix");
  ($inref, $outref) = $Grp->inout("suffix",2);
  $outref= $Grp->inout("suffix");
\end{verbatim}
\item[\textbf{lastmember}] \mbox{}

Method to determine whether the supplied argument
matches the last member of the group. Returns a 1 if
it is the last member and a zero otherwise.

\begin{verbatim}
   $islast = $Grp->lastmember($Frm);
\end{verbatim}
\item[\textbf{membernames}] \mbox{}

Return a list of all the files associated with the group. This is
achieved by invoking the file() method for each object stored in the
Members array.  For this to work each member must be an object capable
of invoking the file() method (e.g. \textbf{ORAC::Frame}). Currently the
routine does not check to make sure this is possible - the program
will die if you try to use a SCALAR member.



If an argument list is given the file names for each member of the
group are updated. This will only be attempted if the number of 
arguments given matches the number of members in the group.

\begin{verbatim}
  $Grp->membernames(@newnames);
  @names = $Grp->membernames;
\end{verbatim}


Only the first file from each frame object is returned.

\item[\textbf{membernumbers}] \mbox{}

Return a list of all the observation numbers associated with
the group. This is achieved by invoking the number() method for
each object stored in the Members array.
For this to work each member must be an object capable of invoking
numbers() (e.g. \textbf{ORAC::Frame}). Currently the routine does not check
to make sure this is possible - the program will die if you try
to use a SCALAR member.

\begin{verbatim}
  @numbers = $Grp->membernumbers;
\end{verbatim}
\item[\textbf{membertagset}] \mbox{}

Set the tag in each of the members.

\begin{verbatim}
  $Grp->membertagset( 'TAG' );
\end{verbatim}


Runs the \texttt{tagset} method on each of the member frames.

\item[\textbf{membertagretrieve}] \mbox{}

Run the \texttt{tagretrieve()} method for each of the members.

\begin{verbatim}
  $Grp->membertagretrieve
\end{verbatim}
\item[\textbf{num}] \mbox{}

Return the number of frames in a group minus one.
This is identical to the \$\# construct.

\begin{verbatim}
  $number_of_frames = $Grp->num;
\end{verbatim}
\item[\textbf{push}] \mbox{}

Method to push an observation into the group. Multiple observations
can be pushed on at once (see the \emph{perl} manpage "push()" command).

\begin{verbatim}
  $Grp->push("observation2");
  $Grp->push(@obs);
\end{verbatim}


There are no return arguments.

\item[\textbf{readhdr}] \mbox{}

A method that is used to read header information from the group
file. This method does nothing by default since the base
class does not know the format of the file associated with an
object.



The calc\_orac\_headers() method is called automatically.

\item[\textbf{template}] \mbox{}

Method to change all the current filenames in the group so that they
match the supplied template. This method invokes the template()
method for each member of the group.

\begin{verbatim}
  $Grp->template("filename_template");
\end{verbatim}


A second argument can be specified to modify the specified frame
number rather than simply the first (see the template() method
in \textbf{ORAC::Frame} for more details):

\begin{verbatim}
  $Grp->template($template,2);
\end{verbatim}


There are no return arguments. The intelligence for this method resides
in the individual frame objects.

\item[\textbf{updateout}] \mbox{}

This method updates the current filename of each member of the group
when supplied with a suffix (and optionally, a file number -- see the
inout() method in \textbf{ORAC::Frame} for more information). The inout() 
method (of the individual frame) is invoked for each member to 
generate the output name.

\begin{verbatim}
  $Grp->updateout("suffix");
  $Grp->updateout("suffix", 5);
\end{verbatim}


This can be used to update the member filenames after an operation
has been applied to every file in the group. Alternatively the 
membernames() method can be invoked with the output of the inout()
method.

\end{description}
\subsubsection*{DISPLAY COMPATIBILITY\label{ORAC::Group_DISPLAY_COMPATIBILITY}\index{ORAC::Group!DISPLAY COMPATIBILITY}}

These methods are provided for compatibility with the ORAC display
system.

\begin{description}
\item[\textbf{gui\_id}] \mbox{}

Returns the identification string that is used to compare the
current frame with the frames selected for display in the
display definition file.



In the default case, this method returns everything after the
last suffix stored in file().



In some derived implementation of this method an argument
may be used so that multiple IDs can be extracted from objects
that contain more than one output file per observation.

\item[\textbf{nfiles}] \mbox{}

This method is used by the display system to determine the
number of files to display. Since the Group base class can only
ever contain one file name (as returned by file()) this method
always returns a 1.

\end{description}
\subsubsection*{PRIVATE METHODS\label{ORAC::Group_PRIVATE_METHODS}\index{ORAC::Group!PRIVATE METHODS}}

The following methods are intended for use inside the module.
They are included here so that authors of derived classes are 
aware of them.

\begin{description}
\item[\textbf{stripfname}] \mbox{}

Method to strip file extensions from the filename string. This method
is called by the file() method. For the base class this method
does nothing. It is intended for derived classes (e.g. so that ".sdf"
can be removed). Granted that I could simply force the "file" method
to be modified for derived classes....(which is why this method is
private).

\end{description}
\subsubsection*{SEE ALSO\label{ORAC::Group_SEE_ALSO}\index{ORAC::Group!SEE ALSO}}

the \emph{ORAC::Frame} manpage

\subsubsection*{REVISION\label{ORAC::Group_REVISION}\index{ORAC::Group!REVISION}}

\$Id$

\subsubsection*{COPYRIGHT\label{ORAC::Group_COPYRIGHT}\index{ORAC::Group!COPYRIGHT}}

Copyright (C) 1998-2000 Particle Physics and Astronomy Research
Council. All Rights Reserved.

\subsection{ORAC::Group::NDF\label{ORAC::Group::NDF}\index{ORAC::Group::NDF}}

Class for dealing with groups based on NDF files

\subsubsection*{SYNOPSIS\label{ORAC::Group::NDF_SYNOPSIS}\index{ORAC::Group::NDF!SYNOPSIS}}\begin{verbatim}
  use ORAC::Group::NDF
\end{verbatim}
\begin{verbatim}
  $Grp = new ORAC::Group::NDF;
\end{verbatim}
\subsubsection*{DESCRIPTION\label{ORAC::Group::NDF_DESCRIPTION}\index{ORAC::Group::NDF!DESCRIPTION}}

This class rovides implementations of the methods that require
knowledge of the NDF file format rather than generic methods or
methods that require knowledge of a specific instrument.  In general,
the specific instrument sub-classes will inherit from the file type
(which inherits from ORAC::Group) rather than directly from
ORAC::Group. For JCMT and UKIRT the group files are based on NDFs and
inherit from this class.



The format specific sub-classes do not contain constructors; they 
should be defined in either the base class or the instrument specific
sub-class.

\subsubsection*{PUBLIC METHODS\label{ORAC::Group::NDF_PUBLIC_METHODS}\index{ORAC::Group::NDF!PUBLIC METHODS}}

The following methods are modified from the base class versions.

\paragraph*{General Methods\label{ORAC::Group::NDF_General_Methods}\index{ORAC::Group::NDF!General Methods}}\begin{description}
\item[\textbf{coaddsread}] \mbox{}

Method to read the COADDS information from the group file. If the
Group file exists, the file is opened and the the \emph{.ORAC} manpage extension is
located. The .COADDS component (ie groupfile.MORE.ORAC.COADDS) is then
opened as \_INTEGER and the contents are stored in the group using the
coadds() method. If a .MORE.ORAC.COADDS component can not be found (e.g.
because the file or component do not exist), the routine returns
ORAC\_\_ERROR, else returns ORAC\_\_OK.

\begin{verbatim}
  $Grp->coaddsread;
\end{verbatim}


There are no arguments.

\item[\textbf{coaddswrite}] \mbox{}

Writes the current contents of coadds() into the current group file().
Returns ORAC\_\_OK if the coadds information was written successfully,
else returns ORAC\_\_ERROR.

\begin{verbatim}
  $Grp->coaddswrite;
\end{verbatim}


There are no arguments. The information is written to a .ORAC.COADDS
component in the Group file.  If coadds() contains no entries, all
coadds information is removed from the group file if present (and good
status is returned). A .ORAC extension is always made if one does not
exist and the file is present.

\item[\textbf{erase}] \mbox{}

Erases the current group file. Assumes a \texttt{.sdf} extension.
Returns ORAC\_\_OK if successful, ORAC\_\_ERROR otherwise.

\item[\textbf{file\_exists}] \mbox{}

Checks for the existence of the Group file(). Assumes a \texttt{.sdf}
extension.

\item[\textbf{readhdr}] \mbox{}

Reads the header from the reduced group file (the filename is stored
in the Group object) and sets the Group header. This method sets the
header in the object.

\begin{verbatim}
    $Grp->readhdr;
\end{verbatim}


All exisiting header information is lost.  If there is an error during
the read an empty hash is stored in the header.



Currently this method assumes that the reduced group is stored in
NDF format. Only the FITS header is retrieved from the NDF.



There are no input or return arguments.

\end{description}
\subsubsection*{PRIVATE METHODS\label{ORAC::Group::NDF_PRIVATE_METHODS}\index{ORAC::Group::NDF!PRIVATE METHODS}}

The following methods are intended for use inside the module.
They are included here so that authors of derived classes are 
aware of them.

\begin{description}
\item[\textbf{stripfname}] \mbox{}

Method to strip file extensions from the filename string. This method
is called by the file() method. For UKIRT we strip all extensions of the
form ".sdf", ".sdf.gz" and ".sdf.Z" since Starlink tasks do not require
the extension when accessing the file name.

\end{description}
\subsubsection*{REQUIREMENTS\label{ORAC::Group::NDF_REQUIREMENTS}\index{ORAC::Group::NDF!REQUIREMENTS}}

This module requires the the \emph{NDF} manpage module.

\subsubsection*{SEE ALSO\label{ORAC::Group::NDF_SEE_ALSO}\index{ORAC::Group::NDF!SEE ALSO}}

the \emph{ORAC::Group} manpage

\subsubsection*{REVISION\label{ORAC::Group::NDF_REVISION}\index{ORAC::Group::NDF!REVISION}}

\$Id$

\subsubsection*{COPYRIGHT\label{ORAC::Group::NDF_COPYRIGHT}\index{ORAC::Group::NDF!COPYRIGHT}}

Copyright (C) 1998-2000 Particle Physics and Astronomy Research
Council. All Rights Reserved.

\subsection{ORAC::Group::UFTI\label{ORAC::Group::UFTI}\index{ORAC::Group::UFTI}}

Class for dealing with UFTI observation groups in ORAC-DR

\subsubsection*{SYNOPSIS\label{ORAC::Group::UFTI_SYNOPSIS}\index{ORAC::Group::UFTI!SYNOPSIS}}\begin{verbatim}
  use ORAC::Group::UFTI;
\end{verbatim}
\begin{verbatim}
  $Grp = new ORAC::Group::UFTI("group1");
  $Grp->file("group_file")
  $Grp->readhdr;
  $value = $Grp->hdr("KEYWORD");
\end{verbatim}
\subsubsection*{DESCRIPTION\label{ORAC::Group::UFTI_DESCRIPTION}\index{ORAC::Group::UFTI!DESCRIPTION}}

This module provides methods for handling group objects that
are specific to UFTI. It provides a class derived from \textbf{ORAC::Group::NDF}.
All the methods available to ORAC::Group objects are available
to \textbf{ORAC::Group::UFTI} objects.

\subsubsection*{PUBLIC METHODS\label{ORAC::Group::UFTI_PUBLIC_METHODS}\index{ORAC::Group::UFTI!PUBLIC METHODS}}

The following methods are available in this class in addition to
those available from ORAC::Group.

\paragraph*{Constructor\label{ORAC::Group::UFTI_Constructor}\index{ORAC::Group::UFTI!Constructor}}\begin{description}
\item[\textbf{new}] \mbox{}

Create a new instance of a \textbf{ORAC::Group::UFTI} object.
This method takes an optional argument containing the
name of the new group. The object identifier is returned.

\begin{verbatim}
   $Grp = new ORAC::Group::UFTI;
   $Grp = new ORAC::Group::UFTI("group_name");
\end{verbatim}


This method calls the base class constructor but initialises
the group with a file suffix of '.sdf' and a fixed part
of 'g'.

\end{description}
\paragraph*{General Methods\label{ORAC::Group::UFTI_General_Methods}\index{ORAC::Group::UFTI!General Methods}}\begin{description}
\item[\textbf{calc\_orac\_headers}] \mbox{}

This method calculates header values that are required by the
pipeline by using values stored in the header.



An example is ORACTIME that should be set to the time of the
observation in hours. Instrument specific frame objects
are responsible for setting this value from their header.



Should be run after a header is set. Currently the hdr()
method calls this whenever it is updated.



Calculates ORACUT and ORACTIME



This method updates the frame header.
Returns a hash containing the new keywords.

\end{description}
\subsubsection*{SEE ALSO\label{ORAC::Group::UFTI_SEE_ALSO}\index{ORAC::Group::UFTI!SEE ALSO}}

the \emph{ORAC::Group} manpage, the \emph{ORAC::Group::NDF} manpage

\subsubsection*{REVISION\label{ORAC::Group::UFTI_REVISION}\index{ORAC::Group::UFTI!REVISION}}

\$Id$

\subsubsection*{COPYRIGHT\label{ORAC::Group::UFTI_COPYRIGHT}\index{ORAC::Group::UFTI!COPYRIGHT}}

Copyright (C) 1998-2000 Particle Physics and Astronomy Research
Council. All Rights Reserved.

\subsection{ORAC::Group::JCMT\label{ORAC::Group::JCMT}\index{ORAC::Group::JCMT}}

JCMT class for dealing with observation groups in ORAC-DR

\subsubsection*{SYNOPSIS\label{ORAC::Group::JCMT_SYNOPSIS}\index{ORAC::Group::JCMT!SYNOPSIS}}\begin{verbatim}
  use ORAC::Group::JCMT;
\end{verbatim}
\begin{verbatim}
  $Grp = new ORAC::Group::JCMT("group1");
  $Grp->file("group_file")
  $Grp->readhdr;
  $value = $Grp->hdr("KEYWORD");
\end{verbatim}
\subsubsection*{DESCRIPTION\label{ORAC::Group::JCMT_DESCRIPTION}\index{ORAC::Group::JCMT!DESCRIPTION}}

This module provides methods for handling group objects that
are specific to JCMT. It provides a class derived from \textbf{ORAC::Group::NDF}.
All the methods available to \textbf{ORAC::Group} objects are available
to \textbf{ORAC::Group::JCMT} objects. Some additional methods are supplied.

\subsubsection*{PUBLIC METHODS\label{ORAC::Group::JCMT_PUBLIC_METHODS}\index{ORAC::Group::JCMT!PUBLIC METHODS}}

The following methods are available in this class in addition to
those available from \textbf{ORAC::Group}.

\paragraph*{Constructor\label{ORAC::Group::JCMT_Constructor}\index{ORAC::Group::JCMT!Constructor}}\begin{description}
\item[\textbf{new}] \mbox{}

Create a new instance of a \textbf{ORAC::Group::JCMT} object.
This method takes an optional argument containing the
name of the new group. The object identifier is returned.

\begin{verbatim}
   $Grp = new ORAC::Group::JCMT;
   $Grp = new ORAC::Group::JCMT("group_name");
\end{verbatim}


This method calls the base class constructor but initialises
the group with a file suffix of '.sdf' and a fixed part
of '\_grp\_'.

\end{description}
\paragraph*{General Methods\label{ORAC::Group::JCMT_General_Methods}\index{ORAC::Group::JCMT!General Methods}}\begin{description}
\item[\textbf{calc\_orac\_headers}] \mbox{}

Calculate ORAC-specific headers from instrument specific FITS 
keywords.



Does nothing for JCMT (group headers are not used).

\item[\textbf{file}] \mbox{}

This is an extension to the default file() method.
This method accepts a root name for the group file
(independent of sub-instrument) - same as for the base 
class. If a number is supplied the root name is returned
with the appropriate extension relating to the 
sub-instrument order in the current frame.



The number to sub-instrument conversion uses the last frame in the
group to calculate the allowed number of sub-instruments and
the order. Note that this may well not be what you want.
Use the grpoutsub() method if you know the name of the sub-instrument.

\item[\textbf{file\_from\_bits}] \mbox{}

Method to return the group filename derived from a fixed
variable part (eg UT) and a group designator (usually obs
number). The full filename is returned (including suffix).

\begin{verbatim}
  $file = $Grp->file_from_bits("UT","num");
\end{verbatim}


Returns file of form UT\_grp\_00num.sdf



Note that this is the filename before sub-instruments
have been taken into account (essentially this is the
default root name for file() - the suffix is stripped).

\item[\textbf{gui\_id}] \mbox{}

The file identification for comparison with the \textbf{ORAC::Display}
system. Input argument is the file number (starting from 1).



This routine calculates the current suffix from the group file
name base and prepends a string 'gN' signifying that this is
a group observation and the Nth frame is requested (N is less than
or equal to nfiles()).



The assumption is that file() returns a root name (ie without
a sub-instrument designation). This then allows us to create an
ID based on number and suffix without having to chop the
sub-instrument name off the end.

\item[\textbf{nfiles}] \mbox{}

This method returns the number of files currently associated
with the group. What this in fact means is that it returns
the number of files associated with the last member of the 
group (since that is how I construct output names in the
first place). grpoutsub() method is responsible for 
converting this number into a filename via the file() method.

\end{description}
\subsubsection*{NEW METHODS\label{ORAC::Group::JCMT_NEW_METHODS}\index{ORAC::Group::JCMT!NEW METHODS}}

This section describes methods that are available to the
JCMT implementation of ORAC::Group.

\begin{description}
\item[\textbf{grpoutsub}] \mbox{}

Method to determine the group filename associated with
the supplied sub-instrument.



This method uses the file() method to determine the
group rootname and then tags it by the specified sub-instrument.

\begin{verbatim}
  $file = $Grp->grpoutsub($sub);
\end{verbatim}
\item[\textbf{membernamessub}] \mbox{}

Return list of file names associated with the specified
sub instrument.

\begin{verbatim}
  @names = $Grp->membernamessub($sub)
\end{verbatim}
\item[\textbf{subs}] \mbox{}

Returns an array containing all the sub instruments present
in the group (some frames may only have one sub-instrument)

\begin{verbatim}
  @subs = $Grp->subs;
\end{verbatim}


The frames should be able to invoke the subs() method.

\end{description}
\subsubsection*{SEE ALSO\label{ORAC::Group::JCMT_SEE_ALSO}\index{ORAC::Group::JCMT!SEE ALSO}}

the \emph{ORAC::Group} manpage

\subsubsection*{REVISION\label{ORAC::Group::JCMT_REVISION}\index{ORAC::Group::JCMT!REVISION}}

\$Id$

\subsubsection*{COPYRIGHT\label{ORAC::Group::JCMT_COPYRIGHT}\index{ORAC::Group::JCMT!COPYRIGHT}}

Copyright (C) 1998-2000 Particle Physics and Astronomy Research
Council. All Rights Reserved.

\subsection{ORAC::Index\label{ORAC::Index}\index{ORAC::Index}}

Perl routines for manipulating ORAC index files

\subsubsection*{SYNOPSIS\label{ORAC::Index_SYNOPSIS}\index{ORAC::Index!SYNOPSIS}}\begin{verbatim}
 use ORAC::Index;
\end{verbatim}
\subsubsection*{DESCRIPTION\label{ORAC::Index_DESCRIPTION}\index{ORAC::Index!DESCRIPTION}}

This module provides subs for manipulating ORAC index files. ORAC
index files consist of whitespace seperated columns containing
information about a particular frame.



In the case of calibration index files, these may also contain rules
for determining the suitability of use for these frames. These consist
of code that is TRUE or FALSE depending on appropriate header values
of the object to be calibrated.

\subsubsection*{PUBLIC METHODS\label{ORAC::Index_PUBLIC_METHODS}\index{ORAC::Index!PUBLIC METHODS}}

The following methods are available in this class.

\paragraph*{Constructor\label{ORAC::Index_Constructor}\index{ORAC::Index!Constructor}}\begin{description}
\item[\textbf{new}] \mbox{}

Create a new instance of an \textbf{ORAC::Index} object.

\begin{verbatim}
  $Index = new ORAC::Index;
  $Index = new ORAC::Index($rulesfile, $indexfile);
\end{verbatim}


Any arguments are passed to the configure() method.

\end{description}
\paragraph*{Accessor Methods\label{ORAC::Index_Accessor_Methods}\index{ORAC::Index!Accessor Methods}}\begin{description}
\item[\textbf{configure}] \mbox{}

Takes an index file and a rules file and sets up the index object

\begin{verbatim}
  $Index->configure($indexfile, $rulesfile);
\end{verbatim}
\item[\textbf{indexfile}] \mbox{}

Return (or set) the filename of the index file

\begin{verbatim}
  $file = $Index->indexfile;
  $Index->indexfile($file);
\end{verbatim}
\item[\textbf{indexrulesfile}] \mbox{}

Return (or set) the filename of the rules file

\item[\textbf{rulesref}] \mbox{}

Returns or sets the reference to the hash containing the rules

\item[\textbf{indexref}] \mbox{}

Returns or sets the reference to the hash containing the index

\end{description}
\paragraph*{General Methods\label{ORAC::Index_General_Methods}\index{ORAC::Index!General Methods}}\begin{description}
\item[slurprules] \mbox{}

Sets up the index rules in the object. Croaks if it fails.
This converts the index rules file into an internal hash
that can be retrieved with the rulesref() method.

\item[\textbf{slurpindex}] \mbox{}

Sets up the index data in the object. Croaks if it fails.  This
converts the index file name into an internal hash that can be
retrieved using the indexref() method.  There is one optional
argument.  The supplied argument is used to control the behaviour of
the read. If the 'usekey' flag is true the first string in each row
(space separated) is used as a key for the index hash.



If 'usekey' is false the key for each row is created 
automatically. This is useful for indexes where the contents 
of the index is more important than any particular key.

\begin{verbatim}
  $index->slurpindex(0); # Auto-generate keys
\end{verbatim}


Default behaviour (ie no args) is to read the key from the
index file (ie usekey=1).

\item[\textbf{writeindex}] \mbox{}

writes out the current state of the index object into the index file

\item[\textbf{add}] \mbox{}

adds an entry to an index

\begin{verbatim}
  $index->add($name,$hashref)
\end{verbatim}
\item[\textbf{append\_to\_index}] \mbox{}

Method to force an append of the specified index entry to the
the index file on disk.

\begin{verbatim}
  $Index->append_to_index($name);
\end{verbatim}


\$name is the name of the key (indexentry) to use to select the
index entry to append [cf the indexentry() method].



This method is intended to be called from the add() method
to speed up index read/write when appending a new entry.
Do not use this method to write a modified entry to the
index file (since the original entry will still be on disk)



No return value.

\item[\textbf{index\_to\_text}] \mbox{}

Convert an index entry (in the index hash) to text suitable for
writing to an index file. Called by writeindex() and append\_to\_index()

\begin{verbatim}
  $text = $Ind->index_to_text($entry);
\end{verbatim}


Returns the text string (including the entry name but no carriage 
return).

\item[\textbf{indexentry}] \mbox{}

Returns a hash containing the key value pairs of the
selected index entry.



Input argument is the index entry name (ie the key in the hash
that returns the information (in an array).



Returns a hash reference if successful, undef if error.

\item[\textbf{verify}] \mbox{}

verifies a frame (in the form of a hash reference) against a 
(calibration) index entry (ie by supplying the hash key to the index
entry). An optional third argument is available to turn off warning 
messages -- default is for warning messages to be turned on (true)

\begin{verbatim}
  $result = $index->verify(indexkey, \%hash, $warn);
\end{verbatim}


Returns undef (error), 0 (not suitable), or 1 (suitable)

\item[\textbf{choosebydt}] \mbox{}

Chooses the optimal (nearest in time to an observation) calibration
frame from the index hash

\begin{verbatim}
  $calibration = $Index->choosebydt($key, \%header, $warn);
\end{verbatim}


Key is the name of the field that should be compared (eg ORACTIME)
and \%header is the hash containing the header values that are to
be compared with the index rules. \$warn is an optional third argument
that can be used to turn off warning messages from verify (default
is to report messages - true).



This method returns the name of the calibration frame closest in 
time that has met the selection criteria.



If a suitable calibration can not be found an undefined value is returned.

\item[\textbf{chooseby\_positivedt}] \mbox{}

Chooses the calibration frame closest in time from above by looking 
in the index file (ie difference between the index file entry and
the current frame is positive).

\begin{verbatim}
  $calibration = $Index->chooseby_positivedt($key, \%header, $warn);
\end{verbatim}


Key is the name of the field that should be compared (eg ORACTIME)
and \%header is the hash containing the header values that are to
be compared with the index rules. \$warn is an optional third argument
that can be used to turn off warning messages from verify (default
is to report messages - true).



This method returns the name of the calibration frame closest in 
time that has met the selection criteria.



This is similar to the choosebydt() method except that only
calibrations taken after the current time (read from the
header) can be chosen. undef is returned if no suitable
calibration frames can be found (eg because we are running
on-line and they have not even been taken yet).

\item[\textbf{chooseby\_negativedt}] \mbox{}

Chooses the calibration frame closest in time from below by looking 
in the index file (ie delta time between the index entry and the 
current frame is negative).

\begin{verbatim}
  $calibration = $Index->chooseby_negativedt($key, \%header, $warn);
\end{verbatim}


Key is the name of the field that should be compared (eg ORACTIME)
and \%header is the hash containing the header values that are to
be compared with the index rules. \$warn is an optional third argument
that can be used to turn off warning messages from verify (default
is to report messages - true).



This method returns the name of the calibration frame closest in 
time that has met the selection criteria.



This is similar to the choosebydt() method except that only
calibrations taken before the current time (read from the
header) can be chosen. undef is returned if no suitable 
calibration can be found.

\item[\textbf{choosebydt\_generic}] \mbox{}

Internal routine for handling calibraion matches using a 
time difference.

\begin{verbatim}
  $calibration = $Index->choosebydt_generic(TYPE, $key, \%header, $warn);
\end{verbatim}


TYPES can be 'ABS' (chooses the closest calibration in time), 
'POSITIVE' (chooses the closest in time from calibrations earlier
than the current header) and 'NEGATIVE' (chooses calibrations after
the current observation [as described by \%header]).



KEY, HEADER and WARN are described in the choosebydt() documentation.

\item[\textbf{cmp\_with\_hash}] \mbox{}

Compares each index entry with the values in the supplied hash
(supplied as a hash reference). The key to the first matching 
index entry is returned. undef is returned if no match could be 
found.

\begin{verbatim}
  $key = $index->cmp_with_hash(\%hash);
  $key = $index->cmp_with_hash({ key1 => 'value',
                                 key2 => 'value2'});
\end{verbatim}


Use the indexentry() method to convert this key into the actual
index entry. Note that warning messages are turned off during the
verification stage since we are not interested in failed matches.



Returns 'undef' if no match is found or if no argument is supplied
[or that argument itself is undef]

\end{description}
\subsubsection*{SEE ALSO\label{ORAC::Index_SEE_ALSO}\index{ORAC::Index!SEE ALSO}}

the \emph{ORAC::Index::Extern} manpage

\subsubsection*{REVISION\label{ORAC::Index_REVISION}\index{ORAC::Index!REVISION}}

\$Id$

\subsubsection*{COPYRIGHT\label{ORAC::Index_COPYRIGHT}\index{ORAC::Index!COPYRIGHT}}

Copyright (C) 1998-2000 Particle Physics and Astronomy Research
Council. All Rights Reserved.

\subsection{ORAC::LogFile\label{ORAC::LogFile}\index{ORAC::LogFile}}

Routines for generating log files

\subsubsection*{SYNOPSIS\label{ORAC::LogFile_SYNOPSIS}\index{ORAC::LogFile!SYNOPSIS}}\begin{verbatim}
  use ORAC::LogFile;
\end{verbatim}
\begin{verbatim}
  $log = new ORAC::LogFile('logfile.dat');
  $log->header(@header);
  $log->addentry(@lines);
  $log->timestamp(1);
\end{verbatim}
\subsubsection*{DESCRIPTION\label{ORAC::LogFile_DESCRIPTION}\index{ORAC::LogFile!DESCRIPTION}}

Provide simple interface to generation of logfiles (eg logging
of seeing statistics, photometry results or pointing logs).

\subsubsection*{PUBLIC METHODS\label{ORAC::LogFile_PUBLIC_METHODS}\index{ORAC::LogFile!PUBLIC METHODS}}

The following methods are available:

\begin{description}
\item[\textbf{new}] \mbox{}

Create a new instance of ORAC::LogFile and associate it with the 
specified log file.

\begin{verbatim}
  $log = new ORAC::LogFile($logfile);
\end{verbatim}


If no argument is supplied, the logfile name must be set explcitly
by using the logfile() method.



This constructor does not create the logfile itself.

\item[\textbf{logfile}] \mbox{}

Return or set the name of the logfile associated with
this instance. Usually set directly by the constructor.

\begin{verbatim}
  $logfile = $log->logfile;
  $log->logfile($logfile);
\end{verbatim}
\item[\textbf{timestamp}] \mbox{}

Control whether a timestamp is prepended to each entry
written to the logfile. Default is to not print a timestamp.

\begin{verbatim}
  $log->timestamp(1);
  $use = $log->timestamp;
\end{verbatim}


The timestamp will be in UT.

\item[\textbf{header}] \mbox{}

Write header information to the file. Header information is only
written if the logfile does not previously exist (since if the file
exists already a header is not required). If the logfile does not
exist the logfile is created by this method and all arguments written
to it.  A newline character "$\backslash$n" is automatically appended to each
line.

\begin{verbatim}
  $log->header($line1, $line2);
  $log->header(@lines);
\end{verbatim}
\item[\textbf{addentry}] \mbox{}

Add a log entry. Multiple lines can be supplied (eg as an array).
Each line is appended to the logfile (appending a newline "$\backslash$n"
character to each and prepending a timestamp if required).

\begin{verbatim}
  $log->addentry($line);
  $log->addentry(@lines);
\end{verbatim}


The logfile is closed each time this method is invoked.

\end{description}
\subsubsection*{REVISION\label{ORAC::LogFile_REVISION}\index{ORAC::LogFile!REVISION}}

\$Id$

\subsubsection*{COPYRIGHT\label{ORAC::LogFile_COPYRIGHT}\index{ORAC::LogFile!COPYRIGHT}}

Copyright (C) 1998-2000 Particle Physics and Astronomy Research
Council. All Rights Reserved.

\subsection{ORAC::Loop\label{ORAC::Loop}\index{ORAC::Loop}}

Data loops for ORACDR

\subsubsection*{SYNOPSIS\label{ORAC::Loop_SYNOPSIS}\index{ORAC::Loop!SYNOPSIS}}\begin{verbatim}
  use ORAC::Loop;
\end{verbatim}
\begin{verbatim}
  $frm = orac_loop_list($class, $utdate, \@list, $skip);
\end{verbatim}
\begin{verbatim}
  $frm = orac_loop_inf($class, $utdate, \@list);
\end{verbatim}
\begin{verbatim}
  $frm = orac_loop_wait($class, $utdate, \@list, $skip);
\end{verbatim}
\begin{verbatim}
  $frm = orac_loop_flag($class, $utdate, \@list, $skip);
\end{verbatim}
\subsubsection*{DESCRIPTION\label{ORAC::Loop_DESCRIPTION}\index{ORAC::Loop!DESCRIPTION}}

This module provides a set of loop handling routines for ORACDR.
Each subroutine accepts the same arguments and returns the current
observation number (or undef if there was an error or if the loop
should be terminated).



A new  Frame object is returned of class \$class that has been configured
for the new file (ie a \texttt{\$Frm-$>$configure} method has been run)



It is intended that this routine is called inside an infinite while
loop with the same @list array. This array is modified by the loop
routines so that they can keep track of the 'next' frame number.



If a filename can not be found (eg it doesnt exist or the list has
been processed) undef is returned.



The skip flag is used to indicate whether the loop should skip
forward if the current observation number can not be found
but a higher numbered observation is present. Currently no loops
will go back to missing observations if they appear after a higher
number (eg observation 10 appears before observation 9!)

\subsubsection*{LOOP SUBROUTINES\label{ORAC::Loop_LOOP_SUBROUTINES}\index{ORAC::Loop!LOOP SUBROUTINES}}

The following loop facilities are available:

\begin{description}
\item[\textbf{orac\_loop\_list}] \mbox{}

Takes a list of numbers and returns back a frame object 
for each number (one frame object per call)

\begin{verbatim}
  $Frm = orac_loop_list($class, $UT, \@array, $noskip);
\end{verbatim}


undef is returned on error or when all members of the
list have been returned. If the 'skip' flag is true
missing files in the list will be ignored and the next
element of the list selected. If 'skip' is false
the loop will abort if the file is not present

\item[\textbf{orac\_loop\_inf}] \mbox{}

Checks for the frame stored in the first element of the supplied array
and returns the Frame object if the file exists. The number is incremented
such that the next observation is returned next time the routine is
called.

\begin{verbatim}
  $Frm = orac_loop_inf($class, $ut, \@array);
\end{verbatim}


undef is returned on error or when there are no more data files
available.



This loop does not have a facility for skipping files when observations
are not present. This behaviour is obtained by combining 
orac\_check\_data\_dir with the list looping option so that the last
observation number can be determined before running the loop. The skip
flag is ignored in this loop.

\item[\textbf{orac\_loop\_wait}] \mbox{}

Waits for the specified file to appear in the directory.
A timeout of 60 minutes is hard-wired in initially -- undef
is returned if the timeout is exceeded.

\begin{verbatim}
  $frm = orac_loop_wait($class, $utdate, \@list, $skip);
\end{verbatim}


The first member of the array is used to keep track of the
current observation number. This element is incremented so that
the following observation is returned when the routine is called
subsequently. This means that this loop is similar to using the
'-from' option in conjunction with the 'inf' loop except that
new data is expected.



The loop will return undef (i.e. terminate looping) if the
supplied array contains undef in the first entry.



The skip flag is used to indicate whether the loop should skip
forward if the current observation number can not be found
but a higher numbered observation is present.



If no data can be found, the directory is scanned every few seconds
(hard-wired into the routine). A dot is printed to the screen after
a specified number of scans (default is 1 dot per scan and one scan every
2 seconds).

\item[\textbf{orac\_loop\_flag}] \mbox{}

Waits for the specified file to appear in the directory
by looking for the appearance of the associated flag file.
A timeout of 60 minutes is hard-wired in initially -- undef
is returned if the timeout is exceeded.

\begin{verbatim}
  $frm = orac_loop_flag($class, $utdate, \@list, $skip);
\end{verbatim}


The first member of the array is used to keep track of the
current observation number. This element is incremented so that
the following observation is returned when the routine is called
subsequently. This means that this loop is similar to using the
'-from' option in conjunction with the 'inf' loop except that
new data is expected.



The loop will return undef (i.e. terminate looping) if the
supplied array contains undef in the first entry.

\end{description}
\subsubsection*{OTHER EXPORTED SUBROUTINES\label{ORAC::Loop_OTHER_EXPORTED_SUBROUTINES}\index{ORAC::Loop!OTHER EXPORTED SUBROUTINES}}\begin{description}
\item[\textbf{orac\_check\_data\_dir}] \mbox{}

Routine to check the input data directory (ORAC\_DATA\_IN) for
files in order to see whether files exist with a higher number
than the supplied number. The routine is supplied with a class name,
UT date and current observation number. An additional argument
is provided to determine whether data files or flag files should
be used for the directory search.

\begin{verbatim}
   $next = orac_check_data_dir($class, $current, $flag);
   ($next, $high) = orac_check_data_dir($class, $current, $flag);
\end{verbatim}


If called in a scalar context, the return argument is the next 
observation in the sequence. If called in an array context, two
arguments are returned: the next observation number and the highest 
observation number.



undef (or undef,undef) is returned if no higher observations can be
found. If it is necessary to check for the existence of current
file as well (eg via a data detection loop) then simply decrement the
supplied argument by 1.



This routine is used in conjunction with the -from loop (where we
dont know the end) and the waiting loops where we are not sure whether
new data have been written to disk but missing the next observation.



This routine does NOT look in ORAC\_DATA\_OUT.



A global variables (@LIST) is used to speed up the sorting by storing
a list of observation numbers that have previously been shown to have a lower
number than required (NOT YET IMPLEMENTED).

\end{description}
\subsubsection*{PRIVATE SUBROUTINES\label{ORAC::Loop_PRIVATE_SUBROUTINES}\index{ORAC::Loop!PRIVATE SUBROUTINES}}

The following subroutines are not exported.

\begin{description}
\item[\textbf{link\_and\_read}] \mbox{}

General subroutine for converting ut and number into file
and creating a Frame object.

\begin{verbatim}
  $frm = link_and_read($class, $ut, $obsnum)l
\end{verbatim}


undef is returned on error.
A configured Frame object is returned if everything is okay

\item[\textbf{orac\_sleep}] \mbox{}

Pause the checking for new data files by the specified number of seconds.

\begin{verbatim}
  $time = orac_sleep($pause);
\end{verbatim}


Where \$pause is the number of seconds to wait and \$time is the number
of seconds actually waited (see the sleep() command for more details).



If the Tk system is loaded this routine will actually do a Tk event loop
for the required number of seconds. This is so that the X screen will
be refreshed. Currently the only test is the Tk is loaded, not that
we are actually using Tk.....

\end{description}
\subsubsection*{REVISION\label{ORAC::Loop_REVISION}\index{ORAC::Loop!REVISION}}

\$Id$

\subsubsection*{COPYRIGHT\label{ORAC::Loop_COPYRIGHT}\index{ORAC::Loop!COPYRIGHT}}

Copyright (C) 1998-2000 Particle Physics and Astronomy Research
Council. All Rights Reserved.

\subsection{ORAC::Msg::ADAM::Control\label{ORAC::Msg::ADAM::Control}\index{ORAC::Msg::ADAM::Control}}

Control and initialise ADAM messaging from ORAC

\subsubsection*{SYNOPSIS\label{ORAC::Msg::ADAM::Control_SYNOPSIS}\index{ORAC::Msg::ADAM::Control!SYNOPSIS}}\begin{verbatim}
  use ORAC::Msg::ADAM::Control;
\end{verbatim}
\begin{verbatim}
  $ams = new ORAC::Msg::ADAM::Control(1);
  $ams->init;
\end{verbatim}
\begin{verbatim}
  $ams->messages(0);
  $ams->errors(1);
  $ams->timeout(30);
  $ams->stderr(\*ERRHANDLE);
  $ams->stdout(\*MSGHANDLE);
  $ams->paramrep( sub { return "!" } );
\end{verbatim}
\subsubsection*{DESCRIPTION\label{ORAC::Msg::ADAM::Control_DESCRIPTION}\index{ORAC::Msg::ADAM::Control!DESCRIPTION}}

Methods to initialise the ADAM messaging system (AMS) )and control the
behaviour.

\subsubsection*{METHODS\label{ORAC::Msg::ADAM::Control_METHODS}\index{ORAC::Msg::ADAM::Control!METHODS}}

The following methods are available:

\paragraph*{Constructor\label{ORAC::Msg::ADAM::Control_Constructor}\index{ORAC::Msg::ADAM::Control!Constructor}}\begin{description}
\item[\textbf{new}] \mbox{}

Create a new instance of Starlink::AMS::Init.
If a true argument is supplied the messaging system is also
initialised via the init() method.

\end{description}
\paragraph*{Accessor Methods\label{ORAC::Msg::ADAM::Control_Accessor_Methods}\index{ORAC::Msg::ADAM::Control!Accessor Methods}}\begin{description}
\item[\textbf{messages}] \mbox{}

Method to set whether standard messages returned from monoliths
are printed or not. If set to true the messages are printed
else they are ignored.

\begin{verbatim}
  $current = $ams->messages;
  $ams->messages(0);
\end{verbatim}


Default is to print all messages.

\item[\textbf{errors}] \mbox{}

Method to set whether error messages returned from monoliths
are printed or not. If set to true the errors are printed
else they are ignored.

\begin{verbatim}
  $current = $ams->errors;
  $ams->errors(0);
\end{verbatim}


Default is to print all messages.

\item[\textbf{timeout}] \mbox{}

Set or retrieve the timeout (in seconds) for some of the ADAM messages.
Default is 30 seconds.

\begin{verbatim}
  $ams->timeout(10);
  $current = $ams->timeout;
\end{verbatim}
\item[\textbf{stderr}] \mbox{}

Set and retrieve the current filehandle to be used for printing
error messages. Default is to use STDERR.

\item[\textbf{stdout}] \mbox{}

Set and retrieve the current filehandle to be used for printing
normal ADAM messages. Default is to use STDOUT. This can be
a tied filehandle (eg one generated by ORAC::Print).

\item[\textbf{paramrep}] \mbox{}

Set and retrieve the code reference that will be executed if
the parameter system needs to ask for a parameter.
Default behaviour is to call a routine that simply prompts
the user for the required value. The supplied subroutine
should accept three arguments (the parameter name, prompt string and
default value) and should return the required value.

\begin{verbatim}
  $self->paramrep(\&mysub);
\end{verbatim}


A simple check is made to make sure that the supplied argument
is a code reference.



Warning: It is possible to get into an infinite loop if you try
to continually return an unacceptable answer.

\end{description}
\paragraph*{General Methods\label{ORAC::Msg::ADAM::Control_General_Methods}\index{ORAC::Msg::ADAM::Control!General Methods}}\begin{description}
\item[\textbf{init}] \mbox{}

Initialises the ADAM messaging system. This routine should always be
called before attempting to control I-tasks.



A relay task is spawned in order to test that the messaging system
is functioning correctly. The relay itself is not necessary for the
non-event loop implementation. If this command hangs then it is
likely that the messaging system is not running correctly (eg
because the system was shutdown uncleanly - try removing named pipes
from the \texttt{\~{}}/adam directory).

\end{description}
\subsubsection*{VARIABLES\label{ORAC::Msg::ADAM::Control_VARIABLES}\index{ORAC::Msg::ADAM::Control!VARIABLES}}

The ORAC::Msg::ADAM::Control::RUNNING variable can be 
used to determine whether the message system is running or not.
(Multiple message system objects can be created although only
the first will actually start the message system - an error is raised
if multiple objects are created).

\subsubsection*{REQUIREMENTS\label{ORAC::Msg::ADAM::Control_REQUIREMENTS}\index{ORAC::Msg::ADAM::Control!REQUIREMENTS}}

This module requires the Starlink::AMS::Init module.

\subsubsection*{SEE ALSO\label{ORAC::Msg::ADAM::Control_SEE_ALSO}\index{ORAC::Msg::ADAM::Control!SEE ALSO}}

the \emph{Starlink::AMS::Init} manpage

\subsubsection*{REVISION\label{ORAC::Msg::ADAM::Control_REVISION}\index{ORAC::Msg::ADAM::Control!REVISION}}

\$Id$

\subsubsection*{COPYRIGHT\label{ORAC::Msg::ADAM::Control_COPYRIGHT}\index{ORAC::Msg::ADAM::Control!COPYRIGHT}}

Copyright (C) 1998-2000 Particle Physics and Astronomy Research
Council. All Rights Reserved.

\subsection{ORAC::Msg::ADAM::Task\label{ORAC::Msg::ADAM::Task}\index{ORAC::Msg::ADAM::Task}}

Load and control ADAM tasks

\subsubsection*{SYNOPSIS\label{ORAC::Msg::ADAM::Task_SYNOPSIS}\index{ORAC::Msg::ADAM::Task!SYNOPSIS}}\begin{verbatim}
  use ORAC::Msg::ADAM::Task;
\end{verbatim}
\begin{verbatim}
  $kap = new ORAC::Msg::ADAM::Task("kappa","/star/bin/kappa/kappa_mon");
\end{verbatim}
\begin{verbatim}
  $status           = $kap->obeyw("task", "params");
  $status           = $kap->set("task", "param","value");
  ($status, @values) = $kap->get("task", "param");
  ($dir, $status)   = $kap->control("default","dir");
  $kap->control("par_reset");
  $kap->resetpars;
  $kap->cwd("dir");
  $cwd = $kap->cwd;
\end{verbatim}
\subsubsection*{DESCRIPTION\label{ORAC::Msg::ADAM::Task_DESCRIPTION}\index{ORAC::Msg::ADAM::Task!DESCRIPTION}}

Provide methods for loading and communicating with ADAM monoliths.
This module conforms to the ORAC messaging standard. This is an
ORAC interface to the Starlink::AMS::Task module.



By default all tasks loaded by this module will be terminated
on exit from perl.

\subsubsection*{METHODS\label{ORAC::Msg::ADAM::Task_METHODS}\index{ORAC::Msg::ADAM::Task!METHODS}}

The following methods are available:

\paragraph*{Constructor\label{ORAC::Msg::ADAM::Task_Constructor}\index{ORAC::Msg::ADAM::Task!Constructor}}\begin{description}
\item[\textbf{new}] \mbox{}

Create a new instance of a ORAC::Msg::ADAM::Task object.

\begin{verbatim}
  $obj = new ORAC::Msg::ADAM::Task;
  $obj = new ORAC::Msg::ADAM::Task("name_in_message_system","monolith");
  $obj = new ORAC::Msg::ADAM::Task("name_in_message_system","monolith"
                                    { TASKTYPE => 'A'} );
\end{verbatim}


If supplied with arguments (matching those expected by load() ) the
specified task will be loaded upon creating the object. If the load()
fails then undef is returned (which will not be an object reference).

\end{description}
\paragraph*{General Methods\label{ORAC::Msg::ADAM::Task_General_Methods}\index{ORAC::Msg::ADAM::Task!General Methods}}\begin{description}
\item[\textbf{load}] \mbox{}

Load a monolith and set up the name in the messaging system.
This task is called by the 'new' method.

\begin{verbatim}
  $status = $obj->load("name","monolith_binary",{ TASKTYPE => 'A' });
\end{verbatim}


If the second argument is omitted it is assumed that the binary
is already running and can be called by "name".



If a path to a binary with name "name" already exists then the monolith
is not loaded.



Options (in the form of a hash reference) can be supplied
in order to configure the monolith. Currently supported options
are

\begin{verbatim}
  TASKTYPE  - can be 'A' for A-tasks or 'I' for I-tasks
\end{verbatim}
\item[\textbf{obeyw}] \mbox{}

Send an obey to a task and wait for a completion message

\begin{verbatim}
  $status = $obj->obeyw("action","params");
\end{verbatim}
\item[\textbf{get}] \mbox{}

Obtain the value of a parameter

\begin{verbatim}
 ($status, @values) = $obj->get("task", "param");
\end{verbatim}
\item[\textbf{set}] \mbox{}

Set the value of a parameter

\begin{verbatim}
  $status = $obj->set("task", "param", "newvalue");
\end{verbatim}
\item[\textbf{control}] \mbox{}

Send CONTROL messages to the monolith. The type of control
message is specified via the first argument. Allowed values are:

\begin{verbatim}
  default:  Return or set the current working directory
  par_reset: Reset all parameters associated with the monolith.
\end{verbatim}
\begin{verbatim}
  ($current, $status) = )$obj->control("type", "value")
\end{verbatim}


"value" is only relevant for the "default" type and is used
to specify a new working directory. \$current is always returned
even if it is undefined.



These commands are synonymous with the cwd() and resetpars()
methods.

\item[\textbf{resetpars}] \mbox{}

Reset all parameters associated with a monolith

\begin{verbatim}
  $status = $obj->resetpars;
\end{verbatim}
\item[\textbf{cwd}] \mbox{}

Set and retrieve the current working directory of the monolith

\begin{verbatim}
  ($cwd, $status) = $obj->cwd("newdir");
\end{verbatim}
\item[\textbf{contactw}] \mbox{}

This method will not return unless the monolith can be contacted.
It only returns with a timeout. Returns a '1' if we contacted okay
and a '0' if we timed out. It will timeout if it takes longer than
specified in \texttt{ORAC::Msg::ADAM::Control-$>$timeout}.

\item[\textbf{contact}] \mbox{}

This method can be used to determine whether the object can
contact a monolith. Returns a 1 if we can contact a monolith and
a zero if we cant.

\item[\textbf{pid}] \mbox{}

Returns process id of forked task.
Returns undef if there is no external task.

\end{description}
\subsubsection*{REQUIREMENTS\label{ORAC::Msg::ADAM::Task_REQUIREMENTS}\index{ORAC::Msg::ADAM::Task!REQUIREMENTS}}

This module requires the Starlink::AMS::Task module.

\subsubsection*{SEE ALSO\label{ORAC::Msg::ADAM::Task_SEE_ALSO}\index{ORAC::Msg::ADAM::Task!SEE ALSO}}

the \emph{Starlink::AMS::Task} manpage

\subsubsection*{REVISION\label{ORAC::Msg::ADAM::Task_REVISION}\index{ORAC::Msg::ADAM::Task!REVISION}}

\$Id$

\subsubsection*{COPYRIGHT\label{ORAC::Msg::ADAM::Task_COPYRIGHT}\index{ORAC::Msg::ADAM::Task!COPYRIGHT}}

Copyright (C) 1998-2000 Particle Physics and Astronomy Research
Council. All Rights Reserved.

\subsection{ORAC::Print\label{ORAC::Print}\index{ORAC::Print}}

ORAC output message printing

\subsubsection*{SYNOPSIS\label{ORAC::Print_SYNOPSIS}\index{ORAC::Print!SYNOPSIS}}\begin{verbatim}
  use ORAC::Print qw/:func/;
\end{verbatim}
\begin{verbatim}
  orac_print("text",'magenta');
  orac_err("error text",'red');
\end{verbatim}
\begin{verbatim}
  orac_err("error text");
  orac_print("some text");
  orac_warn("some warning");
\end{verbatim}
\begin{verbatim}
  $value = orac_read("Prompt");
\end{verbatim}
\begin{verbatim}
  $prt = new ORAC::Print;
  $prt->out("Message","colour");
  $prt->err("Error message"); 
  $prt->war("warning message");
  $prt->errcol("red");
  $prt->outcol("magenta");
  $prt->errbeep(1);
\end{verbatim}
\begin{verbatim}
  tie *HANDLE, 'ORAC::Print', $ptr;
\end{verbatim}
\subsubsection*{DESCRIPTION\label{ORAC::Print_DESCRIPTION}\index{ORAC::Print!DESCRIPTION}}

This module provides commands for printing messages from ORAC
software. Commands are provided for printing error messages, warning
messages and information messages. The final output location of these
messages is controlled by the object configuration.



If the \texttt{ORAC::Print::TKMW} variable is set, it is assumed that this
is the Tk object referring to the MainWindow, and the
\texttt{Tk-$>$update()} method is run whenever the \texttt{orac\_*} commands are
executed.  This can be used to keep a Tk log window updating even
though no X-events are being processed.



A simplified interface to Term::ReadLine is provided for use with
the orac\_read command. This can only be used on STDIN/STDOUT and
is not object-oriented.

\subsubsection*{NON-OO INTERFACE\label{ORAC::Print_NON-OO_INTERFACE}\index{ORAC::Print!NON-OO INTERFACE}}

A simplified non-object oriented interface is provided.
These routines are exported into the callers namespace by default
and are the commands that should be used by primitive writers.

\begin{description}
\item[orac\_print ( text , [colour])] \mbox{}

Print the supplied text to the ORAC output device(s)
using the (optional) supplied colour.



If the colour is not specified the default value is used (magenta
for primtives).

\item[orac\_warn( text, [colour])] \mbox{}

Print the supplied text as a warning message using the supplied
colour.

\item[orac\_err( text, [colour])] \mbox{}

Print the supplied text as an error message using the supplied
colour.

\item[orac\_debug( text)] \mbox{}

Print the supplied text as a debug message using the supplied
colour.

\item[orac\_read(prompt)] \mbox{}

Read a value from standard input. This is simply a layer
on top of Term::ReadLine.

\begin{verbatim}
  $value = orac_read($prompt);
\end{verbatim}


There is no Object-oriented version of this routine. It always
uses STDIN for input and STDOUT for output.

\end{description}
\subsubsection*{OO INTERFACE\label{ORAC::Print_OO_INTERFACE}\index{ORAC::Print!OO INTERFACE}}

The following methods are available:

\paragraph*{Constructors\label{ORAC::Print_Constructors}\index{ORAC::Print!Constructors}}\begin{description}
\item[new()] \mbox{}

Object constructor. The object is returned.

\end{description}
\paragraph*{Instance Methods\label{ORAC::Print_Instance_Methods}\index{ORAC::Print!Instance Methods}}\begin{description}
\item[debugmsg] \mbox{}

Turns debugging messages on or off. Default is off.

\item[outcol(colour)] \mbox{}

Retrieve (or set) the colour currently used for printing output
messages.

\begin{verbatim}
  $col = $prt->outcol;
  $prt->outcol('red');
\end{verbatim}


Currently no check is made that the supplied colour is acceptable.

\item[warncol(colour)] \mbox{}

Retrieve (or set) the colour currently used for printing warning
messages.

\begin{verbatim}
  $col = $prt->warncol;
  $prt->warncol('red');
\end{verbatim}


Currently no check is made that the supplied colour is acceptable.

\item[errcol(colour)] \mbox{}

Retrieve (or set) the colour currently used for printing error
messages.

\begin{verbatim}
  $col = $prt->errcol;
  $prt->outcol('red');
\end{verbatim}


Currently no check is made that the supplied colour is acceptable.

\item[prefix] \mbox{}

String that is prepended to all messages printed by this class.
Default is to have no prefix.

\begin{verbatim}
  $prefix = $prt->prefix;
  $prt->prefix('Obs52');
\end{verbatim}
\item[outpre] \mbox{}

Prefix that is prepended to all strings printed with the
out() method. Default is to have no prefix.

\begin{verbatim}
  $pre = $prt->outpre;
  $prt->outpre('ORAC says:');
\end{verbatim}
\item[warpre] \mbox{}

Prefix that is prepended to all strings printed with the
out() method. Default is to have the string 'Warning:' prepended.

\begin{verbatim}
  $pre = $prt->warpre;
  $prt->warpre('ORAC Warning:');
\end{verbatim}
\item[errpre] \mbox{}

Prefix that is prepended to all strings printed with the
out() method. Default is to have the string 'Error:' prepended.

\begin{verbatim}
  $pre = $prt->errpre;
  $prt->errpre('ORAC Error:');
\end{verbatim}
\item[outhdl] \mbox{}

Output file handle(s). These are the filehandles that are used
to send all output messages. Multiple filehandles can be supplied.
Returns an IO::Tee object that can be used as a single filehandle.

\begin{verbatim}
  $Prt->outhdl(\*STDOUT, $fh);
\end{verbatim}
\begin{verbatim}
  $fh = $Prt->outhdl;
\end{verbatim}


Default is to use STDOUT.

\item[warhdl] \mbox{}

Warning output file handle(s). These are the filehandles that are used
to print all warning messages. Multiple filehandles can be supplied.
Returns an IO::Tee object that can be used as a single filehandle.

\begin{verbatim}
  $Prt->warhdl(\*STDOUT, $fh);
\end{verbatim}
\begin{verbatim}
  $fh = $Prt->warhdl;
\end{verbatim}


Default is to use STDOUT.

\item[errhdl] \mbox{}

Error output file handle(s). These are the filehandles that are used
to print all error messages. Multiple filehandles can be supplied.
Returns an IO::Tee object that can be used as a single filehandle.

\begin{verbatim}
  $Prt->errhdl(\*STDERR, $fh);
\end{verbatim}
\begin{verbatim}
  $fh = $Prt->errhdl;
\end{verbatim}


Default is to use STDERR.

\item[\textbf{errbeep}] \mbox{}

Specifies whether the terminal is to beep when an error
message is printed. Default is not to beep (false).

\begin{verbatim}
  $dobeep = $Prt->errbeep;
\end{verbatim}
\item[debughdl] \mbox{}

This specifies the debug file handle. Defaults to STDERR if not 
defined. Returns an IO::Tee object that can be used as a single
filehandle.

\end{description}
\paragraph*{Methods\label{ORAC::Print_Methods}\index{ORAC::Print!Methods}}\begin{description}
\item[out(text, [col])] \mbox{}

Print output messages.
By default messages are written to STDOUT. This can be overridden with
the outhdl() method.

\item[war(text, [col])] \mbox{}

Print warning messages.
Default is to print warnings to STDOUT. This can be overriden with
the warhdl() method.

\item[err(text, [col])] \mbox{}

Print error messages.
Default is to use STDERR. This can be overriden with the errhdl()
method.

\item[debug (text)] \mbox{}

Prints debug messages to the debug filehandle so long as debugging
is turned on.

\end{description}
\subsubsection*{TIED INTERFACE\label{ORAC::Print_TIED_INTERFACE}\index{ORAC::Print!TIED INTERFACE}}

An ORAC::Print object can also be tied to filehandle using the
tie command:

\begin{verbatim}
  tie *HANDLE, 'ORAC::Print', $prt, 'out|war|err';
\end{verbatim}


where \$prt is an ORAC::Print object. Currently all strings printed
to this handle will be redirected to the orac\_print command
(and will therefore use the output filehandles associated with the
most recent ORAC::Print object created). The default color used
by the tied handle can be set using the outcol() method of the
object associated with the filehandle

\begin{verbatim}
  $prt = new ORAC::Print;
  $prt->outcol('clear');
  tie *HANDLE, 'ORAC::Print', $prt;
\end{verbatim}


will result in all messages printed to HANDLE, being printed
with no color codes to STDOUT.



The optional fourth argument to the tie() command can be used
to set the default output stream. Allowed values are 'out',
'war' and 'err'. These correspond directly to the orac\_print,
orac\_warn and orac\_err commands. Default is to use orac\_print
for all tied filehandles.



It is not possible to read from this tied filehandle.

\subsubsection*{SEE ALSO\label{ORAC::Print_SEE_ALSO}\index{ORAC::Print!SEE ALSO}}

the \emph{Term::ANSIColor} manpage, the \emph{IO::Tee} manpage.

\subsubsection*{REVISION\label{ORAC::Print_REVISION}\index{ORAC::Print!REVISION}}

\$Id$

\subsubsection*{COPYRIGHT\label{ORAC::Print_COPYRIGHT}\index{ORAC::Print!COPYRIGHT}}

Copyright (C) 1998-2000 Particle Physics and Astronomy Research
Council. All Rights Reserved.

\subsection{ORAC::TempFile\label{ORAC::TempFile}\index{ORAC::TempFile}}

Generate temporary files for ORAC-DR

\subsubsection*{SYNOPSIS\label{ORAC::TempFile_SYNOPSIS}\index{ORAC::TempFile!SYNOPSIS}}\begin{verbatim}
  use ORAC::TempFile;
\end{verbatim}
\begin{verbatim}
  $temp = new ORAC::TempFile;
  $temp = new ORAC::TempFile(0);
  $fname = $temp->file;
  print {$temp->handle} "Some temporary data";
\end{verbatim}
\begin{verbatim}
  $temp->handle->close; # Close temporary file
\end{verbatim}
\begin{verbatim}
  undef $temp;          # Close file and remove it
\end{verbatim}
\subsubsection*{DESCRIPTION\label{ORAC::TempFile_DESCRIPTION}\index{ORAC::TempFile!DESCRIPTION}}

Provide a simplified means of handling temporary files from within
ORAC-DR. The temporary file is automatically removed when the
object goes out of scope.



The temporary file name can also be used as a temporary name for
NDF files. NDF files (extension '.sdf') are automatically deleted
in addition to the temporary file created by this class.

\subsubsection*{PUBLIC METHODS\label{ORAC::TempFile_PUBLIC_METHODS}\index{ORAC::TempFile!PUBLIC METHODS}}

The following methods are available in this class.

\paragraph*{Constructor\label{ORAC::TempFile_Constructor}\index{ORAC::TempFile!Constructor}}

The following constructors are available:

\begin{description}
\item[\textbf{new}] \mbox{}

Create a new instance of a \textbf{ORAC::TempFile} object.

\begin{verbatim}
  $temp = new ORAC::TempFile;
\end{verbatim}


If a false argument is supplied the temporary file
name will be allocated (and the file opened) but the 
file itself will be closed before the new object is returned.
This is so that the temporary file name can be passed directly
to another process without wanting to write anything to the
file yourself (for example if you want to generate a file
in an external program and then read the results back into
perl).

\begin{verbatim}
  $temp = new ORAC::TempFile(0);
\end{verbatim}


Returns 'undef' if the tempfile could not be created.
The file is opened for read/write with autoflush set to true.
The file should be closed (using the close() method on the
object file handle) before sending the file name to an external
process (unless a false argument is supplied to the constructor).

\begin{verbatim}
  $temp->handle->close;
\end{verbatim}
\end{description}
\paragraph*{Accessor Methods\label{ORAC::TempFile_Accessor_Methods}\index{ORAC::TempFile!Accessor Methods}}

The following methods are available for accessing the 
'instance' data.

\begin{description}
\item[\textbf{handle}] \mbox{}

Return (or set) the file handle associated with the temporary 
file.

\begin{verbatim}
  print {$tmp->handle} "some information\n";
\end{verbatim}
\item[\textbf{file}] \mbox{}

Return the file name associated with the temporary file.

\begin{verbatim}
  $name = $tmp->file;
\end{verbatim}
\end{description}
\paragraph*{Destructor\label{ORAC::TempFile_Destructor}\index{ORAC::TempFile!Destructor}}

This section details the object destructor.

\begin{description}
\item[\textbf{DESTROY}] \mbox{}

The destructor is run when the object goes out of scope
or no longer has any references to it. When called, the
temporary file is closed and unlinked. If necessary
and files of the same name but with a '.sdf' extension
are also unlinked. This allows the same class to be used
for temporary plain files and temporary NDF files.



No files are removed if the debugging flag (\$DEBUG) is set to
true (the default is false)

\end{description}
\subsubsection*{PRIVATE METHODS\label{ORAC::TempFile_PRIVATE_METHODS}\index{ORAC::TempFile!PRIVATE METHODS}}

The following methods are intended for use inside the module.
They are included here so that authors of derived classes are 
aware of them.

\begin{description}
\item[\textbf{Initialise}] \mbox{}

This method is used to initialise the object. It is called
automatically by the object constructor. It generates
a temporary file name and attempts to open it. If the 
open is not successful the state of the object remains 
unchanged. In general, this means that the object
constructor has failed.

\end{description}
\subsubsection*{GLOBAL VARIABLES\label{ORAC::TempFile_GLOBAL_VARIABLES}\index{ORAC::TempFile!GLOBAL VARIABLES}}

The following global variables are available.
They can be accessed directly or via Class methods of the same name.

\begin{itemize}
\item \$VERSION

The current version number of this module.

\begin{verbatim}
  $version = $ORAC::TempFile::VERSION;
  $version = ORAC::TempFile->VERSION;
\end{verbatim}
\item \$DEBUG

Debugging flag. When this flag is set to true the temporary
files are not deleted by the object destructor. They can be
examined at a later time.

\begin{verbatim}
  $debug = ORAC::TempFile->DEBUG;
  ORAC::TempFile->DEBUG(1);
  $ORAC::TempFile::DEBUG = 0;
\end{verbatim}
\end{itemize}
\subsubsection*{SEE ALSO\label{ORAC::TempFile_SEE_ALSO}\index{ORAC::TempFile!SEE ALSO}}

the \emph{IO::File} manpage, the \emph{File::MkTemp} manpage, the \textsf{tmpnam()}
 entry in the \emph{POSIX} manpage

\subsubsection*{REVISION\label{ORAC::TempFile_REVISION}\index{ORAC::TempFile!REVISION}}

\$Id$

\subsubsection*{COPYRIGHT\label{ORAC::TempFile_COPYRIGHT}\index{ORAC::TempFile!COPYRIGHT}}

Copyright (C) 1998-2000 Particle Physics and Astronomy Research
Council. All Rights Reserved.

