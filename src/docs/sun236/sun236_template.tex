\documentclass[twoside,11pt]{article}

% +
%  Name:
%     sun236.tex

%  Purpose:
%     SUN documentation for ORAC-DR spectroscopy (SUN/236)

%  Authors:
%     Paul Hirst (JAC)
%     Tim Jenness (JAC)

%  Copyright:
%     Copyright (C) 2001 Particle Physics and Astronomy
%     Research Council. All Rights Reserved.

%  History:
%     $Log$
%     Revision 1.3  2001/11/28 03:18:28  timj
%     Add xref to other oracdr docs
%
%     Revision 1.2  2001/11/28 01:58:23  timj
%     - Fix spelling mistakes
%     - Add logo
%     - Remove DESCRIPTION paragraph headings
%     - Tweak some sections to be real lists
%     - Some verbatim tables were not formatted verbatim
%

%  Revision:
%     $Id$

% -


% ? Specify used packages
\usepackage{graphicx}        %  Use this one for final production.
% \usepackage[draft]{graphicx} %  Use this one for drafting.
% ? End of specify used packages

\pagestyle{myheadings}

% -----------------------------------------------------------------------------
% ? Document identification
% Fixed part
\newcommand{\stardoccategory}  {Starlink User Note}
\newcommand{\stardocinitials}  {SUN}
\newcommand{\stardocsource}    {sun\stardocnumber}
\newcommand{\stardoccopyright} 
{Copyright \copyright\ 2000 Council for the Central Laboratory of the Research Councils}

% Variable part - replace [xxx] as appropriate.
\newcommand{\stardocnumber}    {236.0}
\newcommand{\stardocauthors}   {Paul Hirst \\
                                Joint Astronomy Centre, Hilo, Hawaii}
\newcommand{\stardocdate}      {November 2001}
\newcommand{\stardoctitle}     {ORAC-DR -- spectroscopy data reduction}
\newcommand{\stardocversion}   {}
\newcommand{\stardocmanual}    {User Guide}
\newcommand{\stardocabstract}  {ORAC-DR is a
general-purpose automatic data-reduction pipeline environment.  This
document describes its use to reduce spectroscopy data collected at the
United Kingdom Infrared Telescope (UKIRT) with the CGS4 and MICHELLE
instruments. }
% ? End of document identification
% -----------------------------------------------------------------------------

% +
%  Name:
%     sun.tex
%
%  Purpose:
%     Template for Starlink User Note (SUN) documents.
%     Refer to SUN/199
%
%  Authors:
%     AJC: A.J.Chipperfield (Starlink, RAL)
%     BLY: M.J.Bly (Starlink, RAL)
%     PWD: Peter W. Draper (Starlink, Durham University)
%
%  History:
%     17-JAN-1996 (AJC):
%        Original with hypertext macros, based on MDL plain originals.
%     16-JUN-1997 (BLY):
%        Adapted for LaTeX2e.
%        Added picture commands.
%     13-AUG-1998 (PWD):
%        Converted for use with LaTeX2HTML version 98.2 and
%        Star2HTML version 1.3.
%      1-FEB-2000 (AJC):
%        Add Copyright statement in LaTeX
%     {Add further history here}
%
% -

\newcommand{\stardocname}{\stardocinitials /\stardocnumber}
\markboth{\stardocname}{\stardocname}
\setlength{\textwidth}{160mm}
\setlength{\textheight}{230mm}
\setlength{\topmargin}{-2mm}
\setlength{\oddsidemargin}{0mm}
\setlength{\evensidemargin}{0mm}
\setlength{\parindent}{0mm}
\setlength{\parskip}{\medskipamount}
\setlength{\unitlength}{1mm}

% -----------------------------------------------------------------------------
%  Hypertext definitions.
%  ======================
%  These are used by the LaTeX2HTML translator in conjunction with star2html.

%  Comment.sty: version 2.0, 19 June 1992
%  Selectively in/exclude pieces of text.
%
%  Author
%    Victor Eijkhout                                      <eijkhout@cs.utk.edu>
%    Department of Computer Science
%    University Tennessee at Knoxville
%    104 Ayres Hall
%    Knoxville, TN 37996
%    USA

%  Do not remove the %begin{latexonly} and %end{latexonly} lines (used by 
%  LaTeX2HTML to signify text it shouldn't process).
%begin{latexonly}
\makeatletter
\def\makeinnocent#1{\catcode`#1=12 }
\def\csarg#1#2{\expandafter#1\csname#2\endcsname}

\def\ThrowAwayComment#1{\begingroup
    \def\CurrentComment{#1}%
    \let\do\makeinnocent \dospecials
    \makeinnocent\^^L% and whatever other special cases
    \endlinechar`\^^M \catcode`\^^M=12 \xComment}
{\catcode`\^^M=12 \endlinechar=-1 %
 \gdef\xComment#1^^M{\def\test{#1}
      \csarg\ifx{PlainEnd\CurrentComment Test}\test
          \let\html@next\endgroup
      \else \csarg\ifx{LaLaEnd\CurrentComment Test}\test
            \edef\html@next{\endgroup\noexpand\end{\CurrentComment}}
      \else \let\html@next\xComment
      \fi \fi \html@next}
}
\makeatother

\def\includecomment
 #1{\expandafter\def\csname#1\endcsname{}%
    \expandafter\def\csname end#1\endcsname{}}
\def\excludecomment
 #1{\expandafter\def\csname#1\endcsname{\ThrowAwayComment{#1}}%
    {\escapechar=-1\relax
     \csarg\xdef{PlainEnd#1Test}{\string\\end#1}%
     \csarg\xdef{LaLaEnd#1Test}{\string\\end\string\{#1\string\}}%
    }}

%  Define environments that ignore their contents.
\excludecomment{comment}
\excludecomment{rawhtml}
\excludecomment{htmlonly}

%  Hypertext commands etc. This is a condensed version of the html.sty
%  file supplied with LaTeX2HTML by: Nikos Drakos <nikos@cbl.leeds.ac.uk> &
%  Jelle van Zeijl <jvzeijl@isou17.estec.esa.nl>. The LaTeX2HTML documentation
%  should be consulted about all commands (and the environments defined above)
%  except \xref and \xlabel which are Starlink specific.

\newcommand{\htmladdnormallinkfoot}[2]{#1\footnote{#2}}
\newcommand{\htmladdnormallink}[2]{#1}
\newcommand{\htmladdimg}[1]{}
\newcommand{\hyperref}[4]{#2\ref{#4}#3}
\newcommand{\htmlref}[2]{#1}
\newcommand{\htmlimage}[1]{}
\newcommand{\htmladdtonavigation}[1]{}

\newenvironment{latexonly}{}{}
\newcommand{\latex}[1]{#1}
\newcommand{\html}[1]{}
\newcommand{\latexhtml}[2]{#1}
\newcommand{\HTMLcode}[2][]{}

%  Starlink cross-references and labels.
\newcommand{\xref}[3]{#1}
\newcommand{\xlabel}[1]{}

%  LaTeX2HTML symbol.
\newcommand{\latextohtml}{\LaTeX2\texttt{HTML}}

%  Define command to re-centre underscore for Latex and leave as normal
%  for HTML (severe problems with \_ in tabbing environments and \_\_
%  generally otherwise).
\renewcommand{\_}{\texttt{\symbol{95}}}

% -----------------------------------------------------------------------------
%  Debugging.
%  =========
%  Remove % on the following to debug links in the HTML version using Latex.

% \newcommand{\hotlink}[2]{\fbox{\begin{tabular}[t]{@{}c@{}}#1\\\hline{\footnotesize #2}\end{tabular}}}
% \renewcommand{\htmladdnormallinkfoot}[2]{\hotlink{#1}{#2}}
% \renewcommand{\htmladdnormallink}[2]{\hotlink{#1}{#2}}
% \renewcommand{\hyperref}[4]{\hotlink{#1}{\S\ref{#4}}}
% \renewcommand{\htmlref}[2]{\hotlink{#1}{\S\ref{#2}}}
% \renewcommand{\xref}[3]{\hotlink{#1}{#2 -- #3}}
%end{latexonly}
% -----------------------------------------------------------------------------
% ? Document specific \newcommand or \newenvironment commands.
% ? End of document specific commands
% -----------------------------------------------------------------------------
%  Title Page.
%  ===========
\renewcommand{\thepage}{\roman{page}}
\begin{document}
\setcounter{secnumdepth}{5}
\thispagestyle{empty}

%  Latex document header.
%  ======================
\begin{latexonly}
   CCLRC / \textsc{Rutherford Appleton Laboratory} \hfill \textbf{\stardocname}\\
   {\large Particle Physics \& Astronomy Research Council}\\
   {\large Starlink Project\\}
   {\large \stardoccategory\ \stardocnumber}
   \begin{flushright}
   \stardocauthors\\
   \stardocdate
   \end{flushright}
   \vspace{-4mm}
   \rule{\textwidth}{0.5mm}
   \vspace{5mm}
   \begin{center}
   {\Huge\textbf{\stardoctitle \\ [2.5ex]}}
   {\LARGE\textbf{\stardocversion \\ [4ex]}}
   {\Huge\textbf{\stardocmanual}}
   \end{center}
   \vspace{5mm}

% ? Add picture here if required for the LaTeX version.
%   e.g. \includegraphics[scale=0.3]{filename.ps}
% ? End of picture

% ? Heading for abstract if used.
   \vspace{10mm}
   \begin{center}
      {\Large\textbf{Abstract}}
   \end{center}
% ? End of heading for abstract.
\end{latexonly}

%  HTML documentation header.
%  ==========================
\begin{htmlonly}
   \xlabel{}
   \begin{rawhtml} <H1> \end{rawhtml}
      \stardoctitle\\
      \stardocversion\\
      \stardocmanual
   \begin{rawhtml} </H1> <HR> \end{rawhtml}

% ? Add picture here if required for the hypertext version.
%   e.g. \includegraphics[scale=0.7]{filename.ps}
% ? End of picture

   \begin{rawhtml} <P> <I> \end{rawhtml}
   \stardoccategory\ \stardocnumber \\
   \stardocauthors \\
   \stardocdate
   \begin{rawhtml} </I> </P> <H3> \end{rawhtml}
      \htmladdnormallink{CCLRC / Rutherford Appleton Laboratory}
                        {http://www.cclrc.ac.uk} \\
      \htmladdnormallink{Particle Physics \& Astronomy Research Council}
                        {http://www.pparc.ac.uk} \\
   \begin{rawhtml} </H3> <H2> \end{rawhtml}
      \htmladdnormallink{Starlink Project}{http://www.starlink.rl.ac.uk/}
   \begin{rawhtml} </H2> \end{rawhtml}
   \htmladdnormallink{\htmladdimg{source.gif} Retrieve hardcopy}
      {http://www.starlink.rl.ac.uk/cgi-bin/hcserver?\stardocsource}\\

%  HTML document table of contents. 
%  ================================
%  Add table of contents header and a navigation button to return to this 
%  point in the document (this should always go before the abstract \section). 
  \label{stardoccontents}
  \begin{rawhtml} 
    <HR>
    <H2>Contents</H2>
  \end{rawhtml}
  \htmladdtonavigation{\htmlref{\htmladdimg{contents_motif.gif}}
        {stardoccontents}}

% ? New section for abstract if used.
  \section{\xlabel{abstract}Abstract}
% ? End of new section for abstract
\end{htmlonly}

% -----------------------------------------------------------------------------
% ? Document Abstract. (if used)
%  ==================
\stardocabstract
% ? End of document abstract

% -----------------------------------------------------------------------------
% ? Latex Copyright Statement
%  =========================
\begin{latexonly}
\newpage
\vspace*{\fill}
\stardoccopyright
\end{latexonly}
% ? End of Latex copyright statement

% -----------------------------------------------------------------------------
% ? Latex document Table of Contents (if used).
%  ===========================================
  \newpage
  \begin{latexonly}
    \setlength{\parskip}{0mm}
    \tableofcontents
    \setlength{\parskip}{\medskipamount}
    \markboth{\stardocname}{\stardocname}
  \end{latexonly}
% ? End of Latex document table of contents
% -----------------------------------------------------------------------------

\cleardoublepage
\renewcommand{\thepage}{\arabic{page}}
\setcounter{page}{1}

% ? Main text

\section{Introduction}

This document describes the spectroscopy recipes and primitives included with
ORAC-DR. See \xref{SUN/230}{sun230}{} for general ORAC-DR documentation and
\xref{SUN/232}{sun232}{} for information on ORAC-DR imaging data reduction.

At this time (November 2001), ORAC-DR's spectroscopy suite is designed
to reduce data from the UKIRT spectrometer suite - CGS4 and
MICHELLE. Support for UIST is imminent. The recipes and primitives
are written in such a way as to make the addition of new instruments
simple.

\section{Recipe and Primitive Architecture}

In general, an ORAC-DR recipe is simply a list of primitives, which
are executed in turn on the data from each observation that is passed
through the pipeline. The primitives form building blocks, each
carrying out an atomic data reduction operation. Thus, the recipe is
readable, understandable and to some extent modifiable by someone not
familiar with the primitive internals - for example an observer or
scientist reducing data and wishing to add or remove steps from the
data reduction process. The primitives themselves are implemented in
perl5 and are not designed to be usefully modifiable by a
non-programmer.

In actual fact, it has been found to be beneficial in many places to
take this abstraction one level deeper - in the spectroscopy pipeline
many of the steps which we wish to denote in the recipe as a single
primitive are themselves too complex to be considered an atomic data
reduction operation, perhaps a good analogy is that be considered a
molecular operation, themselves implemented as a simple list of
primitives which carry out the true atomic operations. In some cases
there is simple logic - generally simple flow control switches - for
example ``if'' clauses at the molecular level, though this code is
simple and these molecular level primitives can be usefully modified
to add or remove steps by a typical astronomer.

\section{The Recipes}

Each recipe has an in-code documentation section, which documents the
recipe concerned. Here, we simply divide the recipes into groups
corresponding to the type of observations they reduce, and give the
recipe documentation.

Some of the recipes have dedicated primitives, this is simply to
avoid having the perl code in the recipe. In these cases, we include
the primitive documentation with that of the recipe. For the main
science recipes, we will describe the recipes at molecular level
here, with the details of the primitives in a subsequent section.

\subsection{System Verification Recipes and Primitives}

\subsubsection{ARRAY\_TESTS}

Strictly speaking, the ARRAY\_TESTS recipe is not part of the ORAC-DR
spectroscopy suite - each instrument provides its own array tests
recipe if it requires one. However, if one is provided, it can
generate bad pixel masks and read noise measurements that will be used
by the rest of the spectroscopy pipeline. In fact, a read noise number
is mandatory for the spectroscopy pipeline, and thus if not determined
by an array tests procedure, has to be supplied manually, usually by
the instrument scientist. The CGS4 specific array tests recipe is
described here, as it is currently a good example.

%% ORACDRDOC_RECIPE:ARRAY_TESTS

%% ORACDRDOC_PRIMITIVE:_ARRAY_TESTS_

%% ORACDRDOC_PRIMITIVE:_ARRAY_TESTS_STATISTICS_

%% ORACDRDOC_PRIMITIVE:_FIND_BAD_PIXELS_

\subsubsection{EMISSIVITY}

It is not strictly necessary for each instrument to be able to take
emissivity data, though currently both CGS4 and MICHELLE do, and it is
strongly anticipated that UIST also will.

%% ORACDRDOC_RECIPE:EMISSIVITY

%% ORACDRDOC_PRIMITIVE:_EMISSIVITY_

\subsection{Calibration Frame Recipes}

\subsubsection{Flat Fields}

The REDUCE\_FLAT recipe will reduce a FLAT field observation. FLAT
observations are usually taken by observing the black-body source in
the CGS4 calibration unit, though the Tungsten-Halogen lamp is used as
the shorter CGS4 wavelengths. Standard operating procedures for
MICHELLE flats have yet to be ascertained, though REDUCE\_FLAT will
work fine if the (warm) shutter of the instrument is used as a black
body source. Sky flats will probably be reduced by an (as yet
unwritten) separate recipe.

%% ORACDRDOC_RECIPE:REDUCE_FLAT

%% ORACDRDOC_PRIMITIVE:_NORMALISE_FLAT_BY_BB_

\subsubsection{Arc Lamp observations}

The REDUCE\_ARC recipe reduces a CGS4 ARC lamp observation. Currently
the arc frame is not actually used for wavelength calibration, though
this will probably change shortly.

%% ORACDRDOC_RECIPE:REDUCE_ARC

The REDUCE\_DARK recipe will reduce and file a DARK observation. In
practice this is very rarely used as the dark current of the array is
intrinsically subtracted when subtracting a sky or offset beam frame.

\subsubsection{Dark Frames}


%% ORACDRDOC_RECIPE:REDUCE_DARK

\subsubsection{Bias Frames}

The REDUCE\_BIAS recipe reduces a bias observation. The default BIAS
observation takes 3 integrations, each containing many exposures. The
REDUCE\_BIAS recipe forms a variance array based on the variance of the
3 integrations in the observation.


%% ORACDRDOC_RECIPE:REDUCE_BIAS

%% ORACDRDOC_PRIMITIVE:_REDUCE_BIAS_


\subsection{Calibration Star Recipes}

\subsubsection{STANDARD\_STAR}
%% ORACDRDOC_RECIPE:STANDARD_STAR
%% ORACDRDOC_RECIPE:STANDARD_STAR_NOFLAT
%% ORACDRDOC_PRIMITIVE:_STANDARD_STAR_

\subsection{Science target Recipes}

\subsubsection{POINT\_SOURCE}

The POINT\_SOURCE recipe reduces observations of point sources, with
nodding along the slit or off to sky (the reason you'd go off to sky
with a point source being that it's in a crowded field).


Variants: \\
\_NOFLAT - does not do flat fielding \\
\_NOSTD - does not ratio by standard star \\
\_MAYBESTD - ratios by standard if there is one, doesn't complain if not. \\
\_NOFLAT\_NOSTD - does not flat field or ratio by standard star \\
\_BASIC - reduced processing for faster execution (maybe - will see if needed) \\

%% ORACDRDOC_RECIPE:POINT_SOURCE
%% ORACDRDOC_RECIPE:POINT_SOURCE_NOFLAT
%% ORACDRDOC_RECIPE:POINT_SOURCE_NOSTD
%% ORACDRDOC_RECIPE:POINT_SOURCE_NOFLAT_NOSTD

\subsubsection{EXTENDED\_SOURCE}

The EXTENDED\_SOURCE recipe reduces observations of extended sources.

Variants: \\
\_NOFLAT - does not do flat fielding \\
\_NOSTD - does not ratio by standard star \\
\_MAYBESTD - ratios by standard if there is one, doesn't complain if not. \\
\_NOFLAT\_NOSTD - does not flat field or ratio by standard star \\
\_BASIC - reduced processing for faster execution (maybe - will see if needed) \\

%% ORACDRDOC_RECIPE:EXTENDED_SOURCE
%% ORACDRDOC_RECIPE:EXTENDED_SOURCE_NOFLAT
%% ORACDRDOC_RECIPE:EXTENDED_SOURCE_NOSTD
%% ORACDRDOC_RECIPE:EXTENDED_SOURCE_NOFLAT_NOSTD

\subsubsection{Extended source with a stable sky background}

The EXTENDED\_SOURCE\_SEPARATE\_SKY recipe is for use when the sky is
sufficiently stable, that you do not need to spend half your time
observing it. This is only expected to be of use with CGS4 echelle
observations. For point sources, you would nod along the slit anyway.

Variants: \\
\_NOFLAT - does not do flat fielding \\
\_NOSTD - does not ratio by standard star \\
\_NOFLAT\_NOSTD - does not flat field or ratio by standard star \\
\_BASIC - reduced processing for faster execution (maybe - will see if needed) \\

%% ORACDRDOC_RECIPE:EXTENDED_SOURCE_WITH_SEPARATE_SKY
%% ORACDRDOC_RECIPE:EXTENDED_SOURCE_WITH_SEPARATE_SKY_NOFLAT
%% ORACDRDOC_RECIPE:EXTENDED_SOURCE_WITH_SEPARATE_SKY_NOSTD
%% ORACDRDOC_RECIPE:EXTENDED_SOURCE_WITH_SEPARATE_SKY_NOFLAT_NOSTD

\subsubsection{Blank Sky Observations}

%% ORACDRDOC_RECIPE:REDUCE_SKY

\subsection{Utility Recipes}

\subsubsection{Night Logs}

The NIGHT\_LOG recipe is the default recipe for generating summary
logs of a list of observations. The NIGHT\_LOG\_LONG variant adds more
details to the log file.

%% ORACDRDOC_RECIPE:NIGHT_LOG
%% ORACDRDOC_RECIPE:NIGHT_LOG_LONG
%% ORACDRDOC_PRIMITIVE:_NIGHT_LOG_

\subsection{Molecular Primitives}

\subsubsection{To Reduce a Single Observation to a \_wce frame}

%% ORACDRDOC_PRIMITIVE:_REDUCE_SINGLE_FRAME_

\subsubsection{Pairwise Grouping}

%% ORACDRDOC_PRIMITIVE:_PAIRWISE_GROUP_

\subsubsection{Extracting and Coadding Spectra}

%% ORACDRDOC_PRIMITIVE:_EXTRACT_SPECTRA_
%% ORACDRDOC_PRIMITIVE:_EXTRACT_DETERMINE_NBEAMS_
%% ORACDRDOC_PRIMITIVE:_EXTRACT_FIND_ROWS_
%% ORACDRDOC_PRIMITIVE:_EXTRACT_ALL_BEAMS_
%% ORACDRDOC_PRIMITIVE:_DERIPPLE_ALL_BEAMS_
%% ORACDRDOC_PRIMITIVE:_CROSS_CORR_ALL_BEAMS_
%% ORACDRDOC_PRIMITIVE:_COADD_EXTRACTED_BEAMS_

\subsubsection{Using Standard Stars}

%% ORACDRDOC_PRIMITIVE:_DIVIDE_BY_STANDARD_
%% ORACDRDOC_PRIMITIVE:_ALIGN_SPECTRUM_TO_STD_
%% ORACDRDOC_PRIMITIVE:_FLUX_CALIBRATE_

\subsection{Historical Recipe Names}

\begin{itemize}
\item SOURCE\_PAIRS\_ON\_SLIT
\item SOURCE\_PAIRS\_TO\_SKY
\item SOURCE\_WITH\_NOD\_TO\_BLANK\_SKY
\end{itemize}

See the detailed documentation for each recipe and primitive later.


% ? End of main text
\end{document}
