\documentclass[twoside,11pt]{article}

% +
%  Name:
%     sun236.tex

%  Purpose:
%     SUN documentation for ORAC-DR spectroscopy (SUN/236)

%  Authors:
%     Paul Hirst (JAC)
%     Tim Jenness (JAC)

%  Copyright:
%     Copyright (C) 2001 Particle Physics and Astronomy
%     Research Council. All Rights Reserved.

%  History:
%     $Log$
%     Revision 1.4  2001/12/13 01:32:26  phirst
%     Added FAINT_POINT_SOURCE
%
%     Revision 1.3  2001/11/28 03:18:28  timj
%     Add xref to other oracdr docs
%
%     Revision 1.2  2001/11/28 01:58:23  timj
%     - Fix spelling mistakes
%     - Add logo
%     - Remove DESCRIPTION paragraph headings
%     - Tweak some sections to be real lists
%     - Some verbatim tables were not formatted verbatim
%

%  Revision:
%     $Id$

% -


% ? Specify used packages
\usepackage{graphicx}        %  Use this one for final production.
% \usepackage[draft]{graphicx} %  Use this one for drafting.
% ? End of specify used packages

\pagestyle{myheadings}

% -----------------------------------------------------------------------------
% ? Document identification
% Fixed part
\newcommand{\stardoccategory}  {Starlink User Note}
\newcommand{\stardocinitials}  {SUN}
\newcommand{\stardocsource}    {sun\stardocnumber}
\newcommand{\stardoccopyright} 
{Copyright \copyright\ 2000 Council for the Central Laboratory of the Research Councils}

% Variable part - replace [xxx] as appropriate.
\newcommand{\stardocnumber}    {236.0}
\newcommand{\stardocauthors}   {Paul Hirst \\
                                Joint Astronomy Centre, Hilo, Hawaii}
\newcommand{\stardocdate}      {November 2001}
\newcommand{\stardoctitle}     {ORAC-DR -- spectroscopy data reduction}
\newcommand{\stardocversion}   {}
\newcommand{\stardocmanual}    {User Guide}
\newcommand{\stardocabstract}  {ORAC-DR is a
general-purpose automatic data-reduction pipeline environment.  This
document describes its use to reduce spectroscopy data collected at the
United Kingdom Infrared Telescope (UKIRT) with the CGS4 and MICHELLE
instruments. }
% ? End of document identification
% -----------------------------------------------------------------------------

% +
%  Name:
%     sun.tex
%
%  Purpose:
%     Template for Starlink User Note (SUN) documents.
%     Refer to SUN/199
%
%  Authors:
%     AJC: A.J.Chipperfield (Starlink, RAL)
%     BLY: M.J.Bly (Starlink, RAL)
%     PWD: Peter W. Draper (Starlink, Durham University)
%
%  History:
%     17-JAN-1996 (AJC):
%        Original with hypertext macros, based on MDL plain originals.
%     16-JUN-1997 (BLY):
%        Adapted for LaTeX2e.
%        Added picture commands.
%     13-AUG-1998 (PWD):
%        Converted for use with LaTeX2HTML version 98.2 and
%        Star2HTML version 1.3.
%      1-FEB-2000 (AJC):
%        Add Copyright statement in LaTeX
%     {Add further history here}
%
% -

\newcommand{\stardocname}{\stardocinitials /\stardocnumber}
\markboth{\stardocname}{\stardocname}
\setlength{\textwidth}{160mm}
\setlength{\textheight}{230mm}
\setlength{\topmargin}{-2mm}
\setlength{\oddsidemargin}{0mm}
\setlength{\evensidemargin}{0mm}
\setlength{\parindent}{0mm}
\setlength{\parskip}{\medskipamount}
\setlength{\unitlength}{1mm}

% -----------------------------------------------------------------------------
%  Hypertext definitions.
%  ======================
%  These are used by the LaTeX2HTML translator in conjunction with star2html.

%  Comment.sty: version 2.0, 19 June 1992
%  Selectively in/exclude pieces of text.
%
%  Author
%    Victor Eijkhout                                      <eijkhout@cs.utk.edu>
%    Department of Computer Science
%    University Tennessee at Knoxville
%    104 Ayres Hall
%    Knoxville, TN 37996
%    USA

%  Do not remove the %begin{latexonly} and %end{latexonly} lines (used by 
%  LaTeX2HTML to signify text it shouldn't process).
%begin{latexonly}
\makeatletter
\def\makeinnocent#1{\catcode`#1=12 }
\def\csarg#1#2{\expandafter#1\csname#2\endcsname}

\def\ThrowAwayComment#1{\begingroup
    \def\CurrentComment{#1}%
    \let\do\makeinnocent \dospecials
    \makeinnocent\^^L% and whatever other special cases
    \endlinechar`\^^M \catcode`\^^M=12 \xComment}
{\catcode`\^^M=12 \endlinechar=-1 %
 \gdef\xComment#1^^M{\def\test{#1}
      \csarg\ifx{PlainEnd\CurrentComment Test}\test
          \let\html@next\endgroup
      \else \csarg\ifx{LaLaEnd\CurrentComment Test}\test
            \edef\html@next{\endgroup\noexpand\end{\CurrentComment}}
      \else \let\html@next\xComment
      \fi \fi \html@next}
}
\makeatother

\def\includecomment
 #1{\expandafter\def\csname#1\endcsname{}%
    \expandafter\def\csname end#1\endcsname{}}
\def\excludecomment
 #1{\expandafter\def\csname#1\endcsname{\ThrowAwayComment{#1}}%
    {\escapechar=-1\relax
     \csarg\xdef{PlainEnd#1Test}{\string\\end#1}%
     \csarg\xdef{LaLaEnd#1Test}{\string\\end\string\{#1\string\}}%
    }}

%  Define environments that ignore their contents.
\excludecomment{comment}
\excludecomment{rawhtml}
\excludecomment{htmlonly}

%  Hypertext commands etc. This is a condensed version of the html.sty
%  file supplied with LaTeX2HTML by: Nikos Drakos <nikos@cbl.leeds.ac.uk> &
%  Jelle van Zeijl <jvzeijl@isou17.estec.esa.nl>. The LaTeX2HTML documentation
%  should be consulted about all commands (and the environments defined above)
%  except \xref and \xlabel which are Starlink specific.

\newcommand{\htmladdnormallinkfoot}[2]{#1\footnote{#2}}
\newcommand{\htmladdnormallink}[2]{#1}
\newcommand{\htmladdimg}[1]{}
\newcommand{\hyperref}[4]{#2\ref{#4}#3}
\newcommand{\htmlref}[2]{#1}
\newcommand{\htmlimage}[1]{}
\newcommand{\htmladdtonavigation}[1]{}

\newenvironment{latexonly}{}{}
\newcommand{\latex}[1]{#1}
\newcommand{\html}[1]{}
\newcommand{\latexhtml}[2]{#1}
\newcommand{\HTMLcode}[2][]{}

%  Starlink cross-references and labels.
\newcommand{\xref}[3]{#1}
\newcommand{\xlabel}[1]{}

%  LaTeX2HTML symbol.
\newcommand{\latextohtml}{\LaTeX2\texttt{HTML}}

%  Define command to re-centre underscore for Latex and leave as normal
%  for HTML (severe problems with \_ in tabbing environments and \_\_
%  generally otherwise).
\renewcommand{\_}{\texttt{\symbol{95}}}

% -----------------------------------------------------------------------------
%  Debugging.
%  =========
%  Remove % on the following to debug links in the HTML version using Latex.

% \newcommand{\hotlink}[2]{\fbox{\begin{tabular}[t]{@{}c@{}}#1\\\hline{\footnotesize #2}\end{tabular}}}
% \renewcommand{\htmladdnormallinkfoot}[2]{\hotlink{#1}{#2}}
% \renewcommand{\htmladdnormallink}[2]{\hotlink{#1}{#2}}
% \renewcommand{\hyperref}[4]{\hotlink{#1}{\S\ref{#4}}}
% \renewcommand{\htmlref}[2]{\hotlink{#1}{\S\ref{#2}}}
% \renewcommand{\xref}[3]{\hotlink{#1}{#2 -- #3}}
%end{latexonly}
% -----------------------------------------------------------------------------
% ? Document specific \newcommand or \newenvironment commands.
% ? End of document specific commands
% -----------------------------------------------------------------------------
%  Title Page.
%  ===========
\renewcommand{\thepage}{\roman{page}}
\begin{document}
\setcounter{secnumdepth}{5}
\thispagestyle{empty}

%  Latex document header.
%  ======================
\begin{latexonly}
   CCLRC / \textsc{Rutherford Appleton Laboratory} \hfill \textbf{\stardocname}\\
   {\large Particle Physics \& Astronomy Research Council}\\
   {\large Starlink Project\\}
   {\large \stardoccategory\ \stardocnumber}
   \begin{flushright}
   \stardocauthors\\
   \stardocdate
   \end{flushright}
   \vspace{-4mm}
   \rule{\textwidth}{0.5mm}
   \vspace{5mm}
   \begin{center}
   {\Huge\textbf{\stardoctitle \\ [2.5ex]}}
   {\LARGE\textbf{\stardocversion \\ [4ex]}}
   {\Huge\textbf{\stardocmanual}}
   \end{center}
   \vspace{5mm}

% ? Add picture here if required for the LaTeX version.
%   e.g. \includegraphics[scale=0.3]{filename.ps}
% ? End of picture

% ? Heading for abstract if used.
   \vspace{10mm}
   \begin{center}
      {\Large\textbf{Abstract}}
   \end{center}
% ? End of heading for abstract.
\end{latexonly}

%  HTML documentation header.
%  ==========================
\begin{htmlonly}
   \xlabel{}
   \begin{rawhtml} <H1> \end{rawhtml}
      \stardoctitle\\
      \stardocversion\\
      \stardocmanual
   \begin{rawhtml} </H1> <HR> \end{rawhtml}

% ? Add picture here if required for the hypertext version.
%   e.g. \includegraphics[scale=0.7]{filename.ps}
% ? End of picture

   \begin{rawhtml} <P> <I> \end{rawhtml}
   \stardoccategory\ \stardocnumber \\
   \stardocauthors \\
   \stardocdate
   \begin{rawhtml} </I> </P> <H3> \end{rawhtml}
      \htmladdnormallink{CCLRC / Rutherford Appleton Laboratory}
                        {http://www.cclrc.ac.uk} \\
      \htmladdnormallink{Particle Physics \& Astronomy Research Council}
                        {http://www.pparc.ac.uk} \\
   \begin{rawhtml} </H3> <H2> \end{rawhtml}
      \htmladdnormallink{Starlink Project}{http://www.starlink.rl.ac.uk/}
   \begin{rawhtml} </H2> \end{rawhtml}
   \htmladdnormallink{\htmladdimg{source.gif} Retrieve hardcopy}
      {http://www.starlink.rl.ac.uk/cgi-bin/hcserver?\stardocsource}\\

%  HTML document table of contents. 
%  ================================
%  Add table of contents header and a navigation button to return to this 
%  point in the document (this should always go before the abstract \section). 
  \label{stardoccontents}
  \begin{rawhtml} 
    <HR>
    <H2>Contents</H2>
  \end{rawhtml}
  \htmladdtonavigation{\htmlref{\htmladdimg{contents_motif.gif}}
        {stardoccontents}}

% ? New section for abstract if used.
  \section{\xlabel{abstract}Abstract}
% ? End of new section for abstract
\end{htmlonly}

% -----------------------------------------------------------------------------
% ? Document Abstract. (if used)
%  ==================
\stardocabstract
% ? End of document abstract

% -----------------------------------------------------------------------------
% ? Latex Copyright Statement
%  =========================
\begin{latexonly}
\newpage
\vspace*{\fill}
\stardoccopyright
\end{latexonly}
% ? End of Latex copyright statement

% -----------------------------------------------------------------------------
% ? Latex document Table of Contents (if used).
%  ===========================================
  \newpage
  \begin{latexonly}
    \setlength{\parskip}{0mm}
    \tableofcontents
    \setlength{\parskip}{\medskipamount}
    \markboth{\stardocname}{\stardocname}
  \end{latexonly}
% ? End of Latex document table of contents
% -----------------------------------------------------------------------------

\cleardoublepage
\renewcommand{\thepage}{\arabic{page}}
\setcounter{page}{1}

% ? Main text

\section{Introduction}

This document describes the spectroscopy recipes and primitives included with
ORAC-DR. See \xref{SUN/230}{sun230}{} for general ORAC-DR documentation and
\xref{SUN/232}{sun232}{} for information on ORAC-DR imaging data reduction.

At this time (November 2001), ORAC-DR's spectroscopy suite is designed
to reduce data from the UKIRT spectrometer suite - CGS4 and
MICHELLE. Support for UIST is imminent. The recipes and primitives
are written in such a way as to make the addition of new instruments
simple.

\section{Recipe and Primitive Architecture}

In general, an ORAC-DR recipe is simply a list of primitives, which
are executed in turn on the data from each observation that is passed
through the pipeline. The primitives form building blocks, each
carrying out an atomic data reduction operation. Thus, the recipe is
readable, understandable and to some extent modifiable by someone not
familiar with the primitive internals - for example an observer or
scientist reducing data and wishing to add or remove steps from the
data reduction process. The primitives themselves are implemented in
perl5 and are not designed to be usefully modifiable by a
non-programmer.

In actual fact, it has been found to be beneficial in many places to
take this abstraction one level deeper - in the spectroscopy pipeline
many of the steps which we wish to denote in the recipe as a single
primitive are themselves too complex to be considered an atomic data
reduction operation, perhaps a good analogy is that be considered a
molecular operation, themselves implemented as a simple list of
primitives which carry out the true atomic operations. In some cases
there is simple logic - generally simple flow control switches - for
example ``if'' clauses at the molecular level, though this code is
simple and these molecular level primitives can be usefully modified
to add or remove steps by a typical astronomer.

\section{The Recipes}

Each recipe has an in-code documentation section, which documents the
recipe concerned. Here, we simply divide the recipes into groups
corresponding to the type of observations they reduce, and give the
recipe documentation.

Some of the recipes have dedicated primitives, this is simply to
avoid having the perl code in the recipe. In these cases, we include
the primitive documentation with that of the recipe. For the main
science recipes, we will describe the recipes at molecular level
here, with the details of the primitives in a subsequent section.

\subsection{System Verification Recipes and Primitives}

\subsubsection{ARRAY\_TESTS}

Strictly speaking, the ARRAY\_TESTS recipe is not part of the ORAC-DR
spectroscopy suite - each instrument provides its own array tests
recipe if it requires one. However, if one is provided, it can
generate bad pixel masks and read noise measurements that will be used
by the rest of the spectroscopy pipeline. In fact, a read noise number
is mandatory for the spectroscopy pipeline, and thus if not determined
by an array tests procedure, has to be supplied manually, usually by
the instrument scientist. The CGS4 specific array tests recipe is
described here, as it is currently a good example.

%% ORACDRDOC_RECIPE:ARRAY_TESTS
\paragraph*{ARRAY\_TESTS\label{ARRAY_TESTS}\index{ARRAY\ TESTS}}

Tests the array, determining the readnoise and dark current

\subparagraph*{DESCRIPTION\label{ARRAY_TESTS_DESCRIPTION}\index{ARRAY TESTS!DESCRIPTION}}

Performs array tests for an array. This recipe uses the raw data,
reduces it, and calculates read noise, median and modal dark currents
for the array. It assumes that data is taken in the standard CGS4
array tests configuration



See the \_ARRAY\_TESTS\_ CGS4 primitive for details.

\subparagraph*{AUTHOR\label{ARRAY_TESTS_AUTHOR}\index{ARRAY TESTS!AUTHOR}}

Paul Hirst <p.hirst@jach.hawaii.edu>

\subparagraph*{COPYRIGHT\label{ARRAY_TESTS_COPYRIGHT}\index{ARRAY TESTS!COPYRIGHT}}

Copyright (C) 1998-2001 Particle Physics and Astronomy Research
Council. All Rights Reserved.


%% ORACDRDOC_PRIMITIVE:_ARRAY_TESTS_
\paragraph*{\_ARRAY\_TESTS\_\label{_ARRAY_TESTS_}\index{\ ARRAY\ TESTS\ }}

CGS4 array tests

\subparagraph*{DESCRIPTION\label{_ARRAY_TESTS__DESCRIPTION}\index{ ARRAY TESTS !DESCRIPTION}}

Calculates the readnoise and dark current parameters for the array.
Also forms bad pixel masks.



This primitive ensures that the BIAS and DARK frames taken as part of
the array tests are also filed with the calibration system so that
they can be used to reduce subsequent data.



The \_FIND\_BAD\_PIXELS\_ primitive is called to detect bad pixels at each
stage of the array tests.



The \_ARRAY\_TESTS\_STATISTICS\_ primitive is called to determine the
readnoise and dark currents.

\subparagraph*{AUTHOR\label{_ARRAY_TESTS__AUTHOR}\index{ ARRAY TESTS !AUTHOR}}

Paul Hirst <p.hirst@jach.hawaii.edu>

\subparagraph*{COPYRIGHT\label{_ARRAY_TESTS__COPYRIGHT}\index{ ARRAY TESTS !COPYRIGHT}}

Copyright 1998-2001 Particle Physics and Astronomy Research Council.
All Rights Reserved.


%% ORACDRDOC_PRIMITIVE:_ARRAY_TESTS_STATISTICS_
\paragraph*{\_ARRAY\_TESTS\_STATISTICS\_\label{_ARRAY_TESTS_STATISTICS_}\index{\ ARRAY\ TESTS\ STATISTICS\ }}

Determines array tests statistics

\subparagraph*{DESCRIPTION\label{_ARRAY_TESTS_STATISTICS__DESCRIPTION}\index{ ARRAY TESTS STATISTICS !DESCRIPTION}}

Does the array tests statistics for CGS4.



The DARK frames used must have been reduced with the \_REDUCE\_DARK\_
primitive; this is normally called from the \_ARRAY\_TESTS\_ primitive
during array test reduction.



The basic principle used to determine the readnoise is to subtract 2
dark frames of the same exposure time, on the assumption that the
readnoise is the only difference between them.

\subparagraph*{AUTHOR\label{_ARRAY_TESTS_STATISTICS__AUTHOR}\index{ ARRAY TESTS STATISTICS !AUTHOR}}

Paul Hirst <p.hirst@jach.hawaii.edu>

\subparagraph*{COPYRIGHT\label{_ARRAY_TESTS_STATISTICS__COPYRIGHT}\index{ ARRAY TESTS STATISTICS !COPYRIGHT}}

Copyright 1998-2001 Particle Physics and Astronomy Research Council.
All Rights Reserved.


%% ORACDRDOC_PRIMITIVE:_FIND_BAD_PIXELS_
\paragraph*{\_FIND\_BAD\_PIXELS\_\label{_FIND_BAD_PIXELS_}\index{\ FIND\ BAD\ PIXELS\ }}

Finds bad pixels in a frame

\subparagraph*{DESCRIPTION\label{_FIND_BAD_PIXELS__DESCRIPTION}\index{ FIND BAD PIXELS !DESCRIPTION}}

Used in array tests to add the bad pixels detected in this frame into the 
bad pixel mask.



Applies thresholds to the data frames and flags pixels outside
the limits as bad.



For BIAS frames, we do a 3-sigma clipped stats and flag pixels more
than 6 sigma from the mean. The actual threshold limits applied are
reported. In addition, we also flag pixels whose BIAS variance is more
than 6-sigma above the mean in a 3-sigma clipped stats.



For DARK frames, we set the thresholds to 1700 and 15 if the exposure
is greater than 80 seconds, and 1500 and -100 if it is not. This will
need updating if MICHELLE elects to use an automated bad pixel
detection scheme in the future and when UIST support is added - these
values are CGS4 specific.



The bad pixels detected are added into the current bad pixel mask and
then this is filed with the calibration system as a new and current
bad pixel mask.

\subparagraph*{AUTHOR\label{_FIND_BAD_PIXELS__AUTHOR}\index{ FIND BAD PIXELS !AUTHOR}}

Paul Hirst <p.hirst@jach.hawaii.edu>

\subparagraph*{COPYRIGHT\label{_FIND_BAD_PIXELS__COPYRIGHT}\index{ FIND BAD PIXELS !COPYRIGHT}}

Copyright (C) 1998-2001 Particle Physics and Astronomy Research
Council. All Rights Reserved.


\subsubsection{EMISSIVITY}

It is not strictly necessary for each instrument to be able to take
emissivity data, though currently both CGS4 and MICHELLE do, and it is
strongly anticipated that UIST also will.

%% ORACDRDOC_RECIPE:EMISSIVITY
\paragraph*{EMISSIVITY\label{EMISSIVITY}\index{EMISSIVITY}}

A recipe to measure telescope emissivity

\subparagraph*{DESCRIPTION\label{EMISSIVITY_DESCRIPTION}\index{EMISSIVITY!DESCRIPTION}}

Used to measure the telescope emissivity. See the \_EMISSIVITY\_
primitive for details of the calculation, this recipe simply calls
\_REDUCE\_SINGLE\_FRAME\_ followed by \_EMISSIVITY\_.

\subparagraph*{AUTHOR\label{EMISSIVITY_AUTHOR}\index{EMISSIVITY!AUTHOR}}

Paul Hirst <p.hirst@jach.hawaii.edu>

\subparagraph*{COPYRIGHT\label{EMISSIVITY_COPYRIGHT}\index{EMISSIVITY!COPYRIGHT}}

Copyright (C) 1998-2001 Particle Physics and Astronomy Research
Council. All Rights Reserved.


%% ORACDRDOC_PRIMITIVE:_EMISSIVITY_
\paragraph*{\_EMISSIVITY\_\label{_EMISSIVITY_}\index{\ EMISSIVITY\ }}

Calculates Telescope Emissivity

\subparagraph*{DESCRIPTION\label{_EMISSIVITY__DESCRIPTION}\index{ EMISSIVITY !DESCRIPTION}}

Expects to be run on a SKY frame (taken pointing at the sky) and an
OBJECT frame (taken pointing at the mirror covers ie dome).



File extensions created:



Frames:

\begin{description}
\item[\textbf{\_ess}] \mbox{}

Emissivity Sky Spectrum

\item[\textbf{\_eds}] \mbox{}

Emissivity Dome Spectrum

\end{description}


Group:



Group file contains ratio of sky and dome spectra

\begin{description}
\item[\textbf{\_sm}] \mbox{}

Smoothed by 5 pixels - used to select region

\end{description}


Spectra are extracted from both files and divided by the exposure time
for that frame.



The "emissivity" spectrum is the ratio of these two spectra.



The emissivity spectrum is smoothed with a box of 5 pixels.  The
location of the minimum of the smoothed spectrum is found, and a
5-pixel window centred at this point is extracted from the sky and
dome spectra.  The mean values of these two windows are calculated and
printed.  The ratio of these two mean values is printed as the
emissivity.

\subparagraph*{PARAMETERS\label{_EMISSIVITY__PARAMETERS}\index{ EMISSIVITY !PARAMETERS}}\begin{description}
\item[START] \mbox{}

The row from which to start extracting.

\item[END] \mbox{}

The row at which to end extracting.

\end{description}
\subparagraph*{ORAC\label{_EMISSIVITY__ORAC}\index{ EMISSIVITY !ORAC}}

Engines referenced: KAPPA

\subparagraph*{AUTHOR\label{_EMISSIVITY__AUTHOR}\index{ EMISSIVITY !AUTHOR}}

Paul Hirst <p.hirst@jach.hawaii.edu>

\subparagraph*{COPYRIGHT\label{_EMISSIVITY__COPYRIGHT}\index{ EMISSIVITY !COPYRIGHT}}

Copyright (C) 1998-2001 Particle Physics and Astronomy Research
Council. All Rights Reserved.


\subsection{Calibration Frame Recipes}

\subsubsection{Flat Fields}

The REDUCE\_FLAT recipe will reduce a FLAT field observation. FLAT
observations are usually taken by observing the black-body source in
the CGS4 calibration unit, though the Tungsten-Halogen lamp is used as
the shorter CGS4 wavelengths. Standard operating procedures for
MICHELLE flats have yet to be ascertained, though REDUCE\_FLAT will
work fine if the (warm) shutter of the instrument is used as a black
body source. Sky flats will probably be reduced by an (as yet
unwritten) separate recipe.

%% ORACDRDOC_RECIPE:REDUCE_FLAT
\paragraph*{REDUCE\_FLAT\label{REDUCE_FLAT}\index{REDUCE\ FLAT}}

Reduces a spectroscopy Flat Field

\subparagraph*{DESCRIPTION\label{REDUCE_FLAT_DESCRIPTION}\index{REDUCE FLAT!DESCRIPTION}}

Reduces a flat field in the conventional manner, including
normalisation by a model of the blackbody source. Files the normalised
flat field frame for use by subsequent flat fielding operations.

\subparagraph*{NOTES\label{REDUCE_FLAT_NOTES}\index{REDUCE FLAT!NOTES}}

Normalising by a black body probably isn't optimal for sky-flats,
especially if they contain strong lines.
This recipe will need updating if Michelle is to use sky flats.

\subparagraph*{AUTHOR\label{REDUCE_FLAT_AUTHOR}\index{REDUCE FLAT!AUTHOR}}

Paul Hirst <p.hirst@jach.hawaii.edu>

\subparagraph*{COPYRIGHT\label{REDUCE_FLAT_COPYRIGHT}\index{REDUCE FLAT!COPYRIGHT}}

Copyright (C) 1998-2001 Particle Physics and Astronomy Research
Council. All Rights Reserved.


%% ORACDRDOC_PRIMITIVE:_NORMALISE_FLAT_BY_BB_
\paragraph*{\_NORMALISE\_FLAT\_BY\_BB\_\label{_NORMALISE_FLAT_BY_BB_}\index{\ NORMALISE\ FLAT\ BY\ BB\ }}

Normalise a flat field frame with a Black Body curve

\subparagraph*{DESCRIPTION\label{_NORMALISE_FLAT_BY_BB__DESCRIPTION}\index{ NORMALISE FLAT BY BB !DESCRIPTION}}

Normalises a frame (usually a CGS4 flat field frame).



First, creates a black body spectrum, using the temperature from the
BBTEMP header, grows this to the size of the image, and divides by
it. Then divides the image by the image's mean pixel value, so as to
normalise it's absolute level to 1.

\subparagraph*{NOTE\label{_NORMALISE_FLAT_BY_BB__NOTE}\index{ NORMALISE FLAT BY BB !NOTE}}

Uses the BBTEMP FITS header. Perhaps this should have a system internal
translated value to cope with future instrumentation.

\subparagraph*{AUTHOR\label{_NORMALISE_FLAT_BY_BB__AUTHOR}\index{ NORMALISE FLAT BY BB !AUTHOR}}

Paul Hirst <p.hirst@jach.hawaii.edu>

\subparagraph*{COPYRIGHT\label{_NORMALISE_FLAT_BY_BB__COPYRIGHT}\index{ NORMALISE FLAT BY BB !COPYRIGHT}}

Copyright (C) 1998-2001 Particle Physics and Astronomy Research
Council. All Rights Reserved.


\subsubsection{Arc Lamp observations}

The REDUCE\_ARC recipe reduces a CGS4 ARC lamp observation. Currently
the arc frame is not actually used for wavelength calibration, though
this will probably change shortly.

%% ORACDRDOC_RECIPE:REDUCE_ARC
\paragraph*{REDUCE\_ARC\label{REDUCE_ARC}\index{REDUCE\ ARC}}

Reduces an Arc Lamp observation

\subparagraph*{DESCRIPTION\label{REDUCE_ARC_DESCRIPTION}\index{REDUCE ARC!DESCRIPTION}}

Reduces an arc lamp observation in the conventional manner, and
applies an estimated wavelength scale (based on the CGS4 motor
positions) so that you can easily refer to an arc line list or map to
check that you are hitting your wavelength region of interest.

\subparagraph*{NOTE\label{REDUCE_ARC_NOTE}\index{REDUCE ARC!NOTE}}

Currently, this recipe does not attempt to use the arc lamp data to
carry out proper wavelength calibration. This will change at some
point in the future.

\subparagraph*{AUTHORS\label{REDUCE_ARC_AUTHORS}\index{REDUCE ARC!AUTHORS}}

Frossie Economou <frossie@jach.hawaii.edu>
Paul Hirst <p.hirst@jach.hawaii.edu>

\subparagraph*{COPYRIGHT\label{REDUCE_ARC_COPYRIGHT}\index{REDUCE ARC!COPYRIGHT}}

Copyright (C) 1998-2001 Particle Physics and Astronomy Research
Council. All Rights Reserved.


The REDUCE\_DARK recipe will reduce and file a DARK observation. In
practice this is very rarely used as the dark current of the array is
intrinsically subtracted when subtracting a sky or offset beam frame.

\subsubsection{Dark Frames}


%% ORACDRDOC_RECIPE:REDUCE_DARK
\paragraph*{REDUCE\_DARK\label{REDUCE_DARK}\index{REDUCE\ DARK}}

Reduce a dark frame

\subparagraph*{DESCRIPTION\label{REDUCE_DARK_DESCRIPTION}\index{REDUCE DARK!DESCRIPTION}}

Reduces a Dark frame. This is generally only used by the array test
and engineering. It is unusual to require a dark frame with CGS4 as
the dark frame would be the same in the offset beam or sky image and
thus cancels when sky subtraction is carried out.

\subparagraph*{AUTHORS\label{REDUCE_DARK_AUTHORS}\index{REDUCE DARK!AUTHORS}}

Frossie Economou <frossie@jach.hawaii.edu>
Paul Hirst <p.hirst@jach.hawaii.edu>

\subparagraph*{COPYRIGHT\label{REDUCE_DARK_COPYRIGHT}\index{REDUCE DARK!COPYRIGHT}}

Copyright (C) 1998-2001 Particle Physics and Astronomy Research
Council. All Rights Reserved.


\subsubsection{Bias Frames}

The REDUCE\_BIAS recipe reduces a bias observation. The default BIAS
observation takes 3 integrations, each containing many exposures. The
REDUCE\_BIAS recipe forms a variance array based on the variance of the
3 integrations in the observation.


%% ORACDRDOC_RECIPE:REDUCE_BIAS
\paragraph*{REDUCE\_BIAS\label{REDUCE_BIAS}\index{REDUCE\ BIAS}}

Reduce a spectrocopy BIAS observation

\subparagraph*{DESCRIPTION\label{REDUCE_BIAS_DESCRIPTION}\index{REDUCE BIAS!DESCRIPTION}}

Reduces a Spectroscopy BIAS observation, including coadding multiple
integrations. Files the reduced bias frame for use by subsequent
reduction of STARE and CHOP mode data.

\subparagraph*{NOTES\label{REDUCE_BIAS_NOTES}\index{REDUCE BIAS!NOTES}}

Creates a variance array for the bias frame, determined from the 
variance of the multiple integrations in the bias observation. There
should be at least 3 integrations in a BIAS observation.

\subparagraph*{SEE ALSO\label{REDUCE_BIAS_SEE_ALSO}\index{REDUCE BIAS!SEE ALSO}}

The \_REDUCE\_BIAS\_ spectroscopy primitive.

\subparagraph*{AUTHORS\label{REDUCE_BIAS_AUTHORS}\index{REDUCE BIAS!AUTHORS}}

Paul Hirst <p.hirst@jach.hawaii.edu>
Frossie Economou <frossie@jach.hawaii.edu>

\subparagraph*{COPYRIGHT\label{REDUCE_BIAS_COPYRIGHT}\index{REDUCE BIAS!COPYRIGHT}}

Copyright (C) 1998-2001 Particle Physics and Astronomy Research
Council. All Rights Reserved.


%% ORACDRDOC_PRIMITIVE:_REDUCE_BIAS_
\paragraph*{\_REDUCE\_BIAS\_\label{_REDUCE_BIAS_}\index{\ REDUCE\ BIAS\ }}

Reduces a spectroscopy BIAS frame

\subparagraph*{DESCRIPTION\label{_REDUCE_BIAS__DESCRIPTION}\index{ REDUCE BIAS !DESCRIPTION}}

Averages together multiple integrations to make the output file data array.



If there are more than 3 integrations, a variance array is created as the
statistical variance of the individual input integrations.



Otherwise, the variance is simply from the readnoise added to the integrations
before averaging.

\subparagraph*{AUTHOR\label{_REDUCE_BIAS__AUTHOR}\index{ REDUCE BIAS !AUTHOR}}

Paul Hirst <p.hirst@jach.hawaii.edu>

\subparagraph*{COPYRIGHT\label{_REDUCE_BIAS__COPYRIGHT}\index{ REDUCE BIAS !COPYRIGHT}}

Copyright (C) 1998-2001 Particle Physics and Astronomy Research
Council. All Rights Reserved.



\subsection{Calibration Star Recipes}

\subsubsection{STANDARD\_STAR}
%% ORACDRDOC_RECIPE:STANDARD_STAR
\paragraph*{STANDARD\_STAR\label{STANDARD_STAR}\index{STANDARD\ STAR}}

Reduce a standard star observation

\subparagraph*{DESCRIPTION\label{STANDARD_STAR_DESCRIPTION}\index{STANDARD STAR!DESCRIPTION}}

This recipe reduces a standard star observation, assuming that it is
observed in a conventional nod-along-slit manner. It extracts the
spectrum of the standard, acquires details of the star (either from a
locally held list, or from SIMBAD), and files it such that it can be
later used to create divided by standard and flux calibrated spectra
of targets that are observed.

\subparagraph*{AUTHORS\label{STANDARD_STAR_AUTHORS}\index{STANDARD STAR!AUTHORS}}

Frossie Economou <frossie@jach.hawaii.edu>
Paul Hirst <p.hirst@jach.hawaii.edu>

\subparagraph*{COPYRIGHT\label{STANDARD_STAR_COPYRIGHT}\index{STANDARD STAR!COPYRIGHT}}

Copyright (C) 1998-2001 Particle Physics and Astronomy Research
Council. All Rights Reserved.

%% ORACDRDOC_RECIPE:STANDARD_STAR_NOFLAT
\paragraph*{STANDARD\_STAR\_NOFLAT\label{STANDARD_STAR_NOFLAT}\index{STANDARD\ STAR\ NOFLAT}}

Reduce a standard star without flat fielding

\subparagraph*{DESCRIPTION\label{STANDARD_STAR_NOFLAT_DESCRIPTION}\index{STANDARD STAR NOFLAT!DESCRIPTION}}

This recipe reduces a standard star observation, assuming that it is
observed in a conventional nod-along-slit manner. It extracts the
spectrum of the standard, acquires details of the star (either from a
locally held list, or from SIMBAD), and files it such that it can be
later used to create divided by standard and flux calibrated spectra
of targets that are observed.



This recipe does not attempt to flat-field the frames.

\subparagraph*{AUTHORS\label{STANDARD_STAR_NOFLAT_AUTHORS}\index{STANDARD STAR NOFLAT!AUTHORS}}

Frossie Economou <frossie@jach.hawaii.edu>
Paul Hirst <p.hirst@jach.hawaii.edu>

\subparagraph*{COPYRIGHT\label{STANDARD_STAR_NOFLAT_COPYRIGHT}\index{STANDARD STAR NOFLAT!COPYRIGHT}}

Copyright (C) 1998-2001 Particle Physics and Astronomy Research
Council. All Rights Reserved.

%% ORACDRDOC_PRIMITIVE:_STANDARD_STAR_
\paragraph*{\_STANDARD\_STAR\_\label{_STANDARD_STAR_}\index{\ STANDARD\ STAR\ }}



\subparagraph*{DESCRIPTION\label{_STANDARD_STAR__DESCRIPTION}\index{ STANDARD STAR !DESCRIPTION}}

Picks up the \texttt{\_sp} file, and does the usual stuff

\subparagraph*{AUTHOR\label{_STANDARD_STAR__AUTHOR}\index{ STANDARD STAR !AUTHOR}}

Paul Hirst <p.hirst@jach.hawaii.edu>

\subparagraph*{COPYRIGHT\label{_STANDARD_STAR__COPYRIGHT}\index{ STANDARD STAR !COPYRIGHT}}

Copyright (C) 1998-2001 Particle Physics and Astronomy Research
Council. All Rights Reserved.


\subsection{Science target Recipes}

\subsubsection{POINT\_SOURCE}

The POINT\_SOURCE recipe reduces observations of point sources, with
nodding along the slit or off to sky (the reason you'd go off to sky
with a point source being that it's in a crowded field).


Variants: \\
\_NOFLAT - does not do flat fielding \\
\_NOSTD - does not ratio by standard star \\
\_MAYBESTD - ratios by standard if there is one, doesn't complain if not. \\
\_NOFLAT\_NOSTD - does not flat field or ratio by standard star \\
\_BASIC - reduced processing for faster execution (maybe - will see if needed) \\

%% ORACDRDOC_RECIPE:POINT_SOURCE
\paragraph*{POINT\_SOURCE\label{POINT_SOURCE}\index{POINT\ SOURCE}}

For reducing point source observations

\subparagraph*{DESCRIPTION\label{POINT_SOURCE_DESCRIPTION}\index{POINT SOURCE!DESCRIPTION}}

A spectroscopy recipe for observations of Point Sources
Assumes that the data are taken in a pair-wise manner.



This recipe is suitable for all point source data taken in a pair-wise
manner, including nodding along the slit, nodding to sky, and
chop-mode observations.



The final product of this recipe is an spectrum of the target, with an
approximate wavelength and flux scale applied. Processing includes
optimal extraction and division by a standard star.



For details of the data reduction process used, see the documentation
for the individual primitives.

\subparagraph*{AUTHOR\label{POINT_SOURCE_AUTHOR}\index{POINT SOURCE!AUTHOR}}

Paul Hirst <p.hirst@jach.hawaii.edu>

\subparagraph*{COPYRIGHT\label{POINT_SOURCE_COPYRIGHT}\index{POINT SOURCE!COPYRIGHT}}

Copyright (C) 1998-2001 Particle Physics and Astronomy Research
Council. All Rights Reserved.

%% ORACDRDOC_RECIPE:POINT_SOURCE_NOFLAT
\paragraph*{POINT\_SOURCE\_NOFLAT\label{POINT_SOURCE_NOFLAT}\index{POINT\ SOURCE\ NOFLAT}}

As POINT\_SOURCE but without flat-fielding

\subparagraph*{DESCRIPTION\label{POINT_SOURCE_NOFLAT_DESCRIPTION}\index{POINT SOURCE NOFLAT!DESCRIPTION}}

See the POINT\_SOURCE recipe documentation. This version is identical except
that it does not attempt to flat field the data.



If you acquire suitable flat fields later in the night, you should
reduce them first, then re-process your target data with the
POINT\_SOURCE recipe.



Note that to be suitable, such data must be taken before driving and
of the spectrometer optics motors. You cannot change configurations
inbetween.

\subparagraph*{AUTHOR\label{POINT_SOURCE_NOFLAT_AUTHOR}\index{POINT SOURCE NOFLAT!AUTHOR}}

Paul Hirst <p.hirst@jach.hawaii.edu>

\subparagraph*{COPYRIGHT\label{POINT_SOURCE_NOFLAT_COPYRIGHT}\index{POINT SOURCE NOFLAT!COPYRIGHT}}

Copyright (C) 1998-2001 Particle Physics and Astronomy Research
Council. All Rights Reserved.

%% ORACDRDOC_RECIPE:POINT_SOURCE_NOSTD
\paragraph*{POINT\_SOURCE\_NOSTD\label{POINT_SOURCE_NOSTD}\index{POINT\ SOURCE\ NOSTD}}

As POINT\_SOURCE but without using a standard star

\subparagraph*{DESCRIPTION\label{POINT_SOURCE_NOSTD_DESCRIPTION}\index{POINT SOURCE NOSTD!DESCRIPTION}}

See the POINT\_SOURCE recipe documentation. This version is identical
except that it does not use a standard star, and thus processing stops
after extraction of the spectrum from the group frame.



If you acquire suitable standard star observations later in the night,
you should reduce them first, then re-process your target data with
the POINT\_SOURCE recipe.



Note that to be suitable, such data must be taken before driving and
of the spectrometer optics motors. You cannot change configurations
inbetween.

\subparagraph*{AUTHOR\label{POINT_SOURCE_NOSTD_AUTHOR}\index{POINT SOURCE NOSTD!AUTHOR}}

Paul Hirst <p.hirst@jach.hawaii.edu>

\subparagraph*{COPYRIGHT\label{POINT_SOURCE_NOSTD_COPYRIGHT}\index{POINT SOURCE NOSTD!COPYRIGHT}}

Copyright (C) 1998-2001 Particle Physics and Astronomy Research
Council. All Rights Reserved.

%% ORACDRDOC_RECIPE:POINT_SOURCE_NOFLAT_NOSTD
\paragraph*{POINT\_SOURCE\_NOFLAT\_NOSTD\label{POINT_SOURCE_NOFLAT_NOSTD}\index{POINT\ SOURCE\ NOFLAT\ NOSTD}}

POINT\_SOURCE minus flat-fielding and standard stars

\subparagraph*{DESCRIPTION\label{POINT_SOURCE_NOFLAT_NOSTD_DESCRIPTION}\index{POINT SOURCE NOFLAT NOSTD!DESCRIPTION}}

See the POINT\_SOURCE recipe documentation. This version is identical
except that it does not flat field the data or use a standard
star. Thus processing stops after extraction of the spectrum from the
group frame.



If you acquire suitable flat fields and standard star observations
later in the night, you should reduce them first, then re-process your
target data with the POINT\_SOURCE recipe.



Note that to be suitable, such data must be taken before driving and
of the spectrometer optics motors. You cannot change configurations
inbetween.

\subparagraph*{AUTHOR\label{POINT_SOURCE_NOFLAT_NOSTD_AUTHOR}\index{POINT SOURCE NOFLAT NOSTD!AUTHOR}}

Paul Hirst <p.hirst@jach.hawaii.edu>

\subparagraph*{COPYRIGHT\label{POINT_SOURCE_NOFLAT_NOSTD_COPYRIGHT}\index{POINT SOURCE NOFLAT NOSTD!COPYRIGHT}}

Copyright (C) 1998-2001 Particle Physics and Astronomy Research
Council. All Rights Reserved.


\subsubsection{FAINT\_POINT\_SOURCE}

The FAINT\_POINT\_SOURCE recipe is similar to POINT\_SOURCE, except in
determining where to centre the opt-extract windows. With
POINT\_SOURCE, the windows are centred on peaks and troughs detected
in the y-profile of the sky subtracted group image. This is also the
case for STANDARD\_STAR. However, STANDARD\_STAR writes the locations
of the beams to the rows calibration system. FAINT\_POINT\_SOURCE gets
these values from the calibration system rather than the y-profile of
the image. 

Thus, FAINT\_POINT\_SOURCE is useful both for crowded fields or
faint targets, where it's not obvious from the group image where to
extract. For this to be useful, the object of interest must be centred
on the same rows as the standard star was. This will be the case unles
you deliberately change the telescope-instrument aperture, peak-up row
or offset distance between observing the standard and the target.

Variants: \\
\_NOFLAT - does not do flat fielding \\
\_NOSTD - does not ratio by standard star \\
\_MAYBESTD - ratios by standard if there is one, doesn't complain if not. \\
\_NOFLAT\_NOSTD - does not flat field or ratio by standard star \\
\_BASIC - reduced processing for faster execution (maybe - will see if needed) \\

%% ORACDRDOC_RECIPE:FAINT_POINT_SOURCE
\paragraph*{FAINT\_POINT\_SOURCE\label{FAINT_POINT_SOURCE}\index{FAINT\ POINT\ SOURCE}}

For reducing faint point source observations

\subparagraph*{DESCRIPTION\label{FAINT_POINT_SOURCE_DESCRIPTION}\index{FAINT POINT SOURCE!DESCRIPTION}}

A spectroscopy recipe for observations of Point Sources
Assumes that the data are taken in a pair-wise manner.



This recipe is suitable for all point source data taken in a pair-wise
manner, including nodding along the slit, nodding to sky, and
chop-mode observations.



The final product of this recipe is an spectrum of the target, with an
approximate wavelength and flux scale applied. Processing includes
optimal extraction and division by a standard star.



For details of the data reduction process used, see the documentation
for the individual primitives.



This Optimal Extraction windows used to extract spectra from the group
image are centred on the rows determined for the standard star.

\subparagraph*{AUTHOR\label{FAINT_POINT_SOURCE_AUTHOR}\index{FAINT POINT SOURCE!AUTHOR}}

Paul Hirst <p.hirst@jach.hawaii.edu>

\subparagraph*{COPYRIGHT\label{FAINT_POINT_SOURCE_COPYRIGHT}\index{FAINT POINT SOURCE!COPYRIGHT}}

Copyright (C) 1998-2001 Particle Physics and Astronomy Research
Council. All Rights Reserved.

%% ORACDRDOC_RECIPE:FAINT_POINT_SOURCE_NOFLAT
\paragraph*{FAINT\_POINT\_SOURCE\_NOFLAT\label{FAINT_POINT_SOURCE_NOFLAT}\index{FAINT\ POINT\ SOURCE\ NOFLAT}}

As FAINT\_POINT\_SOURCE but without flat fielding

\subparagraph*{DESCRIPTION\label{FAINT_POINT_SOURCE_NOFLAT_DESCRIPTION}\index{FAINT POINT SOURCE NOFLAT!DESCRIPTION}}

See the FAINT\_POINT\_SOURCE recipe documentation. This version is
identical except that it does not attempt to flat field the data.



If you acquire suitable flat fields later in the night, you should
reduce them first, then re-process your target data with the
FAINT\_POINT\_SOURCE recipe.



Note that to be suitable, such data must be taken before driving and
of the spectrometer optics motors. You cannot change configurations
inbetween.

\subparagraph*{AUTHOR\label{FAINT_POINT_SOURCE_NOFLAT_AUTHOR}\index{FAINT POINT SOURCE NOFLAT!AUTHOR}}

Paul Hirst <p.hirst@jach.hawaii.edu>

\subparagraph*{COPYRIGHT\label{FAINT_POINT_SOURCE_NOFLAT_COPYRIGHT}\index{FAINT POINT SOURCE NOFLAT!COPYRIGHT}}

Copyright (C) 1998-2001 Particle Physics and Astronomy Research
Council. All Rights Reserved.

%% ORACDRDOC_RECIPE:FAINT_POINT_SOURCE_NOSTD
\paragraph*{FAINT\_POINT\_SOURCE\_NOSTD\label{FAINT_POINT_SOURCE_NOSTD}\index{FAINT\ POINT\ SOURCE\ NOSTD}}

As FAINT\_POINT\_SOURCE but without using a standard star

\subparagraph*{DESCRIPTION\label{FAINT_POINT_SOURCE_NOSTD_DESCRIPTION}\index{FAINT POINT SOURCE NOSTD!DESCRIPTION}}

See the FAINT\_POINT\_SOURCE recipe documentation. This version is identical
except that it does not use a standard star, and thus processing stops
after extraction of the spectrum from the group frame.



If you acquire suitable standard star observations later in the night,
you should reduce them first, then re-process your target data with
the FAINT\_POINT\_SOURCE recipe.



Note that to be suitable, such data must be taken before driving and
of the spectrometer optics motors. You cannot change configurations
inbetween.

\subparagraph*{AUTHOR\label{FAINT_POINT_SOURCE_NOSTD_AUTHOR}\index{FAINT POINT SOURCE NOSTD!AUTHOR}}

Paul Hirst <p.hirst@jach.hawaii.edu>

\subparagraph*{COPYRIGHT\label{FAINT_POINT_SOURCE_NOSTD_COPYRIGHT}\index{FAINT POINT SOURCE NOSTD!COPYRIGHT}}

Copyright (C) 1998-2001 Particle Physics and Astronomy Research
Council. All Rights Reserved.

%% ORACDRDOC_RECIPE:FAINT_POINT_SOURCE_NOFLAT_NOSTD
\paragraph*{FAINT\_POINT\_SOURCE\_NOFLAT\_NOSTD\label{FAINT_POINT_SOURCE_NOFLAT_NOSTD}\index{FAINT\ POINT\ SOURCE\ NOFLAT\ NOSTD}}

FAINT\_POINT\_SOURCE minus flats and standards

\subparagraph*{DESCRIPTION\label{FAINT_POINT_SOURCE_NOFLAT_NOSTD_DESCRIPTION}\index{FAINT POINT SOURCE NOFLAT NOSTD!DESCRIPTION}}

See the FAINT\_POINT\_SOURCE recipe documentation. This version is
identical except that it does not flat field the data or use a
standard star, and thus processing stops after extraction of the
spectrum from the group frame.



If you acquire suitable flat field and standard star observations
later in the night, you should reduce them first, then re-process your
target data with the FAINT\_POINT\_SOURCE recipe.



Note that to be suitable, such data must be taken before driving and
of the spectrometer optics motors. You cannot change configurations
inbetween.

\subparagraph*{AUTHOR\label{FAINT_POINT_SOURCE_NOFLAT_NOSTD_AUTHOR}\index{FAINT POINT SOURCE NOFLAT NOSTD!AUTHOR}}

Paul Hirst <p.hirst@jach.hawaii.edu>

\subparagraph*{COPYRIGHT\label{FAINT_POINT_SOURCE_NOFLAT_NOSTD_COPYRIGHT}\index{FAINT POINT SOURCE NOFLAT NOSTD!COPYRIGHT}}

Copyright (C) 1998-2001 Particle Physics and Astronomy Research
Council. All Rights Reserved.


\subsubsection{EXTENDED\_SOURCE}

The EXTENDED\_SOURCE recipe reduces observations of extended sources.

Variants: \\
\_NOFLAT - does not do flat fielding \\
\_NOSTD - does not ratio by standard star \\
\_MAYBESTD - ratios by standard if there is one, doesn't complain if not. \\
\_NOFLAT\_NOSTD - does not flat field or ratio by standard star \\
\_BASIC - reduced processing for faster execution (maybe - will see if needed) \\

%% ORACDRDOC_RECIPE:EXTENDED_SOURCE
\paragraph*{EXTENDED\_SOURCE\label{EXTENDED_SOURCE}\index{EXTENDED\ SOURCE}}

For reducing extended source observations

\subparagraph*{DESCRIPTION\label{EXTENDED_SOURCE_DESCRIPTION}\index{EXTENDED SOURCE!DESCRIPTION}}

A spectroscopy recipe for observations of Extended Sources
Assumes that the data are taken in a pair-wise manner.



This recipe is suitable for all extended source data taken in a pair-wise
manner, including nodding along the slit, nodding to sky, and
chop-mode observations.



The final product of this recipe is a sky subtracted image spectrum of
the target, with an approximate wavelength and flux scale
applied. Processing includes division by a standard star.



For details of the data reduction process used, see the documentation
for the individual primitives.

\subparagraph*{AUTHOR\label{EXTENDED_SOURCE_AUTHOR}\index{EXTENDED SOURCE!AUTHOR}}

Paul Hirst <p.hirst@jach.hawaii.edu>

\subparagraph*{COPYRIGHT\label{EXTENDED_SOURCE_COPYRIGHT}\index{EXTENDED SOURCE!COPYRIGHT}}

Copyright (C) 1998-2001 Particle Physics and Astronomy Research
Council. All Rights Reserved.

%% ORACDRDOC_RECIPE:EXTENDED_SOURCE_NOFLAT
\paragraph*{EXTENDED\_SOURCE\_NOFLAT\label{EXTENDED_SOURCE_NOFLAT}\index{EXTENDED\ SOURCE\ NOFLAT}}

EXTENDED\_SOURCE without flat-fielding

\subparagraph*{DESCRIPTION\label{EXTENDED_SOURCE_NOFLAT_DESCRIPTION}\index{EXTENDED SOURCE NOFLAT!DESCRIPTION}}

See the documentation for the EXTENDED\_SOURCE recipe. This version is
identical except that it does not attempt to flat-field the data.



If you acquire suitable flat fields later in the night, you should
reduce them first, then re-process your target data with the
EXTENDED\_SOURCE recipe.



Note that to be suitable, such data must be taken before driving and
of the spectrometer optics motors. You cannot change configurations
inbetween.

\subparagraph*{AUTHOR\label{EXTENDED_SOURCE_NOFLAT_AUTHOR}\index{EXTENDED SOURCE NOFLAT!AUTHOR}}

Paul Hirst <p.hirst@jach.hawaii.edu>

\subparagraph*{COPYRIGHT\label{EXTENDED_SOURCE_NOFLAT_COPYRIGHT}\index{EXTENDED SOURCE NOFLAT!COPYRIGHT}}

Copyright (C) 1998-2001 Particle Physics and Astronomy Research
Council. All Rights Reserved.

%% ORACDRDOC_RECIPE:EXTENDED_SOURCE_NOSTD
\paragraph*{EXTENDED\_SOURCE\_NOSTD\label{EXTENDED_SOURCE_NOSTD}\index{EXTENDED\ SOURCE\ NOSTD}}

EXTENDED\_SOURCE without using a standard star.

\subparagraph*{DESCRIPTION\label{EXTENDED_SOURCE_NOSTD_DESCRIPTION}\index{EXTENDED SOURCE NOSTD!DESCRIPTION}}

See the documentation for the EXTENDED\_SOURCE recipe. This version is
identical except that it does not attempt to use a standard star and
thus processing stops after forming the group image.



If you acquire suitable standard star observations later in the night,
you should reduce them first, then re-process your target data with
the EXTENDED\_SOURCE recipe.



Note that to be suitable, such data must be taken before driving and
of the spectrometer optics motors. You cannot change configurations
inbetween.

\subparagraph*{AUTHOR\label{EXTENDED_SOURCE_NOSTD_AUTHOR}\index{EXTENDED SOURCE NOSTD!AUTHOR}}

Paul Hirst <p.hirst@jach.hawaii.edu>

\subparagraph*{COPYRIGHT\label{EXTENDED_SOURCE_NOSTD_COPYRIGHT}\index{EXTENDED SOURCE NOSTD!COPYRIGHT}}

Copyright (C) 1998-2001 Particle Physics and Astronomy Research
Council. All Rights Reserved.

%% ORACDRDOC_RECIPE:EXTENDED_SOURCE_NOFLAT_NOSTD
\paragraph*{EXTENDED\_SOURCE\_NOFLAT\_NOSTD\label{EXTENDED_SOURCE_NOFLAT_NOSTD}\index{EXTENDED\ SOURCE\ NOFLAT\ NOSTD}}

EXTENDED\_SOURCE without flat-fields or std stars.

\subparagraph*{DESCRIPTION\label{EXTENDED_SOURCE_NOFLAT_NOSTD_DESCRIPTION}\index{EXTENDED SOURCE NOFLAT NOSTD!DESCRIPTION}}

See the documentation for the EXTENDED\_SOURCE recipe. This version is
identical except that it does not attempt to flat-field the data or
use a standard star and thus processing stops after forming the group
image.



If you acquire suitable flat-field and standard star observations later
in the night, you should reduce them first, then re-process your
target data with the EXTENDED\_SOURCE recipe.



Note that to be suitable, such data must be taken before driving and
of the spectrometer optics motors. You cannot change configurations
inbetween.

\subparagraph*{AUTHOR\label{EXTENDED_SOURCE_NOFLAT_NOSTD_AUTHOR}\index{EXTENDED SOURCE NOFLAT NOSTD!AUTHOR}}

Paul Hirst <p.hirst@jach.hawaii.edu>

\subparagraph*{COPYRIGHT\label{EXTENDED_SOURCE_NOFLAT_NOSTD_COPYRIGHT}\index{EXTENDED SOURCE NOFLAT NOSTD!COPYRIGHT}}

Copyright (C) 1998-2001 Particle Physics and Astronomy Research
Council. All Rights Reserved.


\subsubsection{Extended source with a stable sky background}

The EXTENDED\_SOURCE\_SEPARATE\_SKY recipe is for use when the sky is
sufficiently stable, that you do not need to spend half your time
observing it. This is only expected to be of use with CGS4 echelle
observations. For point sources, you would nod along the slit anyway.

Variants: \\
\_NOFLAT - does not do flat fielding \\
\_NOSTD - does not ratio by standard star \\
\_NOFLAT\_NOSTD - does not flat field or ratio by standard star \\
\_BASIC - reduced processing for faster execution (maybe - will see if needed) \\

%% ORACDRDOC_RECIPE:EXTENDED_SOURCE_WITH_SEPARATE_SKY
\paragraph*{EXTENDED\_SOURCE\_WITH\_SEPARATE\_SKY\label{EXTENDED_SOURCE_WITH_SEPARATE_SKY}\index{EXTENDED\ SOURCE\ WITH\ SEPARATE\ SKY}}

For extended source on stable sky background.

\subparagraph*{DESCRIPTION\label{EXTENDED_SOURCE_WITH_SEPARATE_SKY_DESCRIPTION}\index{EXTENDED SOURCE WITH SEPARATE SKY!DESCRIPTION}}

For use when you want to take several on-target OBJECT frames for each
off-target SKY frame. It is only sensible to use this recipe when
working with low, stable sky counts, and with extended targets where
you cannot nod along the slit. Currently, this can only really be
considered to apply to certain observations using the CGS4 echelle
grating.



Before the pipeline sees an OBJECT frame with this recipe, you must
have had it reduce a suitable SKY frame with the REDUCE\_SKY recipe,
otherwise the pipeline cannot continue. Thus you should use a sequence
like "SKY, OBJECT, OBJECT" rather than "OBJECT, SKY, OBJECT". In each
set of observations you can follow the SKY frame with as many OBJECT
frames as you like. There is obviously a trade-off in that the more
OBJECT frames you do, the more time you spend observing your target
rather than the sky, but also the longer the interval between SKY
frames, and thus the more liable you are to be affected by sky
variation.

\subparagraph*{AUTHORS\label{EXTENDED_SOURCE_WITH_SEPARATE_SKY_AUTHORS}\index{EXTENDED SOURCE WITH SEPARATE SKY!AUTHORS}}

Paul Hirst <p.hirst@jach.hawaii.edu>

\subparagraph*{COPYRIGHT\label{EXTENDED_SOURCE_WITH_SEPARATE_SKY_COPYRIGHT}\index{EXTENDED SOURCE WITH SEPARATE SKY!COPYRIGHT}}

Copyright (C) 1998-2001 Particle Physics and Astronomy Research
Council. All Rights Reserved.

%% ORACDRDOC_RECIPE:EXTENDED_SOURCE_WITH_SEPARATE_SKY_NOFLAT
\paragraph*{EXTENDED\_SOURCE\_WITH\_SEPARATE\_SKY\_NOFLAT\label{EXTENDED_SOURCE_WITH_SEPARATE_SKY_NOFLAT}\index{EXTENDED\ SOURCE\ WITH\ SEPARATE\ SKY\ NOFLAT}}

EXTENDED\_SOURCE\_WITH\_SEPARATE\_SKY without flat-fielding.

\subparagraph*{DESCRIPTION\label{EXTENDED_SOURCE_WITH_SEPARATE_SKY_NOFLAT_DESCRIPTION}\index{EXTENDED SOURCE WITH SEPARATE SKY NOFLAT!DESCRIPTION}}

See the documentation for the EXTENDED\_SOURCE\_WITH\_SEPARATE\_SKY
recipe. This version is identical except that it does not attempt to
flat-field the data.



If you acquire suitable flat fields later in the night, you should
reduce them first, then re-process your target data with the
EXTENDED\_SOURCE\_WITH\_SEPARATE\_SKY recipe.



Note that to be suitable, such data must be taken before driving and
of the spectrometer optics motors. You cannot change configurations
inbetween.

\subparagraph*{AUTHORS\label{EXTENDED_SOURCE_WITH_SEPARATE_SKY_NOFLAT_AUTHORS}\index{EXTENDED SOURCE WITH SEPARATE SKY NOFLAT!AUTHORS}}

Paul Hirst <p.hirst@jach.hawaii.edu>

\subparagraph*{COPYRIGHT\label{EXTENDED_SOURCE_WITH_SEPARATE_SKY_NOFLAT_COPYRIGHT}\index{EXTENDED SOURCE WITH SEPARATE SKY NOFLAT!COPYRIGHT}}

Copyright (C) 1998-2001 Particle Physics and Astronomy Research
Council. All Rights Reserved.

%% ORACDRDOC_RECIPE:EXTENDED_SOURCE_WITH_SEPARATE_SKY_NOSTD
\paragraph*{EXTENDED\_SOURCE\_WITH\_SEPARATE\_SKY\_NOSTD\label{EXTENDED_SOURCE_WITH_SEPARATE_SKY_NOSTD}\index{EXTENDED\ SOURCE\ WITH\ SEPARATE\ SKY\ NOSTD}}

EXTENDED\_SOURCE\_WITH\_SEPARATE\_SKY without std star.

\subparagraph*{DESCRIPTION\label{EXTENDED_SOURCE_WITH_SEPARATE_SKY_NOSTD_DESCRIPTION}\index{EXTENDED SOURCE WITH SEPARATE SKY NOSTD!DESCRIPTION}}

See the documentation for the EXTENDED\_SOURCE\_WITH\_SEPARATE\_SKY
recipe. This version is identical except that it does not attempt to
use a standard star, and thus processing stops after forming the group
image.



If you acquire suitable standard star observations later in the night,
you should reduce them first, then re-process your target data with
the EXTENDED\_SOURCE\_WITH\_SEPARATE\_SKY recipe.



Note that to be suitable, such data must be taken before driving and
of the spectrometer optics motors. You cannot change configurations
inbetween.

\subparagraph*{AUTHORS\label{EXTENDED_SOURCE_WITH_SEPARATE_SKY_NOSTD_AUTHORS}\index{EXTENDED SOURCE WITH SEPARATE SKY NOSTD!AUTHORS}}

Paul Hirst <p.hirst@jach.hawaii.edu>

\subparagraph*{COPYRIGHT\label{EXTENDED_SOURCE_WITH_SEPARATE_SKY_NOSTD_COPYRIGHT}\index{EXTENDED SOURCE WITH SEPARATE SKY NOSTD!COPYRIGHT}}

Copyright (C) 1998-2001 Particle Physics and Astronomy Research
Council. All Rights Reserved.

%% ORACDRDOC_RECIPE:EXTENDED_SOURCE_WITH_SEPARATE_SKY_NOFLAT_NOSTD
\paragraph*{EXTENDED\_SOURCE\_WITH\_SEPARATE\_SKY\_NOFLAT\_NOSTD\label{EXTENDED_SOURCE_WITH_SEPARATE_SKY_NOFLAT_NOSTD}\index{EXTENDED\ SOURCE\ WITH\ SEPARATE\ SKY\ NOFLAT\ NOSTD}}

EXTENDED\_SOURCE\_WITH\_SEPARATE\_SKY without flats or stds.

\subparagraph*{DESCRIPTION\label{EXTENDED_SOURCE_WITH_SEPARATE_SKY_NOFLAT_NOSTD_DESCRIPTION}\index{EXTENDED SOURCE WITH SEPARATE SKY NOFLAT NOSTD!DESCRIPTION}}

See the documentation for the EXTENDED\_SOURCE\_WITH\_SEPARATE\_SKY
recipe. This version is identical except that it does not attempt to
flat field the data or use a standard star, and thus processing stops
after forming the group image.



If you acquire suitable flat fields and standard star observations
later in the night, you should reduce them first, then re-process your
target data with the EXTENDED\_SOURCE\_WITH\_SEPARATE\_SKY recipe.



Note that to be suitable, such data must be taken before driving and
of the spectrometer optics motors. You cannot change configurations
inbetween.

\subparagraph*{AUTHORS\label{EXTENDED_SOURCE_WITH_SEPARATE_SKY_NOFLAT_NOSTD_AUTHORS}\index{EXTENDED SOURCE WITH SEPARATE SKY NOFLAT NOSTD!AUTHORS}}

Paul Hirst <p.hirst@jach.hawaii.edu>

\subparagraph*{COPYRIGHT\label{EXTENDED_SOURCE_WITH_SEPARATE_SKY_NOFLAT_NOSTD_COPYRIGHT}\index{EXTENDED SOURCE WITH SEPARATE SKY NOFLAT NOSTD!COPYRIGHT}}

Copyright (C) 1998-2001 Particle Physics and Astronomy Research
Council. All Rights Reserved.


\subsubsection{Blank Sky Observations}

%% ORACDRDOC_RECIPE:REDUCE_SKY
\paragraph*{REDUCE\_SKY\label{REDUCE_SKY}\index{REDUCE\ SKY}}

Reduces a sky frame

\subparagraph*{DESCRIPTION\label{REDUCE_SKY_DESCRIPTION}\index{REDUCE SKY!DESCRIPTION}}

Reduces a blank sky observation, and files it with the calibration
system for use in subsequent data reduction. Note that in pair-wise
observing procedures, you should generally not be taking SKY frames -
both beam positions are classified as OBJECT frames, and should be
handled by whatever pair-wise recipe is being used for the main-beam
frames.

\subparagraph*{AUTHOR\label{REDUCE_SKY_AUTHOR}\index{REDUCE SKY!AUTHOR}}

Paul Hirst <p.hirst@jach.hawaii.edu>

\subparagraph*{COPYRIGHT\label{REDUCE_SKY_COPYRIGHT}\index{REDUCE SKY!COPYRIGHT}}

Copyright (C) 1998-2001 Particle Physics and Astronomy Research
Council. All Rights Reserved.


\subsection{Utility Recipes}

\subsubsection{Night Logs}

The NIGHT\_LOG recipe is the default recipe for generating summary
logs of a list of observations. The NIGHT\_LOG\_LONG variant adds more
details to the log file.

%% ORACDRDOC_RECIPE:NIGHT_LOG
\paragraph*{NIGHT\_LOG\label{NIGHT_LOG}\index{NIGHT\ LOG}}

Creates a text log summarising file headers

\subparagraph*{DESCRIPTION\label{NIGHT_LOG_DESCRIPTION}\index{NIGHT LOG!DESCRIPTION}}

This recipe is used to create a text log summarising the file headers
of a group of observations. It is often used to create a log file
describing a whole night's worth of observations.



For full details, see the documentation for the spectroscopy
\_NIGHT\_LOG primitive.



This recipe calls the primitive in such a way that the log file
appears in \$ORAC\_DATA\_IN.



An "on-the-fly" night log is created in \$ORAC\_DATA\_OUT as 
spectroscopy data is reduced by the pipeline. This is done by a
call to \_NIGHT\_LOG\_ from \_SPECTROSCOPY\_HELLO\_

\subparagraph*{SEE ALSO\label{NIGHT_LOG_SEE_ALSO}\index{NIGHT LOG!SEE ALSO}}

The spectroscopy recipe: NIGHT\_LOG\_LONG
The spectroscopy primitive: \_NIGHT\_LOG\_

\subparagraph*{AUTHOR\label{NIGHT_LOG_AUTHOR}\index{NIGHT LOG!AUTHOR}}

Paul Hirst <p.hirst@jach.hawaii.edu>

\subparagraph*{COPYRIGHT\label{NIGHT_LOG_COPYRIGHT}\index{NIGHT LOG!COPYRIGHT}}

Copyright (C) 1998-2001 Particle Physics and Astronomy Research
Council. All Rights Reserved.

%% ORACDRDOC_RECIPE:NIGHT_LOG_LONG
\paragraph*{NIGHT\_LOG\_LONG\label{NIGHT_LOG_LONG}\index{NIGHT\ LOG\ LONG}}

Creates a text log detailing file headers

\subparagraph*{DESCRIPTION\label{NIGHT_LOG_LONG_DESCRIPTION}\index{NIGHT LOG LONG!DESCRIPTION}}

This recipe does the same as the NIGHT\_LOG recipe, except that it
produces a more detailed log.

\subparagraph*{SEE ALSO\label{NIGHT_LOG_LONG_SEE_ALSO}\index{NIGHT LOG LONG!SEE ALSO}}

The spectroscopy recipe: NIGHT\_LOG
The spectroscopy primitive: \_NIGHT\_LOG\_

\subparagraph*{AUTHOR\label{NIGHT_LOG_LONG_AUTHOR}\index{NIGHT LOG LONG!AUTHOR}}

Paul Hirst <p.hirst@jach.hawaii.edu>

\subparagraph*{COPYRIGHT\label{NIGHT_LOG_LONG_COPYRIGHT}\index{NIGHT LOG LONG!COPYRIGHT}}

Copyright (C) 1998-2001 Particle Physics and Astronomy Research
Council. All Rights Reserved.

%% ORACDRDOC_PRIMITIVE:_NIGHT_LOG_
\paragraph*{\_NIGHT\_LOG\_\label{_NIGHT_LOG_}\index{\ NIGHT\ LOG\ }}

Produces a text listing a summary of a frame's headers.

\subparagraph*{DESCRIPTION\label{_NIGHT_LOG__DESCRIPTION}\index{ NIGHT LOG !DESCRIPTION}}

Produces a line of text in a log file summarising the header values of
the frame. Is used both as part of the generall data reduction, so as
to produce an on-the-fly listing of what has been reduced so far, and
as part of the NIGHT\_LOG recipe to provide a summary of a set (usually
the whole night's worth) of observations.

\subparagraph*{ARGUMENTS\label{_NIGHT_LOG__ARGUMENTS}\index{ NIGHT LOG !ARGUMENTS}}

The following arguments are available:

\begin{description}
\item[OUT] \mbox{}

When set (to any value) the log file is written to \emph{\$ORAC\_DATA\_OUT}
rather than to \emph{\$ORAC\_DATA\_IN}.

\end{description}
\subparagraph*{OUTPUT DATA\label{_NIGHT_LOG__OUTPUT_DATA}\index{ NIGHT LOG !OUTPUT DATA}}

The text log file \emph{$<$date$>$.nightlog}, where
$<$date$>$ is the UT date.



The file is in \emph{\$ORAC\_DATA\_IN} by default, or in \emph{\$ORAC\_DATA\_OUT} if
the OUT argument is set. This primitive is called from
\_SPECTROSCOPY\_HELLO\_ with the OUT argument set, so the on-the-fly
nightlog appears in \emph{\$ORAC\_DATA\_OUT}, and without the argument from
the NIGHT\_LOG recipe, so that we can create a summary of a night's
observations in \emph{\$ORAC\_DATA\_IN}.



The on-the-fly log in \emph{\$ORAC\_DATA\_OUT} is always appended to, being
created only if it doesn't exist.  Thus multiple entries for the same
observation may exist in the on-the-fly log if the pipeline is rerun.



The "clean" log file in \emph{\$ORAC\_DATA\_IN} is re-started if the observation
number equals 1 and is appended to otherwise, being created as necessary.

\subparagraph*{NOTES\label{_NIGHT_LOG__NOTES}\index{ NIGHT LOG !NOTES}}\begin{itemize}
\item 

The logfile created by this primitive does not follow the standard
ORAC-DR naming convention (\emph{log.xxxx}) since it can be used to write
log files to directories other than \emph{\$ORAC\_DATA\_OUT} and unique file
names are required.

\item 

No external algorithm engines are required by this primitive.

\item 

Run with the oracdr option -resume to prevent deletion of allready existing group files

\item 

Run with the oracdr option -noeng for efficiency

\end{itemize}
\subparagraph*{AUTHORS\label{_NIGHT_LOG__AUTHORS}\index{ NIGHT LOG !AUTHORS}}

Frossie Economou (JAC)



Paul Hirst (JAC) <p.hirst@jach.hawaii.edu>



Tim Jenness (JAC)

\subparagraph*{COPYRIGHT\label{_NIGHT_LOG__COPYRIGHT}\index{ NIGHT LOG !COPYRIGHT}}

Copyright (C) 1998-2001 Particle Physics and Astronomy Research
Council.  All Rights Reserved.


\subsection{Molecular Primitives}

\subsubsection{To Reduce a Single Observation to a \_wce frame}

%% ORACDRDOC_PRIMITIVE:_REDUCE_SINGLE_FRAME_
\paragraph*{\_REDUCE\_SINGLE\_FRAME\_\label{_REDUCE_SINGLE_FRAME_}\index{\ REDUCE\ SINGLE\ FRAME\ }}

Reduces a spectroscopy frame

\subparagraph*{DESCRIPTION\label{_REDUCE_SINGLE_FRAME__DESCRIPTION}\index{ REDUCE SINGLE FRAME !DESCRIPTION}}

Intended to be run on all OBJECT and SKY science data, and also things
like ARC frames aswell.



Contains all the steps necessary to get from raw data to a \_wce file.
Variance should be propogated throughout.



This should be the first major primitive in any recipe handling on-sky
data.

\subparagraph*{PARAMETERS\label{_REDUCE_SINGLE_FRAME__PARAMETERS}\index{ REDUCE SINGLE FRAME !PARAMETERS}}

An optional NOFLAT parameter, that disables flat fielding.
This is used by the \_NOFLAT recipe variants and also when
reducing a flat field frame



An optional NOBIAS parameter, that diables use of a bias frame.
If we're not in an ND mode, this means that we cannot form a sensible
variance array, or of course, subtract a BIAS frame from the data.

\subparagraph*{AUTHOR\label{_REDUCE_SINGLE_FRAME__AUTHOR}\index{ REDUCE SINGLE FRAME !AUTHOR}}

Paul Hirst <p.hirst@jach.hawaii.edu>

\subparagraph*{COPYRIGHT\label{_REDUCE_SINGLE_FRAME__COPYRIGHT}\index{ REDUCE SINGLE FRAME !COPYRIGHT}}

Copyright (C) 1998-2001 Particle Physics and Astronomy Research
Council. All Rights Reserved.


\subsubsection{Pairwise Grouping}

%% ORACDRDOC_PRIMITIVE:_PAIRWISE_GROUP_
\paragraph*{\_PAIRWISE\_GROUP\_\label{_PAIRWISE_GROUP_}\index{\ PAIRWISE\ GROUP\ }}



\subparagraph*{DESCRIPTION\label{_PAIRWISE_GROUP__DESCRIPTION}\index{ PAIRWISE GROUP !DESCRIPTION}}

Takes Reduced single frames taken in a pairwise sequence, and groups them
to make a gc file. Extracts spectra.

\subparagraph*{AUTHOR\label{_PAIRWISE_GROUP__AUTHOR}\index{ PAIRWISE GROUP !AUTHOR}}

Paul Hirst <p.hirst@jach.hawaii.edu>

\subparagraph*{COPYRIGHT\label{_PAIRWISE_GROUP__COPYRIGHT}\index{ PAIRWISE GROUP !COPYRIGHT}}

Copyright (C) 1998-2001 Particle Physics and Astronomy Research
Council. All Rights Reserved.


\subsubsection{Extracting and Coadding Spectra}

%% ORACDRDOC_PRIMITIVE:_EXTRACT_SPECTRA_
\paragraph*{\_EXTRACT\_SPECTRA\_\label{_EXTRACT_SPECTRA_}\index{\ EXTRACT\ SPECTRA\ }}



\subparagraph*{DESCRIPTION\label{_EXTRACT_SPECTRA__DESCRIPTION}\index{ EXTRACT SPECTRA !DESCRIPTION}}

Extracts spectra from an image

\subparagraph*{PARAMETERS\label{_EXTRACT_SPECTRA__PARAMETERS}\index{ EXTRACT SPECTRA !PARAMETERS}}

This primitive simply passes on parameters to the lower level
primitives it calls.



Availiable parameters are:



ROWSCAL: [LOAD$|$SAVE] - instructs the \_EXTRACT\_FIND\_ROWS\_ primitive
whether to load details from the calibration system, save details to
the calibration system, or niether.



PROFILECAL : [LOAD$|$SAVE] - instructs the \_EXTRACT\_ALL\_BEAMS\_ primitive
whether to load details from the calibration system, save details to
the calibration system, or niether.

\subparagraph*{AUTHOR\label{_EXTRACT_SPECTRA__AUTHOR}\index{ EXTRACT SPECTRA !AUTHOR}}

Paul Hirst <p.hirst@jach.hawaii.edu>

\subparagraph*{COPYRIGHT\label{_EXTRACT_SPECTRA__COPYRIGHT}\index{ EXTRACT SPECTRA !COPYRIGHT}}

Copyright (C) 1998-2001 Particle Physics and Astronomy Research
Council. All Rights Reserved.

%% ORACDRDOC_PRIMITIVE:_EXTRACT_DETERMINE_NBEAMS_
\paragraph*{\_EXTRACT\_DETERMINE\_NBEAMS\_\label{_EXTRACT_DETERMINE_NBEAMS_}\index{\ EXTRACT\ DETERMINE\ NBEAMS\ }}



\subparagraph*{DESCRIPTION\label{_EXTRACT_DETERMINE_NBEAMS__DESCRIPTION}\index{ EXTRACT DETERMINE NBEAMS !DESCRIPTION}}

Looks at the chop and offset headers to determine the number of beams
there should be in the group image. Leaves the result in the NBEAMS
group user header.

\subparagraph*{HEURISTICS\label{_EXTRACT_DETERMINE_NBEAMS__HEURISTICS}\index{ EXTRACT DETERMINE NBEAMS !HEURISTICS}}

We declare an angle "along the slit" if the angle is within 5 degrees
of the slit angle. This was chosen as it represents roughly 1
arcsecond over a 10 arcsecond throw.



At the moment, we do not factor in the slit length, so throws to a
position off the end of the slit will count as "along slit".



Possibilities:

\begin{verbatim}
                   CHOP          OFFSET      nbeams
                   no            to-sky      1
                   no            along-slit  2
                   to-sky        to-sky      1
                   to-sky        along-slit  2
                   along-slit    to-sky      2
                   along-slit != along-slit  4
                   along-slit == along-slit  3
\end{verbatim}
\subparagraph*{AUTHOR\label{_EXTRACT_DETERMINE_NBEAMS__AUTHOR}\index{ EXTRACT DETERMINE NBEAMS !AUTHOR}}

Paul Hirst <p.hirst@jach.hawaii.edu>

\subparagraph*{COPYRIGHT\label{_EXTRACT_DETERMINE_NBEAMS__COPYRIGHT}\index{ EXTRACT DETERMINE NBEAMS !COPYRIGHT}}

Copyright (C) 1998-2001 Particle Physics and Astronomy Research
Council. All Rights Reserved.

%% ORACDRDOC_PRIMITIVE:_EXTRACT_FIND_ROWS_
\paragraph*{\_EXTRACT\_FIND\_ROWS\_\label{_EXTRACT_FIND_ROWS_}\index{\ EXTRACT\ FIND\ ROWS\ }}



\subparagraph*{DESCRIPTION\label{_EXTRACT_FIND_ROWS__DESCRIPTION}\index{ EXTRACT FIND ROWS !DESCRIPTION}}

Finds the rows in a group image at which to centre the spectra 
extraction windows.



These are stored in a user-header called BEAMS, which is a reference
to an array of references to hashes, each hash having keys POS and
MULT - the beam position and multiplier

\subparagraph*{PARAMETERS\label{_EXTRACT_FIND_ROWS__PARAMETERS}\index{ EXTRACT FIND ROWS !PARAMETERS}}

The ROWSCAL parameter - if set to 'SAVE', the primitive will store the
beam information it detects to the calibration rows system. If set to
'LOAD', the primitive will ask the Cal system for the beam information
rather than trying to determine it from the data. If unset, the beam
information will be determined from the image, and used without being
stored to the Cal system.

\subparagraph*{AUTHOR\label{_EXTRACT_FIND_ROWS__AUTHOR}\index{ EXTRACT FIND ROWS !AUTHOR}}

Paul Hirst <p.hirst@jach.hawaii.edu>

\subparagraph*{COPYRIGHT\label{_EXTRACT_FIND_ROWS__COPYRIGHT}\index{ EXTRACT FIND ROWS !COPYRIGHT}}

Copyright (C) 1998-2001 Particle Physics and Astronomy Research
Council. All Rights Reserved.

%% ORACDRDOC_PRIMITIVE:_EXTRACT_ALL_BEAMS_
\paragraph*{\_EXTRACT\_ALL\_BEAMS\_\label{_EXTRACT_ALL_BEAMS_}\index{\ EXTRACT\ ALL\ BEAMS\ }}



\subparagraph*{DESCRIPTION\label{_EXTRACT_ALL_BEAMS__DESCRIPTION}\index{ EXTRACT ALL BEAMS !DESCRIPTION}}

Optimally extracts all the beams in a group file

\subparagraph*{PARAMETERS\label{_EXTRACT_ALL_BEAMS__PARAMETERS}\index{ EXTRACT ALL BEAMS !PARAMETERS}}

PROFILECAL: [LOAD$|$SAVE] - determines whether to get the profiles from the cal system, save the profiles to the cal system, or niether.

\subparagraph*{AUTHOR\label{_EXTRACT_ALL_BEAMS__AUTHOR}\index{ EXTRACT ALL BEAMS !AUTHOR}}

Paul Hirst <p.hirst@jach.hawaii.edu>

\subparagraph*{COPYRIGHT\label{_EXTRACT_ALL_BEAMS__COPYRIGHT}\index{ EXTRACT ALL BEAMS !COPYRIGHT}}

Copyright (C) 1998-2001 Particle Physics and Astronomy Research
Council. All Rights Reserved.

%% ORACDRDOC_PRIMITIVE:_DERIPPLE_ALL_BEAMS_
\paragraph*{\_DERIPPLE\_ALL\_BEAMS\_\label{_DERIPPLE_ALL_BEAMS_}\index{\ DERIPPLE\ ALL\ BEAMS\ }}



\subparagraph*{DESCRIPTION\label{_DERIPPLE_ALL_BEAMS__DESCRIPTION}\index{ DERIPPLE ALL BEAMS !DESCRIPTION}}

Exactly what it says on the can.

\subparagraph*{AUTHOR\label{_DERIPPLE_ALL_BEAMS__AUTHOR}\index{ DERIPPLE ALL BEAMS !AUTHOR}}

Paul Hirst <p.hirst@jach.hawaii.edu>

\subparagraph*{COPYRIGHT\label{_DERIPPLE_ALL_BEAMS__COPYRIGHT}\index{ DERIPPLE ALL BEAMS !COPYRIGHT}}

Copyright (C) 1998-2001 Particle Physics and Astronomy Research
Council. All Rights Reserved.

%% ORACDRDOC_PRIMITIVE:_CROSS_CORR_ALL_BEAMS_
\paragraph*{\_CROSS\_CORR\_ALL\_BEAMS\label{_CROSS_CORR_ALL_BEAMS}\index{\ CROSS\ CORR\ ALL\ BEAMS}}

Cross correlates and shifts the extracted beams

\subparagraph*{DESCRIPTION\label{_CROSS_CORR_ALL_BEAMS_DESCRIPTION}\index{ CROSS CORR ALL BEAMS!DESCRIPTION}}

Takes the extracted beams from \_EXTRACT\_ALL\_BEAMS\_ and cross
correlates each beam with the first one, then shifts each beam, so
that they're all shift-aligned with the first beam.

\subparagraph*{AUTHOR\label{_CROSS_CORR_ALL_BEAMS_AUTHOR}\index{ CROSS CORR ALL BEAMS!AUTHOR}}

Paul Hirst <p.hirst@jach.hawaii.edu>

\subparagraph*{COPYRIGHT\label{_CROSS_CORR_ALL_BEAMS_COPYRIGHT}\index{ CROSS CORR ALL BEAMS!COPYRIGHT}}

Copyright (C) 1998-2001 Particle Physics and Astronomy Research
Council. All Rights Reserved.

%% ORACDRDOC_PRIMITIVE:_COADD_EXTRACTED_BEAMS_
\paragraph*{\_COADD\_EXTRACTED\_BEAMS\_\label{_COADD_EXTRACTED_BEAMS_}\index{\ COADD\ EXTRACTED\ BEAMS\ }}

Coadds the beams which were previously extracted

\subparagraph*{DESCRIPTION\label{_COADD_EXTRACTED_BEAMS__DESCRIPTION}\index{ COADD EXTRACTED BEAMS !DESCRIPTION}}

Adds together the beams in the group file. Normally, these will have been aligned
using \_CROSS\_CORR\_ALL\_BEAMS\_ first.

\subparagraph*{AUTHOR\label{_COADD_EXTRACTED_BEAMS__AUTHOR}\index{ COADD EXTRACTED BEAMS !AUTHOR}}

Paul Hirst <p.hirst@jach.hawaii.edu>

\subparagraph*{COPYRIGHT\label{_COADD_EXTRACTED_BEAMS__COPYRIGHT}\index{ COADD EXTRACTED BEAMS !COPYRIGHT}}

Copyright (C) 1998-2001 Particle Physics and Astronomy Research
Council. All Rights Reserved.


\subsubsection{Using Standard Stars}

%% ORACDRDOC_PRIMITIVE:_DIVIDE_BY_STANDARD_
\paragraph*{\_DIVIDE\_BY\_STANDARD\_\label{_DIVIDE_BY_STANDARD_}\index{\ DIVIDE\ BY\ STANDARD\ }}



\subparagraph*{DESCRIPTION\label{_DIVIDE_BY_STANDARD__DESCRIPTION}\index{ DIVIDE BY STANDARD !DESCRIPTION}}

Asks the calibration system for a suitble standard star, and divides by it.

\subparagraph*{NOTES\label{_DIVIDE_BY_STANDARD__NOTES}\index{ DIVIDE BY STANDARD !NOTES}}

The input frames should have been normalised to 1 second exposures.

\subparagraph*{AUTHOR\label{_DIVIDE_BY_STANDARD__AUTHOR}\index{ DIVIDE BY STANDARD !AUTHOR}}

Paul Hirst <p.hirst@jach.hawaii.edu>

\subparagraph*{COPYRIGHT\label{_DIVIDE_BY_STANDARD__COPYRIGHT}\index{ DIVIDE BY STANDARD !COPYRIGHT}}

Copyright (C) 1998-2001 Particle Physics and Astronomy Research
Council. All Rights Reserved.

%% ORACDRDOC_PRIMITIVE:_ALIGN_SPECTRUM_TO_STD_
\paragraph*{\_ALIGN\_SPECTRUM\_TO\_STD\_\label{_ALIGN_SPECTRUM_TO_STD_}\index{\ ALIGN\ SPECTRUM\ TO\ STD\ }}

Cross correlate and shift before divide by std

\subparagraph*{DESCRIPTION\label{_ALIGN_SPECTRUM_TO_STD__DESCRIPTION}\index{ ALIGN SPECTRUM TO STD !DESCRIPTION}}

Cross correlates the spectrum with the standard star spectrum and shifts
it, so as to get better atmospheric cancelation if the instrument has
flexed between the standard and the target observations.



Takes a STANDARD parameter telling it the name of the std to use.



Doesn't do anything unless the group NDIMS is 1

\subparagraph*{AUTHOR\label{_ALIGN_SPECTRUM_TO_STD__AUTHOR}\index{ ALIGN SPECTRUM TO STD !AUTHOR}}

Paul Hirst <p.hirst@jach.hawaii.edu>

\subparagraph*{COPYRIGHT\label{_ALIGN_SPECTRUM_TO_STD__COPYRIGHT}\index{ ALIGN SPECTRUM TO STD !COPYRIGHT}}

Copyright (C) 1998-2001 Particle Physics and Astronomy Research
Council. All Rights Reserved.

%% ORACDRDOC_PRIMITIVE:_FLUX_CALIBRATE_
\paragraph*{\_FLUX\_CALIBRATE\_STD\_\label{_FLUX_CALIBRATE_STD_}\index{\ FLUX\ CALIBRATE\ STD\ }}



\subparagraph*{DESCRIPTION\label{_FLUX_CALIBRATE_STD__DESCRIPTION}\index{ FLUX CALIBRATE STD !DESCRIPTION}}

First phase of flux cal - get the std parameters

\subparagraph*{AUTHOR\label{_FLUX_CALIBRATE_STD__AUTHOR}\index{ FLUX CALIBRATE STD !AUTHOR}}

Paul Hirst <p.hirst@jach.hawaii.edu>

\subparagraph*{COPYRIGHT\label{_FLUX_CALIBRATE_STD__COPYRIGHT}\index{ FLUX CALIBRATE STD !COPYRIGHT}}

Copyright (C) 1998-2001 Particle Physics and Astronomy Research
Council. All Rights Reserved.


\subsection{Historical Recipe Names}

Historical data taken with the main historical recipes, SOURCE\_PAIRS\_ON\_SLIT
and SOURCE\_PAIRS\_TO\_SKY should now be reduced with POINT\_SOURCE or
EXTENDED\_SOURCE as appropriate. Use POINT\_SOURCE if in douubt.

Hitorical data taken with SOURCE\_WITH\_NOD\_TO\_BLANK\_SKY should now
be reduced with EXTENDED\_SOURCE\_WITH\_SEPARATE\_SKY.

% ? End of main text
\end{document}
